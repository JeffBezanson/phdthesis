\chapter{Case studies}
\label{chap:casestudies}

This chapter discusses several examples that illustrate the effectiveness of
Julia's abstractions for technical computing.
The first three sections provide an ``explication through elimination'' of
core features (numbers and arrays) that have usually been built in to
technical computing environments.
The remaining sections introduce some libraries and real-world applications
that benefit from Julia's design.

\section{Conversion and comparison}
\label{sec:conversion}

Type conversion provides a classic example of a binary method.
Multiple dispatch allows us to avoid deciding whether the converted-to type or the
converted-from type is responsible for defining the operation.
Defining a specific conversion is straightforward, and might look like this
in Julia syntax:

\begin{verbatim}
convert(::Type{Float64}, x::Int32) = ...
\end{verbatim}

\noindent
A call to this method would look like \texttt{convert(Float64, 1)}.

Using conversion in generic code requires more sophisticated definitions.
For example we might need to convert one value to the type of another,
by writing \texttt{convert(typeof(y), x)}.
What set of definitions must exist to make that call work in all reasonable cases?
Clearly we don't want to write all $O(n^2)$ possibilities.
We need abstract definitions that cover many points in the dispatch matrix.
One such family of points is particularly important: those that describe converting
a value to a type it is already an instance of.
In our system this can be handled by a single definition that performs ``triangular''
dispatch:

\begin{verbatim}
convert{T,S<:T}(::Type{T}, x::S) = x
\end{verbatim}

\noindent
``Triangular'' refers to the rough shape of the dispatch matrix covered
by such a definition: for all types \texttt{T} in the first argument slot,
match values of any type less than it in the second argument slot.

A similar trick is useful for defining equivalence relations.
It is most likely unavoidable for programming languages to need multiple
notions of equality.
Two in particular are natural: an \emph{intensional}
notion that equates objects that look identical, and an \emph{extensional}
notion that equates objects that mean the same thing for some standard
set of purposes.
Intensional equality (\texttt{===} in Julia, described
in section~\ref{sec:corecalc}) lends itself to being implemented once
by the language implementer, since it can work by directly comparing
the representations of objects.
Extensional equality (\texttt{==} in Julia), on the other hand, must be
extensible to user-defined data types.
The latter function must call the former in order to have any basis for
its comparisons.

As with conversion, we would like to provide default definitions for
\texttt{==} that cover families of cases.
Numbers are a reasonable domain to pick, since all numbers should be
equality-comparable to each other.
We might try (\texttt{Number} is the abstract supertype of all numeric types):

\begin{singlespace}
\begin{lstlisting}[language=julia]
==(x::Number, y::Number) = x === y
\end{lstlisting}
\end{singlespace}

\noindent
meaning that number comparison can simply fall back to intensional
equality.
However this definition is rather dubious.
It gets the wrong answer every time the arguments are different representations
(e.g.\ integer and floating point) of the same quantity.
We might hope that its behavior will be ``patched up'' later by more specific
definitions for various concrete number types, but it still covers a dangerous
amount of dispatch space.
If later definitions somehow miss a particular combination of number types,
we could get a silent wrong answer instead of an error.
(Note that statically checking method exhaustiveness is no help here.)

``Diagonal'' dispatch lets us improve the definition:

\begin{singlespace}
\begin{lstlisting}[language=julia]
=={T<:Number}(x::T, y::T) = x === y
\end{lstlisting}
\end{singlespace}

\noindent
Now \texttt{===} will only be used on arguments of the same type,
making it far more likely to give the right answer.
Even better, any case where it does not give the right answer can be fixed with
a single definition, i.e.\ \texttt{==(x::S, y::S)} for some
concrete type \texttt{S}.
The more general \texttt{(Number, Number)} case is left open, and in the next
section we will take advantage of this to implement ``automatic'' type promotion.

%- The abstractions of equality and comparison. Different equivalence classes between
%is/===, isequal and ==
%- Numeric vs lexicographic ordering?
%cmp, lexcmp, vs isless, <


\section{Numeric types and embeddings}

We might prefer ``number'' to be a single, concrete concept, but the history of
mathematics has seen the concept extended many times, from integers to rationals to
reals, and then to complex, quaternion, and more.
These constructions tend to follow a pattern: a new set of numbers is constructed
around a subset isomorphic to an existing set of numbers.
For example, the reals are isomorphic to the complex numbers with zero imaginary part.

Human beings happen to be good at equating and moving between isomorphic sets,
so it is easy to imagine that the reals and complexes with zero imaginary
part are one and the same.
But a computer forces us to be specific, and admit
that a real number is not complex, and a complex number is not real.
And yet the close relationship between them is too compelling not to model in a
computer somehow.
Here we have a numerical analog to the famous ``circle and ellipse'' problem in
object-oriented programming~\cite{cline1995c++}: the set of circles is
isomorphic to the set of ellipses with equal axes, yet neither ``is a''
relationship in a class hierarchy seems fully correct.
An ellipse is not a circle, and in general a circle cannot serve as an ellipse
(for example, the set of circles is not closed under the same operations that the
set of ellipses is, so a program written for ellipses might not work on circles).
This problem implies that a single built-in type hierarchy is not
sufficient: we want to model custom \emph{kinds} of relationships between
types (e.g.\ ``can be embedded in'' in addition to ``is a'').

%% Two further problems should also be kept in mind. First, the natural isomorphisms
%% between sets of numbers might not be isomorphisms on a real computer. For example,
%% due to the behavior of floating-point arithmetic, an operation on complex numbers
%% with zero imaginary part might not give an answer equal to the same operation on
%% real numbers. Second, the contexts that demand use of one type of number or
%% another are often not easily described by type systems.

% ex: generic sum function accumulating result in at least
% double precision. just using +::T->T->T doesn't work.

% \subsection{Current approaches}

Numbers tend to be among the most complex features of a language.
Numeric types usually need to be a special case: in a typical language with
built-in numeric types, describing their behavior is beyond the expressive power
of the language itself.
For example, in C arithmetic operators like \texttt{+} accept multiple types of
arguments (ints and floats), but no user-defined C function can do this (the situation
is of course improved in C++).
In Python, a special arrangement is made for \texttt{+} to call either an
\texttt{\_\_add\_\_} or \texttt{\_\_radd\_\_} method,
effectively providing double-dispatch for arithmetic in a language that is
idiomatically single-dispatch.

\subsubsection{Implementing type embeddings}
\label{sec:promotion}

Most functions are naturally implemented in the value domain, but some are
actually easier to implement in the type domain.
One reason is that there is a bottom element, which most data types lack.

It has been suggested on theoretical grounds \cite{categorytheoryoperators}
that generic binary operators should have ``key variants'' where the
arguments are of the same type.
We implement this in Julia with a default definition that uses diagonal
dispatch:

\begin{singlespace}
\begin{lstlisting}[language=julia]
+{T<:Number}(x::T, y::T) = no_op_err("+", T)
\end{lstlisting}
\end{singlespace}

Then we can implement a more general definition for different-type arguments
that tries to promote the arguments to a common type by calling \texttt{promote}:

\begin{singlespace}
\begin{lstlisting}[language=julia]
+(x::Number, y::Number) = +(promote(x,y)...)
\end{lstlisting}
\end{singlespace}

\noindent
\texttt{promote} returns a pair of the values of its arguments after conversion
to a common type, so that a ``key variant'' can be invoked.

\texttt{promote} is designed to be extended by new numeric types.
A full-featured promotion operator is a tall order.
We would like

\begin{itemize}
\item Each combination of types only needs to be defined in one order; we
don't want to redundantly define the behavior of \texttt{(T,S)} and \texttt{(S,T)}.
\item It falls back to join for types without any defined promotion.
\item It must prevent promotion above a certain point to avoid circularity.
%in promoting fallback definitions of operators.
\end{itemize}

The core of the mechanism is three functions: \texttt{promote\_type}, which
performs promotion in the type domain only, \texttt{promote\_rule}, which is
the function defined by libraries, and a utility function \texttt{promote\_result}.
The default definition of \texttt{promote\_rule} returns \texttt{Bottom},
indicating no promotion is defined.
\texttt{promote\_type} calls \texttt{promote\_rule} with its arguments in
both possible orders.
If one order is defined to give \texttt{X} and the other is not defined,
\texttt{promote\_result} recursively promotes \texttt{X} and \texttt{Bottom},
giving \texttt{X}.
If neither order is defined, the last definition below is called, which
falls back to \texttt{typejoin}:

\begin{singlespace}
\begin{lstlisting}[language=julia]
promote{T,S}(x::T, y::S) =
    (convert(promote_type(T,S),x), convert(promote_type(T,S),y))

promote_type{T,S}(::Type{T}, ::Type{S}) =
    promote_result(T, S, promote_rule(T,S), promote_rule(S,T))

promote_rule(T, S) = Bottom

promote_result(t,s,T,S) = promote_type(T,S)
# If no promote_rule is defined, both directions give Bottom.
# In that case use typejoin on the original types instead.
promote_result{T,S}(::Type{T}, ::Type{S},
                    ::Type{Bottom}, ::Type{Bottom}) = typejoin(T, S)
\end{lstlisting}
\end{singlespace}

The obvious way to extend this system is to define \texttt{promote\_rule},
but it can also be extended in a less obvious way.
For the purpose of manipulating container types, we would like e.g.\ \texttt{Int}
and \texttt{Real} to promote to \texttt{Real}.
However, we do not want to introduce a rule that promotes arbitrary pairs of
numeric types to \texttt{Number}, since that would make the above
default definition of \texttt{+} circular.
The following definitions accomplish this, by promoting to the larger
numeric type if one is a subtype of the other:

\begin{singlespace}
\begin{lstlisting}[language=julia]
promote_result{T<:Number,S<:Number}(::Type{T}, ::Type{S},
                                    ::Type{Bottom}, ::Type{Bottom}) =
    promote_sup(T, S, typejoin(T,S))

# promote numeric types T and S to typejoin(T,S) if T<:S or S<:T
# for example this makes promote_type(Integer,Real) == Real without
# promoting arbitrary pairs of numeric types to Number.
promote_sup{T<:Number          }(::Type{T},::Type{T},::Type{T}) = T
promote_sup{T<:Number,S<:Number}(::Type{T},::Type{S},::Type{T}) = T
promote_sup{T<:Number,S<:Number}(::Type{T},::Type{S},::Type{S}) = S
promote_sup{T<:Number,S<:Number}(::Type{T},::Type{S},::Type) =
    error("no promotion exists for ", T, " and ", S)
\end{lstlisting}
\end{singlespace}


\subsubsection{Application to ranges}

\emph{Ranges} illustrate an interesting application of type promotion.
A range data type, notated \texttt{a:s:b}, represents a sequence of values
starting at \texttt{a} and ending at \texttt{b}, with a distance of \texttt{s}
between elements (internally, this notation is translated to
\texttt{colon(a, s, b)}).
Ranges seem simple enough, but a reliable, efficient, and generic implementation
is difficult to achieve.
We propose the following requirements:

\begin{itemize}
\item The start and stop values can be passed as different types, but internally
  should be of the same type.
\item Ranges should work with ordinal types, not just numbers (examples include
  characters, pointers, and calendar dates).
\item If any of the arguments is a floating-point number, a special
  \texttt{FloatRange} type designed to cope well with roundoff is returned.
\end{itemize}

In the case of ordinal types, the step value is naturally of a different type
than the elements of the range.
For example, one may add 1 to a character to get the ``next'' encoded character,
but it does not make sense to add two characters.

It turns out that the desired behavior can be achieved with six definitions.

First, given three floats of the same type we can construct a \texttt{FloatRange}
right away:

\begin{verbatim}
colon{T<:FloatingPoint}(start::T, step::T, stop::T) =
    FloatRange{T}(start, step, stop)
\end{verbatim}

Next, if \texttt{a} and \texttt{b} are of the same type and there are no floats,
we can construct a general range:

\begin{verbatim}
colon{T}(start::T, step, stop::T) = StepRange(start, step, stop)
\end{verbatim}

Now there is a problem to fix: if the first and last arguments are of some
non-floating-point numeric type, but the step is floating point, we want to
promote all arguments to a common floating point type.
We must also do this if the first and last arguments are floats, but the step
is some other kind of number:

\begin{verbatim}
colon{T<:Real}(a::T, s::FloatingPoint, b::T) = colon(promote(a,s,b)...)

colon{T<:FloatingPoint}(a::T, s::Real, b::T) = colon(promote(a,s,b)...)
\end{verbatim}

These two definitions are correct, but ambiguous: if the step is a float
of a different type than \texttt{a} and \texttt{b} both definitions are
equally applicable.
We can add the following disambiguating definition:

\begin{verbatim}
colon{T<:FloatingPoint}(a::T, s::FloatingPoint, b::T) =
    colon(promote(a,s,b)...)
\end{verbatim}

All of these five definitions require \texttt{a} and \texttt{b} to be of the
same type.
If they are not, we must promote just those two arguments, and leave
the step alone (in case we are dealing with ordinal types):

\begin{verbatim}
colon{A,B}(a::A, s, b::B) =
    colon(convert(promote_type(A,B),a), s, convert(promote_type(A,B),b))
\end{verbatim}

This example shows that it is not always sufficient to have a built-in set of
``promoting operators''.
Library functions like this \texttt{colon} need more control.

\subsubsection{On implicit conversion}

In order to facilitate human styles of thinking, many programming languages offer
some flexibility in when types are required to match.
%This often takes the form of implicit conversion.
For example, a C function declared to accept type \texttt{float}
can also be called with an \texttt{int}, and the compiler will
insert a conversion automatically.
We have found that most cases where this behavior is desirable
are handled simply by method applicability, as shown in this and the
previous section.
Most of the remaining cases involve assignment.
Julia has a notion of a \emph{typed location}, which is a variable
or mutable object field with an associated type.
The core language requires types to match (as determined by subtyping)
on assignment.
However, updating assignments that are syntactically identifiable
(via use of \texttt{=}) cause the front-end to insert calls to
\texttt{convert} to try to convert the assignment right-hand side
to the type of the assigned location.
Abstract data types can easily mimic this behavior by (optionally)
calling \texttt{convert} in their implementations of mutating
operations.
In our experience, this arrangement provides the needed convenience
without complicating the language.
Program analyses consuming lowered syntax trees (or any lower level
representation) do not need to know anything about conversion
or coercion.

Fortress~\cite{fortresspec} also provides powerful type conversion
machinery.
However, it does so through a special \texttt{coerce} feature.
This allows conversions to happen during dispatch, so a method can
become applicable if arguments can be converted to its declared type.
There is also a built-in notion of type widening.
This allows conversions to be inserted between nested operator
applications, which can improve the accuracy of mixed-precision code.
It is not possible to provide these features in Julia without doing
violence to our ``everything is a method call'' semantics.
We feel that it is worth giving up these features for a simpler
mental model.
We also believe that the ability to provide so much ``magic'' numeric
type behavior only through method definitions is interesting and novel.

\subsubsection{Number-like types in practice}

Originally, our reasons for implementing all numeric types at the library
level were not entirely practical.
We had a principled opposition to including such definitions in a compiler,
and guessed that being able to define numeric types would help ensure the
language was powerful enough.
However, defining numeric and number-like types and their interactions turns
out to be surprisingly useful.
Once such types are easy to obtain, people find more and more uses for them.

Even among basic types that might reasonably be built in, there is enough
complexity to require an organizational strategy.
We might have

\begin{itemize}
  \item Ordinal types \texttt{Pointer}, \texttt{Char}, \texttt{Enum}

  \item Integer types \texttt{Bool}, \texttt{Int8}, \texttt{Int16}, \texttt{Int32}, \texttt{Int64}, \texttt{Int128}, \texttt{UInt8}, \texttt{UInt16}, \texttt{UInt32}, \texttt{UInt64}, \texttt{UInt128}, \texttt{BigInt}

  \item Floating point types \texttt{Float16}, \texttt{Float32}, \texttt{Float64}, \texttt{Float80}$^*$, \texttt{Float128}$^*$, \texttt{BigFloat}, \texttt{DoubleDouble}

  \item Extensions \texttt{Rational}, \texttt{Complex}, \texttt{Quaternion}
\end{itemize}

\noindent
Types with $^*$ have not been implemented yet, but the rest have.
In external packages, there are types for interval arithmetic,
dual and hyper-dual numbers for computing first and second derivatives,
and finite fields.

Some applications benefit in performance from fixed-point arithmetic.
This has been implemented in a package as \texttt{Fixed32\{b\}}, where
the number of fraction bits is a parameter.

A problem in the design of image processing libraries was solved by
defining a new kind of fixed-point type \cite{ufixed}.
The problem is that image scientists often want to work with fractional
pixel values in the interval $[0,1]$, but most graphics libraries (and memory
efficiency concerns) require 8-bit integer pixel components with
values in the interval $[0,255]$.
The solution is a \texttt{Ufixed8} type that uses an unsigned 8-bit integer
as its representation, but behaves like a fraction over $255$.

Many real-world quantities are not numbers exactly, but benefit from the
same mechanisms in their implementation.
Examples include colors (which form a vector space, and where many different
useful bases have been standardized), physical units, and DNA nucleotides.
Date and time arithmetic is especially intricate and irregular, and
benefits from permissiveness in operator definitions.
% (more on this in section~\ref{sec:dates}).
% dates: different cardinal and ordinal behavior
% musical notes


\section{Multidimensional array indexing}
\label{sec:ndindexing}

One-dimensional arrays are a simple and essential data structure found in
most programming languages.
The multi-dimensional arrays required in scientific computing, however, are
a different beast entirely.
Allowing any number of dimensions entails a significant increase in complexity.
Why?
The essential reason is that core properties of the data structure no
longer fit in a constant amount of space.
The space needed to store the sizes of the dimensions (the array shape) is
proportional to the number of dimensions.
This does not seem so bad, but becomes a large problem due to three additional
facts:

\begin{enumerate}
\item Code that operates on the dimension sizes needs to be highly efficient.
Typically the overhead of a loop is unacceptable, and such code needs to be fully
unrolled.
\item In some code the number of dimensions is a \emph{dynamic} property --- it is
only known at run time.
\item Programs may wish to treat arrays with different numbers of dimensions very
differently.
A vector (1d) might have rather different behaviors than a matrix (2d)
(for example, when computing a norm).
This kind of behavior makes the number of dimensions a crucial part of program
semantics, preventing it from remaining a compiler implementation detail.
\end{enumerate}

These facts pull in different directions.
The first fact asks for static analysis.
The second fact asks for run time flexibility.
The third fact asks for dimensionality to be part of the type system, but
partly determined at run time (for example, via virtual method dispatch).
Current approaches choose a compromise.
In some systems, the number of dimensions has a strict limit (e.g.\ 3 or 4),
%% TODO: examples of systems limited to n==3
so that separate classes for each case can be written out in full.
Other systems choose flexibility, and just accept that most
or all operations will be dynamically dispatched.
Other systems might provide flexibility only at compile time, for example a
template library where the number of dimensions must be statically known.

Whatever tradeoff is made, rules must be defined for how various operators
act on dimensions.
Here we focus on indexing, since selecting parts of arrays has particularly
rich behavior with respect to dimensionality.
For example, if a single row or column of a matrix is
selected, does the result have one or two dimensions?
Array implementations prefer to invoke general rules to answer such questions.
Such a rule might say ``dimensions indexed with scalars are dropped'', or ``trailing
dimensions of size one are dropped'', or ``the rank of the result
is the sum of the ranks of the indexes'' (as in APL).
A significant amount of work has been done on inferring properties like these
for existing array-based languages
(e.g.\ \cite{Joisha:2006:AAS:1152649.1152651,Garg:2014:JSI:2627373.2627382},
although these methods go farther and attempt to infer complete size
information).

C++ and Haskell are both examples of languages with sufficiently powerful
static semantics to support defining efficient and reasonably flexible
multi-dimensional arrays in libraries.
In both languages, implementing these libraries is fairly difficult and
places some limits on how indexing can behave.
In C++, the popular Boost libraries include a multi-array
type~\cite{garcia2005multiarray}.
To index an array with this library, one builds up an \texttt{index\_gen}
object by repeatedly applying the \texttt{[]} operator.
On each application, if the argument is a \texttt{range} then the
corresponding dimension is kept, and otherwise it is dropped.
This is implemented with a template as follows:

\begin{singlespace}
\begin{lstlisting}[language=c++,style=ttcode]
template <int NumRanges, int NumDims>
struct index_gen {
index_gen<NumRanges+1,NumDims+1> operator[](const range& r) const
    { ... }

index_gen<NumRanges+1,NumDims> operator[](index idx) const
    { ... }
}
\end{lstlisting}
\end{singlespace}

\noindent
The number of dimensions in the result is determined by arithmetic
on template parameters and static overloading.
Handling arguments one at a time recursively is a common pattern
for feeding more complex type information to compilers.
In fact, the Haskell library Repa~\cite{Keller:2010rs} uses
essentially the same approach:

\begin{singlespace}
\begin{lstlisting}[language=haskell,style=ttcode]
instance Slice sl => Slice (sl :. Int) where
    sliceOfFull (fsl :. _) (ssl :. _) = sliceOfFull fsl ssl

instance Slice sl => Slice (sl :. All) where
    sliceOfFull (fsl :. All) (ssl :. s) = sliceOfFull fsl ssl :. s
\end{lstlisting}
\end{singlespace}

\noindent
The first instance declaration says that given an \texttt{Int} index,
the dimension corresponding to \texttt{\_} is dropped.
The second declaration says that given an \texttt{All} index,
dimension \texttt{s} is kept.

\iffalse
Our goal here is a bit unusual: we are not concerned with which rules
might work best, but merely with how they can be specified, so that
domain experts can experiment.
In fact different domains want different things.
Working with images, each dimension might be quite different, e.g.\ representing
time, space, or color, so you don't want to drop or rearrange dimensions very often.
\fi

%Here are our ground rules:
%\begin{enumerate}
%\item You can't manually implement the behavior inside the compiler
%\item The compiler must be able to reasonably understand the program
%\item The code must be reasonably easy to write
%\end{enumerate}

\iffalse
%How are such rules implemented?
In a language with built-in multidimensional arrays, the compiler will
analyze indexing expressions and determine an answer using hard-coded
logic.
% TODO: example from one of these compilers
However, this approach is not satisfying: we would rather
implement the behavior in libraries, so that different kinds of arrays
may be defined, or so that rules of similar complexity may be
defined for other kinds of objects.
In fact different domains want different things.
Working with images, each dimension might be quite different, e.g.\ representing
time, space, or color, so you don't want to drop or rearrange dimensions very often.
%But these kinds of rules are unusually difficult to implement in libraries.
%If a library writes out its indexing logic using imperative code, the host
%language compiler is not likely to be able to analyze it.
Using compile time abstracton (templates) provides better performance, but
such libraries tend to be difficult to write (and read), and the full
complement of indexing behavior expected by technical users strains the
capabilities of such systems.
\fi

Our dispatch mechanism permits a novel solution \cite{Bezanson2014}.
Indexing behavior can be defined with method signatures, using a combination
of variadic methods and argument ``splatting'' (the ability
to pass a structure of $n$ values as $n$ separate arguments to a function,
known as \texttt{apply} in Lisp dialects).
This solution is still a compromise among the factors outlined above,
but it is a new compromise that provides a modest increment of power.
% TODO more

Below we define a function \texttt{index\_shape} that computes the
shape of a result array given a series of index arguments.
We show three versions, each implementing a different rule that users in
different domains might want:

% TODO: point out array = (shape, data), so those are the two parts
% we need to handle. In julia the ``data'' part is not a first class
% object; it is not directly exposed to the user, but this is more
% of an implementation detail.

\vspace{-3ex}
\begin{singlespace}
\begin{verbatim}
# drop dimensions indexed with scalars
index_shape() = ()
index_shape(i::Real, I...) = index_shape(I...)
index_shape(i, I...)       = (length(i), index_shape(I...)...)

# drop trailing dimensions indexed with scalars
index_shape(i::Real...) = ()
index_shape(i, I...)    = (length(i), index_shape(I...)...)

# rank summing (APL)
index_shape()        = ()
index_shape(i, I...) = (size(i)..., index_shape(I...)...)
\end{verbatim}
\end{singlespace}

Inferring the length of the result of \texttt{index\_shape} is sufficient
to infer the rank of the result array.

These definitions are concise, easy to write, and possible for a
compiler to understand fully using straightforward techniques.

% TODO: point out how this combines the ``object part'' and the
% ``array part'' into a coherent whole.

%% Here is a sample derivation for the call \texttt{index\_shape(1:m,1,1:n)}
%% (the argument type tuple is \texttt{(Range1,Int,Range1)}), using the first
%% definition above (dropping scalar-indexed dimensions):

%% \begin{verbatim}
%% index_shape(1:n, 1, 1:m) => tuple(length(::Range1)::Int, index_shape((::Int, ::Range1)...)...)
%% index_shape((::Int, ::Range1)...) => index_shape(::Int, ::Range1)
%% index_shape(::Int, ::Range1) => index_shape((::Range1,)...)
%% index_shape((::Range1,)...) => index_shape(::Range1)
%% index_shape(::Range1) => tuple(length(::Range1)::Int, index_shape(()...)...)
%% index_shape(()...) => index_shape()::()
%% back substitute => tuple(length(::Range1)::Int, index_shape(()...)::()...)::(Int,)
%% back substitute => tuple(length(::Range1)::Int, ::(Int,)...)
%% tuple(::Int, ::(Int,)...) => tuple(::Int, ::Int)
%% ::(Int, Int)
%% \end{verbatim}

The result type is determined using only data flow type inference, plus a
rule for splicing an immediate container (the type of \texttt{f((a,b)...)} is
the type of \texttt{f(a,b)}). Argument list destructuring takes place inside
the type intersection operator used to combine argument types with method
signatures.

This approach does not depend on any heuristics. Each call to \texttt{index\_shape}
simply requires one recursive invocation of type inference. This process reaches
the base case \texttt{()} for these definitions, since each recursive call
handles a shorter argument list (for less-well-behaved definitions, we might
end up invoking a widening operator instead).

%% \subsection{Even more elimination?}

%% Some features of the language could be even further eliminated. For example data
%% types could be implemented in terms of lambda abstractions. But certain patterns
%% are so useful that they might as well be provided in a standard form. It also
%% probably makes the compiler much more efficient not to need to pass around and
%% repeatedly analyze full representations of the meanings of such ubiquitous constructs.

\iffalse
\section{Array views}

tim holy in issue 8839:

``without staged functions in my initial post in 8235. The take-home message: generating all methods through dimension 8 resulted in more than 5000 separate methods, and required over 4 minutes of parsing \& lowering time (i.e., a 4-minute delay while compiling julia). By comparison, the stagedfunction implementation loads in a snap, and of course can go even beyond 8 dimensions.''
\fi

\section{Numerical linear algebra}
\label{sec:linalg}

The crucial roles of code selection and code specialization in technical
computing are captured well in the linear algebra software engineer's dilemma,
described in the context of LAPACK and ScaLAPACK by Demmel and Dongarra,
et.al. \cite{lawn181}:
\vspace{-3ex}
\begin{singlespace}
\begin{small}
\begin{verbatim}
(1) for all linear algebra problems (linear systems, eigenproblems, ...)
(2)    for all matrix types (general, symmetric, banded, ...)
(3)       for all data types (real, complex, single, double, higher precision)
(4)          for all machine architectures
                and communication topologies
(5)                for all programming interfaces
(6)                   provide the best algorithm(s) available in terms of
                      performance and accuracy ("algorithms" is plural
                      because sometimes no single one is always best)
\end{verbatim}
\end{small}
\end{singlespace}

\iffalse
Multiple dispatch on special matrices

29 LAPACK types via composition of 9 types, issue 8240

storage formats:
Matrix
Banded
Packed
RFP (rectangular full packed, LAWN 199)

symmetry labels:
Hermitian
Hessenberg
Symmetric
Triangular
Trapezoid

Bidiagonal{T} <: AbstractMatrix{T}
Diagonal{T} <: AbstractMatrix{T}
SymmetricRFP{T<:BlasFloat} <: AbstractMatrix{T}
TriangularRFP{T<:BlasFloat} <: AbstractMatrix{T}
Symmetric{T,S<:AbstractMatrix} <: AbstractMatrix{T}
Hermitian{T,S<:AbstractMatrix} <: AbstractMatrix{T}
AbstractTriangular{T,S<:AbstractMatrix} <: AbstractMatrix{T}
LowerTriangular{T,S<:AbstractMatrix} <: AbstractTriangular{T,S}
UnitLowerTriangular{T,S<:AbstractMatrix} <: AbstractTriangular{T,S}
UpperTriangular{T,S<:AbstractMatrix} <: AbstractTriangular{T,S}
UnitUpperTriangular{T,S<:AbstractMatrix} <: AbstractTriangular{T,S}
SymTridiagonal{T} <: AbstractMatrix{T}
Tridiagonal{T} <: AbstractMatrix{T}
UniformScaling{T<:Number}

abstract Factorization{T}
Cholesky{T,S<:AbstractMatrix} <: Factorization{T}
CholeskyPivoted{T,S<:AbstractMatrix} <: Factorization{T}
QR{T,S<:AbstractMatrix} <: Factorization{T}
QRCompactWY{S,M<:AbstractMatrix} <: Factorization{S}
QRPivoted{T,S<:AbstractMatrix} <: Factorization{T}
QRPackedQ{T,S<:AbstractMatrix} <: AbstractMatrix{T}
QRPackedWYQ{S,M<:AbstractMatrix} <: AbstractMatrix{S}
QRCompactWYQ{S, M<:AbstractMatrix} <: AbstractMatrix{S}
Hessenberg{T,S<:AbstractMatrix} <: Factorization{T}
HessenbergQ{T,S<:AbstractMatrix} <: AbstractMatrix{T}
Eigen{T,V,S<:AbstractMatrix,U<:AbstractVector} <: Factorization{T}
GeneralizedEigen{T,V,S<:AbstractMatrix,U<:AbstractVector} <: Factorization{T}
SVD{T<:BlasFloat,Tr,M<:AbstractArray} <: Factorization{T}
Schur{Ty<:BlasFloat, S<:AbstractMatrix} <: Factorization{Ty}
GeneralizedSchur{Ty<:BlasFloat, M<:AbstractMatrix} <: Factorization{Ty}
LDLt{T,S<:AbstractMatrix} <: Factorization{T}
LU{T,S<:AbstractMatrix} <: Factorization{T}
CholeskyDenseRFP{T<:BlasFloat} <: Factorization{T}
\fi

Describing the applicability of BLAS:

\begin{singlespace}
\begin{lstlisting}[language=julia]
const BlasFloat = Union(Float64,Float32,Complex128,Complex64)

axpy!{T<:BlasFloat}(alpha::Number,
                    x::Union(DenseArray{T},StridedVector{T}),
                    y::Union(DenseArray{T},StridedVector{T}))
\end{lstlisting}
\end{singlespace}

\iffalse
QR factorization: want to support it on all input types, but many
types (integer, rational) are not closed under the needed operations.
compare to eigen: in C++, if you want a qrfact of an integer matrix,
you might get a compile time error, or it might work but all the values
will be truncated when stored. this is a big generic programming
challenge. you can't expect types to have a ``type to use for QR fact''
trait.

qrfact computes the type of the result needed:

\begin{singlespace}
\begin{lstlisting}[language=julia]
function qrfact{T}(A::StridedMatrix{T}; pivot=false)
    S = typeof(one(T)/norm(one(T)))
    if S != T
        qrfact!(convert(AbstractMatrix{S},A), pivot=pivot)
    else
        qrfact!(copy(A),pivot=pivot))
    end
end
\end{lstlisting}
\end{singlespace}

This code is not statically typeable, and yet with specialization a
compiler could in fact determine the type of each call site.
It just happens to be convenient to specify this behavior with a
branch.
\fi

\section{Units}

Despite their enormous importance in science, unit quantities have
not reached widespread use in programming.
This is not surprising, considering the technical difficulties involved:
units are symbolic objects, so attaching them to numbers can bring
significant overhead.
To restore peak performance, units need to be ``lifted'' into a type
system of some kind, to move their overhead to compile time.
However at that point we encounter a tradeoff similar to that present
in multi-dimensional arrays.
Will it be possible to have an array of numbers with different units?
What if a program wants to return different units based on criteria
the compiler cannot resolve?
Julia's automatic blending of binding times is again helpful.

The SIUnits package by Keno Fischer~\cite{Fischer:2014si} implements
Julia types for SI units and unit quantities:

\begin{singlespace}
\begin{lstlisting}[language=julia]
  immutable SIUnit{m,kg,s,A,K,mol,cd} <: Number
  end

  immutable SIQuantity{T<:Number,m,kg,s,A,K,mol,cd} <: Number
    val::T
  end
\end{lstlisting}
\end{singlespace}

\noindent
This implementation uses a type parameter to store the exponent (an integer,
or perhaps in the near future a rational number) of each base unit associated
with a value.
An \texttt{SIUnit} has only the symbolic part, and can be used to mention
units without committing to a representation.
Its size, as a data type, is zero.
An \texttt{SIQuantity} contains the same symbolic information, but wraps a
number used to represent the scalar part of the unit quantity.
Definitions like the following are used to provide convenient names for units:

\begin{singlespace}
\begin{lstlisting}[language=julia]
  const Meter    = SIUnit{1,0,0,0,0,0,0}()
  const KiloGram = SIUnit{0,1,0,0,0,0,0}()
\end{lstlisting}
\end{singlespace}

\noindent
The approach described in section~\ref{sec:promotion} is used to combine
numbers with units.
Arithmetic is implemented as follows:

\begin{singlespace}
\begin{lstlisting}[language=julia]
function +{T,S,m,kg,s,A,K,mol,cd}(x::SIQuantity{T,m,kg,s,A,K,mol,cd},
                                  y::SIQuantity{S,m,kg,s,A,K,mol,cd})
  val = x.val+y.val
  SIQuantity{typeof(val),m,kg,s,A,K,mol,cd}(val)
end

function +(x::SIQuantity, y::SIQuantity)
  error("Unit mismatch.")
end
\end{lstlisting}
\end{singlespace}

\noindent
In the first definition, the representation types of the two
arguments do not have to match, but the units do (checked via diagonal
dispatch).
Any combination of \texttt{SIQuantity}s that is not otherwise implemented
is a unit mismatch error.

When storing unit quantities in arrays, the array constructor in Julia's
standard library is able to automatically select an appropriate type.
If all elements have the same units, the symbolic unit information is
stored only once in the array's type tag, and the array data uses a
compact representation:

\begin{singlespace}
\begin{verbatim}
julia> a = [1m,2m,3m]
3-element Array{SIQuantity{Int64,1,0,0,0,0,0,0},1}:
 1 m
 2 m
 3 m

julia> reinterpret(Int,a)
3-element Array{Int64,1}:
 1
 2
 3
\end{verbatim}
\end{singlespace}

\noindent
If different units are present, a wider type is chosen via the array
constructor invoking \texttt{promote\_type}:

\begin{singlespace}
\begin{verbatim}
julia> a = [1m,2s]
2-element Array{SIQuantity{Int64,m,kg,s,A,K,mol,cd},1}:
 1 m
 2 s
\end{verbatim}
\end{singlespace}

Unit quantities are different from most number types in that they are not
closed under multiplication.
Nevertheless, generic functions behave as expected.
Consider a generic \texttt{prod} function that multiplies elements of a
collection.
With units, its result type can depend on the size of the collection:

\begin{singlespace}
\begin{lstlisting}[language=julia]
julia> prod([2m,3m]), typeof(prod([2m,3m]))
(6 m~$^2$~, SIQuantity{Int64,2,0,0,0,0,0,0})

julia> prod([2m,3m,4m]), typeof(prod([2m,3m,4m]))
(24 m~$^3$~, SIQuantity{Int64,3,0,0,0,0,0,0})
\end{lstlisting}
\end{singlespace}

\noindent
For simple uses of units, Julia's compiler will generally be able to
infer result types.
For more complex cases like \texttt{prod} it may not be able to, but
this will only result in a loss of performance.
This is the tradeoff a Julia programmer accepts.


%\section{Dates}
%\label{sec:dates}

%compare to python DateTime, compare code length


\section{Algebraic modeling with JuMP}
\label{sec:jump}

An entire subfield of technical computing is devoted to solving optimization
problems.
Linear programming, the problem of finding variable values that maximize a
linear function subject to linear constraints, is especially important and
widely used, for example in operations research.

Real-world problem instances can be quite large, requiring many variables
and constraints.
Simply writing down a large linear program is challenging enough to
require automation.
Specialized languages known as Algebraic Modeling Languages (AMLs)
have been developed for this purpose.
One popular AML is AMPL~\cite{fourer1993ampl}, which allows users to
specify problems using a high-level mathematical syntax.
Here is an example model of a ``knapsack'' problem (from~\cite{LubinDunningIJOC})
in AMPL syntax:

\vspace{-3ex}
\begin{singlespace}
\begin{verbatim}
var x{j in 1..N} >= 0.0, <= 1.0;

maximize Obj:
  sum {j in 1..N} profit[j] * x[j];

subject to CapacityCon:
  sum {j in 1..N} weight[j] * x[j] <= capacity;
\end{verbatim}
\end{singlespace}

\noindent
This syntax is ``compiled'' to a numeric matrix representation that can
be consumed by standard solver libraries.
Although solving the problem can take a significant amount of time,
building the problem representation is often the bottleneck.
AMPL is highly specialized to this task and offers good performance.

Providing the same level of performance and convenience for linear programming
within a general purpose language has proven difficult.
However many users would prefer this, since it would make it easier to
provide data to the model and use the results within a larger system.
Using Julia, Miles Lubin and Iain Dunning solved this problem, creating the
first embedded AML, JuMP~\cite{LubinDunningIJOC}, to match both the performance
and notational advantages of specialized tools.

The solution is a multi-stage program: first the input syntax is converted to
conventional loops and function calls with a macro, then the types of
arguments are used to decide how to update the model, and finally code
specialized for the structure of the input problem runs.
The second stage is handled by a combination of generic functions and
Julia's standard specialization process.
Model parameters can refer to variables in the surrounding Julia program,
without JuMP needing to understand the entire context.

Lubin and Dunning provide an example of how this works.
The input

\vspace{-3ex}
\[ \texttt{@addConstraint(m, sum{weight[j]*x[j], j=1:N} + s == capacity)} \]

\noindent
is lowered to (lightly edited):

\begin{singlespace}
\begin{lstlisting}[language=julia]
aff = AffExpr()
for i = 1:N
  addToExpression(aff, 1.0*weight[i], x[i])
end
addToExpression(aff, 1.0, s)
addToExpression(aff, -1.0, capacity)
addConstraint(m, Constraint(aff,"=="))
\end{lstlisting}
\end{singlespace}

\noindent
\texttt{addToExpression} includes the following methods (plus others):

\begin{singlespace}
\begin{lstlisting}[language=julia]
addToExpression(ex::Number,   c::Number,   x::Number)
addToExpression(ex::Number,   c::Number,   x::Variable)
addToExpression(ex::Number,   c::Number,   x::AffExpr)
addToExpression(ex::Number,   c::Variable, x::Variable)
addToExpression(ex::Number,   c::AffExpr,  x::AffExpr)
addToExpression(aff::AffExpr, c::Number,   x::Number)
addToExpression(aff::AffExpr, c::Number,   x::Variable)
addToExpression(aff::AffExpr, c::Variable, x::Variable)
addToExpression(aff::AffExpr, c::Number,   x::AffExpr)
addToExpression(aff::AffExpr, c::Number,   x::QuadExpr)
\end{lstlisting}
\end{singlespace}

When the arguments are all numbers, \texttt{ex + c*x} is computed directly.
Or, given an \texttt{AffExpr} and a variable, the \texttt{AffExpr}'s lists of
variables and coefficients are extended.
Given two variables, a quadratic term is added.
With this structure, new kinds of terms can be added with minimal code,
without needing to update the more complicated syntax transformation code.

\begin{table}[!t]
\begin{center}
\begin{tabular}{|l|r|r|r|r|}\hline
\textbf{$L$} & \textbf{JuMP/Julia} & \textbf{AMPL} & \textbf{Gurobi/C++} & \textbf{PuLP/PyPy} \\
\hline \hline
1000    & 1.0 & 0.7 & 0.8 & 4.8 \\
\hline
5000    & 4.2 & 3.3 & 3.9 & 26.4 \\
\hline
10000   & 8.9 & 6.7 & 8.3 & 50.6 \\
\hline
50000   & 48.3 & 35.0 & 39.3 & 225.6 \\
\hline
\end{tabular}
\end{center}
\caption[Performance of linear programming tools]{
\small{
Performance (time in seconds) of several linear programming tools for
generating models in MPS format (excerpted from \cite{LubinDunningIJOC}).
$L$ is the number of locations solved for in a facility location problem.
}
}
\label{jumpperf}
\end{table}

Table~\ref{jumpperf} shows performance results.
JuMP is only marginally slower than AMPL or direct use of the Gurobi solver's
C++ API.
The C++ implementation, unlike the other systems compared in the table,
is not solver-independent and does not provide convenient high-level syntax.
PuLP~\cite{mitchell2011pulp} is Python-based, and was benchmarked under the
PyPy JIT compiler~\cite{pypyjit}.


\section{Boundary element method}
\label{sec:BEM}

There are lots of general
packages~\cite{LoggOlgaardEtAl2012a,Rathgeber2015} implementing the
finite-element method (FEM)~\cite{Zienkiewicz13} to solve partial
differential equations (PDEs), but it is much more difficult to create
such a ``multi-physics'' package for the boundary-element method
(BEM)~\cite{Bonnet99,Chew09} to solve surface-integral equations
(SIEs).  One reason is that BEM requires integrating functions with singularities
many times in the inner loop of the code that builds the problem matrix, and
doing this efficiently requires integration code, typically written by hand, that is specialized to
the specific problem being solved.

Recent work \cite{ReidWhJo14} managed a more general solution
using Mathematica to generate C++ code for different cases, in combination with some hand-written code.
This worked well, but was difficult to implement and the resulting system
is difficult to apply to new problems..
We see the familiar pattern of using multiple languages and
code-generation techniques, with coordination of the overall process done
either manually or with ad-hoc scripts.
To polish the implementation for use as a practical library, a likely next
step would be to add a Python interface, adding yet another layer of complexity.
Instead, we believe that the entire problem can be solved efficiently in Julia,
relying on Julia's rich dispatch system and integrated code-generation facilities,
and especially on staged functions to automate the generation of singular-integration
code while retaining the interface of a simple function library.

%%%%%%%%%%%%%%%%%%%%%%%%%%%%%%%%%%%%%%%%%%%%%%%%%%%%%%%%%%%%%%%%%%%%%%%%%%%

% writeup by SGJ

Thanks to Professor Steven Johnson for providing the following section describing
the problem.

\subsubsection{Galerkin matrix assembly for singular kernels}

A typical problem in computational science is to form a discrete approximation
of some infinite-dimensional linear operator $\mathcal{L}$ with some finite set of
basis functions $\{ b_m \}$ via a Galerkin approach\cite{Boyd01,Bonnet99,Zienkiewicz13}, which leads to a
matrix $L$ with entries
$L_{mn} = \langle b_m, \mathcal{L} b_n \rangle = \langle b_m, b_n \rangle_\mathcal{L}$
where $\langle \cdot, \cdot \rangle$ denotes some inner product
(e.g.\ $\langle u, v \rangle = \int u v$ is typical) and
$\langle \cdot, \cdot \rangle_\mathcal{L}$ is the \emph{bilinear form} of the problem.
Computing these matrix elements is known as the matrix \emph{assembly} step, and its
performance is a crucial concern for solving partial differential equations (PDEs) and
integral equations (IEs).

\iffalse
\subsubsection{The ``easy'' case: nonsingular assembly}

For example, in the finite-element method (FEM)~\cite{Zienkiewicz13}, the basis functions $b_m$
are typically low-order polynomials defined piecewise over geometric elements
(typically triangles or tetrahedra), and $\mathcal{L}$ is typically a differential
operator like $-\nabla \cdot c(x) \nabla$ for some coefficients $c(x)$, which leads to
a bilinear form
$\langle b_m, b_n \rangle_\mathcal{L} = \int \nabla b_m \cdot c(x) \nabla b_n$
(after integration by parts).
Because the basis functions are localized and $\mathcal{L}$ consists of local operations,
the matrix $L$ is sparse and $L_{mn}$ need only be computed for $m$ and $n$ corresponding
to neighboring elements.
Moreover, these integrals are straightforward to evaluate by standard cubature schemes
because the integrands are \emph{nonsingular}: they typically have no divergences or
discontinuities.
In particular, because the functions $b_m$ and $c$ are usually smooth within a single
element, one can use a fixed low-order cubature rule: you evaluate the integrand at a
handful of precomputed points within each element, multiply by precomputed weights, and
sum to obtain the approximate integral.

Even so, the basis functions and the coefficient function $c(x)$ may need to be
evaluated tens of millions of times for even a moderate-size mesh in three dimensions,
so production FEM implementations in traditional high-level dynamic languages such as
Matlab and Python are forced to offload matrix assembly to external C and C++ code.
For example, the popular FEniCS~\cite{LoggOlgaardEtAl2012a} and
Firedrake~\cite{Rathgeber2015} FEM packages for
Python both implement domain-specific compilers: a symbolic expression for the bilinear
form is combined with fragments of user-specified C++ code to define functions like
$c(x)$, compiled to C++ code, and then compiled to object code which is dynamically
loaded.
In Julia, we believe this could be simplified considerably because functions like
$c(x)$ could be defined directly in Julia and code generation/compilation could be
performed entirely within Julia without a C++ intermediary.
Indeed, preliminary experiments with pure Julia FEM implementations
[TODO ref http://www.codeproject.com/Articles/579983/Finite-Element-programming-in-Julia]
have demonstrated performance comparable to sophisticated solutions like FEniCS in
Python~\cite{FEniCS12} and FreeFem++ in C++~\cite{MR3043640}.

\subsubsection{Singular assembly for integral operators}

\fi

A challenging case of Galerkin matrix assembly arises for singular
\emph{integral} operators $\mathcal{L}$, which act by
convolving their operand against a singular ``kernel'' function
$K(x)$: $u = \mathcal{L} v$ means that $u(x) = \int K(x - x') v(x')
dx'$.
For example, in electrostatics and other Poisson problems, the
kernel is $K(x) = 1/|x|$ in three dimensions and $\ln |x|$ in two
dimensions, while in scalar Helmholtz (wave) problems it is
$e^{ik|x|}/|x|$ in three dimensions and a Hankel function
$H^{(1)}_0(k|x|)$ in two dimensions.
Formally, Galerkin discretizations lead to matrix assembly problems
similar to those above:
$L_{mn} =: \langle b_m, \mathcal{L} b_n \rangle = \int b_m(x) K(x - x') b_n(x') dx\,dx'$.
However, there are several important differences from FEM:

\begin{itemize}
\item The kernel $K(x)$ nearly always diverges for $|x|=0$, which means that generic
cubature schemes are either unacceptably inaccurate (for low-order schemes) or
unacceptably costly (for adaptive high-order schemes, which require huge numbers
of cubature points around the singularity), or both.

\item Integral operators typically arise for \emph{surface} integral
equations (SIEs)~\cite{Bonnet99,Chew09}, and involve unknowns on a surface.
The analogue of the FEM discretization is then a boundary element method
(BEM)~\cite{Bonnet99,Chew09}, which discretizes a surface into elements
(e.g.\ triangles), with basis functions that are low-order
polynomials defiend piecewise in the elements.
However, there are also volume integral equations (VIEs) which have FEM-like
volumetric meshes and basis functions.

\item The matrix $L$ is typically dense, since $K$ is long-range.
For large problems, $L$ is often stored and applied implicitly via
fast-multipole methods and similar schemes~\cite{Chew01,Liu14}, but even in this
case the diagonal $L_{mm}$ and the entries $L_{mn}$ for adjacent
elements must typically be computed explicitly.
(Moreover, these are the integrals in which the $K$ singularity is present.)
\end{itemize}

These difficulties are part of the reason why there is currently \emph{no}
truly ``generic'' BEM software, analogous to FEniCS~\cite{FEniCS12} for FEM: essentially
all practical BEM code is written for a specific integral kernel and
a specific class of basis functions arising in a particular physical problem.
Changing anything about the kernel or the basis---for example, going
from two- to three-dimensional problems---is a major undertaking.

\subsubsection{Novel solution using Julia}

Julia's dispatch and specialization features make this problem
significantly easier to address:

\begin{itemize}
\item Dispatch on structured types allows the cubature scheme to be selected %at compile time
based on the dimensionality, the degree of the singularity, the degree of
the polynomial basis, and so on, and allows specialized schemes to be added
easily for particular problems with no run time penalty.

\item Staged functions allow computer algebra systems to be invoked at
compile time to generate specialized cubature schemes for particular
kernels.
New developments in BEM integration schemes \cite{ReidWhJo14} have
provided efficient cubature-generation algorithms of this sort, but it
has not yet been practical to integrate them with run time code in
a completely automated way.
\end{itemize}

A prototype implementation of this approach follows.
First we define nominal function types that represent kernel functions:

\begin{singlespace}
\begin{lstlisting}[language=julia]
abstract AbstractKernel

# any kernel that decays as X^p for X ~$\ll$~ s and X^q for X ~$\gg$~ s
abstract APowerLaw{p,q,s} <: AbstractKernel

# ~$r^p$~ power law
type PowerLaw{p} <: APowerLaw{p,p}; end
\end{lstlisting}
\end{singlespace}

\noindent
Next we add a type representing the integral
$\mathcal{K}_n(X) = \int_0^1 w^n K(wX) dw$, and implement it for the
case $K(x) = x^p$:

\begin{singlespace}
\begin{lstlisting}[language=julia]
type FirstIntegral{K<:AbstractKernel,n}; end

function call{p,n}(::FirstIntegral{PowerLaw{p},n}, X::Number)
    return p >= 0 ? X^p / (1 + n + p) : inv(X^(-p) * (1 + n + p))
end
\end{lstlisting}
\end{singlespace}

Code for instances of this function is specialized for $p$ and $n$.
Here is a sample session creating a function instance, and showing the
LLVM~\cite{LLVM} code generated for it:

\begin{singlespace}
\begin{lstlisting}[language=julia]
F = FirstIntegral{PowerLaw{-1}, 3}()

@code_llvm F(2.5)

define double @julia_call_90703(%jl_value_t*, double) {
top:
  %2 = fmul double %1, 3.000000e+00
  %3 = fdiv double 1.000000e+00, %2
  ret double %3
}
\end{lstlisting}
\end{singlespace}

Analytically known special cases can be added for other kernels.
Here are two more.
Notice that a Helmholtz kernel is also a power law kernel
\footnote{The \texttt{exprel} function used here is available in the
external GSL package; see appendix~\ref{appendix:integration}.}:

\begin{singlespace}
\begin{lstlisting}[language=julia]
type Helmholtz{k} <: APowerLaw{-1,-1}; end  # ~$e^{ikr} / 4\pi r$~

function call{k,n}(::FirstIntegral{Helmholtz{k},n}, X::Number)
    ikX = im*k*X
    return exp(ikX) * exprel(n, -ikX) / (4~$\pi$~*n*X)
end

# magnetic field integral equation
type MFIE{k} <: APowerLaw{-3,-3}; end  # ~$(ikr - 1) * e^{ikr} / 4\pi r^3$~

function call{k,n}(::FirstIntegral{MFIE{k},n}, X::Number)
    ikX = im*k*X
    return exp(ikX) * (im*k*exprel(n-1,-ikX)/((n-1)*X) -
                       exprel(n-2,-ikX)/((n-2)*X^2)) / (4~$\pi$~*X)
end
\end{lstlisting}
\end{singlespace}

It is possible to implement \texttt{FirstIntegral} for the general
case of any \texttt{APowerLaw} using numerical integration, however
this procedure is too slow to be useful.
Instead, we would like to compute the integral for a range of
parameter values in advance, construct a Chebyshev approximation,
and use efficient polynomial evaluation at run time.
Julia's staged methods provide an easy way to automate this process.
Appendix~\ref{appendix:integration} presents a full working implementation of
the general case, computing Chebyshev coefficients at compile time.
The final generated code is a straight-line sequence of adds and multiplies.
Here we show a rough outline of how the method works:

\begin{singlespace}
\begin{lstlisting}[language=julia]
@generated function call{P<:APowerLaw,n}(K::FirstIntegral{P,n},
                                         X::Real)
    # compute Chebyshev coefficients c based on K
    # ...
    quote
        X <= 0 && throw(DomainError())
        ~$\xi$~ = 1 - (X + $(2^(1/q)))^$q
        C = @evalcheb ~$\xi$~ $(c...)
        return $p < 0 ? C * (X^$p + $(s^p)) : C
    end
end
\end{lstlisting}
\end{singlespace}

The method signature reflects the shape of the problem: the user might
implement any kernel at all, but if it does not take the form of an
\texttt{APowerLaw\{p\}} then this numerical procedure is not useful.
\texttt{p} must be known, and cannot be easily extracted from an arbitrary
opaque function.
If the interface allowed an opaque function to be passed, the value of
\texttt{p} would need to be passed separately, which de-couples the
property from the function, and requires a custom lookup table for caching
generated code.

One might hope that with a sufficient amount of constant propagation and
functional purity analysis, a compiler could automatically move
computation of the Chebyshev coefficients to compile time.
That does indeed seem possible.
The point of \texttt{@generated}, then, is to make it so easy to
get the desired staging that one does not have to worry
about what amazing feats the compiler might be capable of.
Without it, one would face a difficult debugging challenge when the
compiler, for whatever opaque reasons, did not perform the desired
optimizations.
% maybe:
% this idea can be extended to a general philosophy of embracing
% performance tweaking, and trying to make it as easy as possible.
% another example is adding a single OpenMP #pragma to a loop instead of
% hoping for an auto-parallelizing compiler.

To complete the picture, here is pseudocode for how this library would be
used to solve a boundary element problem over a triangle mesh.
The three cases inside the loop are a translation of the three formulas
from figure 1 in \cite{ReidWhJo14} (only the first formula's code is shown in
full for simplicity):

%  # n depends on complicated stuff
\begin{singlespace}
\begin{lstlisting}[language=julia]
FP = [(FirstIntegral{K,i}(), polynomial(i)) for i in N]
for trianglepair in mesh
    i, j = indexes based on triangles
    if common(trianglepair)
        M[i,j] = integrate(y->sum([F(y)*P(y) for (F,P) in FP]),
                           0, 1)
    elseif commonedge(trianglepair)
        # a 2d integral
    elseif commonvertex(trianglepair)
        # a 3d integral
    else
        # separate triangles
    end
end
\end{lstlisting}
\end{singlespace}
% common edge case
% #M[i,j] = integrate((y1,y2)->sum([Fce[p](y1,y2) * Pce[p](y1,y2) for p=1:n_ce]), 0, 1)

% common vertex case
% #M[i,j] = integrate((y1,y2,y3)->sum([Fcv[p](y1,y2,y3) * Pcv[p](y1,y2,y3) for p=1:n_cv]), 0, 1)

%%%%%%%%%%%%%%%%%%%%%%%%%%%%%%%%%%%%%%%%%%%%%%%%%%%%%%%%%%%%%%%%%%%%%%%%%%%

%\section{Computational geometry}

%robust predicates (dispatch over points, lines) and VoronoiDelaunay.jl
%benchmarked against CGAL


\section{Beating the incumbents}
\label{sec:beating}

There is anecdotal evidence that libraries written in Julia are occasionally
faster than their equivalents in established languages like C++, or
than built-in functions in environments like MATLAB~\cite{matlab}.
Here we collect a few examples.

Special functions (e.g.\ Bessel functions, the error function, etc.) are
often evaluated using polynomial approximations.
Traditional libraries usually read polynomial coefficients from arrays.
When we wrote some of these functions in Julia we used a macro
to expand the polynomials in line, storing the coefficients in the
instruction stream, leading to better performance.
There is an efficient algorithm for evaluating polynomials at complex-valued
arguments~\cite{knuth1969art}, but it is fairly subtle and so not always
used.
%- erfinv and digamma using horner macro
If the code can be generated by a macro, though, it is much easier to take
advantage of, and Julia's \texttt{@evalpoly} does so.

%\texttt{@evalpoly} macro has separate cases for real and complex in order to
%automatically take advantage of a subtle algorithm .
%A macro is perfect for generating the neccessary code, however it lacks
%the type information needed to select between the two cases.
%In Julia, it can generate a branch with a type check, and rely on the
%unused case being removed by automatic specialization. (A generic
%function with two definitions could be generated instead, but in
%high-performance programming the ``force inlining'' behavior of macros
%is welcome.)

Julia's function for generating normally distributed random numbers is
significantly faster than that found in many numerical environments.
We implemented the popular ziggurat method~\cite{Marsaglia:Tsang:2000:JSSOBK:v05i08},
which consists of a commonly used fast path and a rarely used more complicated
procedure.
The fast path and check for the rare case can be put in a small function,
which can be inlined at call sites in user code.
We attribute most of the performance gain to this trick.
This demonstrates that removing ``glue code'' overhead can be worthwhile.

%grisu: 6kLOC to 1kLOC (PR 7291), less than factor of 2 slower than c++
Julia has long used the Grisu algorithm for printing floating-point
numbers~\cite{Loitsch:2010:PFN:1809028.1806623}, at first calling the
original implementation in C++.
An opportunity to do an unusually direct comparison arose when this
code was ported to Julia~\cite{grisu}.
The Julia implementation is at worst 20\% slower, but wins on
productivity: the C++ code is around 6000 lines, the Julia code
around 1000 lines.
% TODO can we say why?
