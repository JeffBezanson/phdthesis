% -*-latex-*-
% $Log: cover.tex,v $
% Revision 1.7  2010/04/29 11:35:46  bryt
% changed department chair from Art Smith to Terry Orlando
% changed default copyright flag from author to MIT, left directions for changing it back
% 
% Revision 1.6  1999/10/21 14:49:31  boojum
% changed comment referring to documentstyle
%
% Revision 1.5  1999/10/21 14:39:04  boojum
% *** empty log message ***
%
% Revision 1.4  1997/04/18  17:54:10  othomas
% added page numbers on abstract and cover, and made 1 abstract
% page the default rather than 2.  (anne hunter tells me this
% is the new institute standard.)
%
% Revision 1.4  1997/04/18  17:54:10  othomas
% added page numbers on abstract and cover, and made 1 abstract
% page the default rather than 2.  (anne hunter tells me this
% is the new institute standard.)
%
% Revision 1.3  93/05/17  17:06:29  starflt
% Added acknowledgements section (suggested by tompalka)
% 
% Revision 1.2  92/04/22  13:13:13  epeisach
% Fixes for 1991 course 6 requirements
% Phrase "and to grant others the right to do so" has been added to 
% permission clause
% Second copy of abstract is not counted as separate pages so numbering works
% out
% 
% Revision 1.1  92/04/22  13:08:20  epeisach
\title{Abstraction in Technical Computing}

\author{Jeffrey Werner Bezanson}
\prevdegrees{A.B., Harvard University (2004) \\
             S.M., Massachusetts Institute of Technology (2012)}
\department{Department of Electrical Engineering and Computer Science}
% If the thesis is for two degrees simultaneously, list them both
% separated by \and like this:
% \degree{Doctor of Philosophy \and Master of Science}
\degree{Doctor of Philosophy}
\degreemonth{June}
\degreeyear{2015}
\thesisdate{May 20, 2015}

%% By default, the thesis will be copyrighted to MIT.  If you need to copyright
%% the thesis to yourself, just specify the `vi' documentclass option.  If for
%% some reason you want to exactly specify the copyright notice text, you can
%% use the \copyrightnoticetext command.  
%\copyrightnoticetext{\copyright IBM, 1990.  Do not open till Xmas.}

% If there is more than one supervisor, use the \supervisor command
% once for each.
\supervisor{Alan Edelman}{Professor}

% This is the department committee chairman, not the thesis committee
% chairman.  You should replace this with your Department's Committee
% Chairman.
\chairman{Leslie Kolodziejski}{Chairman, Department Committee on Graduate Students}

% Make the titlepage based on the above information.  If you need
% something special and can't use the standard form, you can specify
% the exact text of the titlepage yourself.  Put it in a titlepage
% environment and leave blank lines where you want vertical space.
% The spaces will be adjusted to fill the entire page.  The dotted
% lines for the signatures are made with the \signature command.
\maketitle

\newpage
\null

% The abstractpage environment sets up everything on the page except
% the text itself.  The title and other header material are put at the
% top of the page, and the supervisors are listed at the bottom.  A
% new page is begun both before and after.  Of course, an abstract may
% be more than one page itself.  If you need more control over the
% format of the page, you can use the abstract environment, which puts
% the word "Abstract" at the beginning and single spaces its text.

%% You can either \input (*not* \include) your abstract file, or you can put
%% the text of the abstract directly between the \begin{abstractpage} and
%% \end{abstractpage} commands.

% First copy: start a new page, and save the page number.
\newpage
% Uncomment the next line if you do NOT want a page number on your
% abstract and acknowledgments pages.
% \pagestyle{empty}
\setcounter{savepage}{\thepage}
\begin{abstractpage}
% $Log: abstract.tex,v $
% Revision 1.1  93/05/14  14:56:25  starflt
% Initial revision
% 
% Revision 1.1  90/05/04  10:41:01  lwvanels
% Initial revision
% 
%
%% The text of your abstract and nothing else (other than comments) goes here.
%% It will be single-spaced and the rest of the text that is supposed to go on
%% the abstract page will be generated by the abstractpage environment.  This
%% file should be \input (not \include 'd) from cover.tex.

Array-based programming environments are popular for scientific and
technical computing.
These systems consist of built-in function libraries paired with high-level
languages for interaction.
Although the libraries perform well, it is widely believed that scripting in these
languages is necessarily slow, and that only heroic feats of engineering can at
best partially ameliorate this problem.

This thesis argues that what is really needed is a more coherent
structure for this functionality.
To find one, we must ask what technical computing is really about.
This thesis suggests that this kind of programming is characterized by an emphasis on operator
complexity and code specialization, and that a language can be designed to
better fit these requirements.

The key idea is to integrate code \emph{selection} with code \emph{specialization},
using generic functions and data-flow type inference.
Systems like these can suffer from inefficient compilation, or from
uncertainty about what to specialize on.
We show that sufficiently powerful type-based dispatch addresses these problems.
The resulting language, Julia, achieves a Quine-style
``explication by elimination'' of many of the productive features
technical computing users expect.

% we show how this can be use to greatly simplify code for demanding
% scientific applications that require a mixture of binding times.

% thesis stmt: integrating code selection and specialization with
% type-based dynamic dispatch captures both the performance and
% productivity requirements of technical computing.

%For this role I propose an abstraction based on an extended version of
%generic functions.
%The novelty of this mechanism is that it is both flexible enough to describe
%the wide variety of behaviors users need in practice, while also providing
%enough information to a compiler to yield good performance.


% integration of selection and specialization

% making data-flow and specialization-based languages practical

% answers the question of what to specialize on

% applies Quine's ``explication through elimination'' to common features of T.C.

\end{abstractpage}

\newpage
\null

% Additional copy: start a new page, and reset the page number.  This way,
% the second copy of the abstract is not counted as separate pages.
% Uncomment the next 6 lines if you need two copies of the abstract
% page.
% \setcounter{page}{\thesavepage}
% \begin{abstractpage}
% % $Log: abstract.tex,v $
% Revision 1.1  93/05/14  14:56:25  starflt
% Initial revision
% 
% Revision 1.1  90/05/04  10:41:01  lwvanels
% Initial revision
% 
%
%% The text of your abstract and nothing else (other than comments) goes here.
%% It will be single-spaced and the rest of the text that is supposed to go on
%% the abstract page will be generated by the abstractpage environment.  This
%% file should be \input (not \include 'd) from cover.tex.

Array-based programming environments are popular for scientific and
technical computing.
These systems consist of built-in function libraries paired with high-level
languages for interaction.
Although the libraries perform well, it is widely believed that scripting in these
languages is necessarily slow, and that only heroic feats of engineering can at
best partially ameliorate this problem.

This thesis argues that what is really needed is a more coherent
structure for this functionality.
To find one, we must ask what technical computing is really about.
This thesis suggests that this kind of programming is characterized by an emphasis on operator
complexity and code specialization, and that a language can be designed to
better fit these requirements.

The key idea is to integrate code \emph{selection} with code \emph{specialization},
using generic functions and data-flow type inference.
Systems like these can suffer from inefficient compilation, or from
uncertainty about what to specialize on.
We show that sufficiently powerful type-based dispatch addresses these problems.
The resulting language, Julia, achieves a Quine-style
``explication by elimination'' of many of the productive features
technical computing users expect.

% we show how this can be use to greatly simplify code for demanding
% scientific applications that require a mixture of binding times.

% thesis stmt: integrating code selection and specialization with
% type-based dynamic dispatch captures both the performance and
% productivity requirements of technical computing.

%For this role I propose an abstraction based on an extended version of
%generic functions.
%The novelty of this mechanism is that it is both flexible enough to describe
%the wide variety of behaviors users need in practice, while also providing
%enough information to a compiler to yield good performance.


% integration of selection and specialization

% making data-flow and specialization-based languages practical

% answers the question of what to specialize on

% applies Quine's ``explication through elimination'' to common features of T.C.

% \end{abstractpage}

\newpage

\begin{singlespace}

\section*{Acknowledgments}

The three people without whom this work would not have been possible
are Alan Edelman, Stefan Karpinski, and Viral Shah.
Their ideas and their support have been extraordinary.
I habitually mentally included them while writing this thesis in the first
person plural.

I am especially indebted to Steven Johnson, for
explaining to me why Julia is doomed, and subsequently working as much as
anybody to make it less doomed.
He is the author of the code in appendix~\ref{appendix:integration} and,
along with Homer Reid, is responsible for the example in section~\ref{sec:BEM}.

I thank Jiahao Chen for coauthoring, contributing to nearly
every aspect of our project,
and even coming up with the title of this very document,
Jean Yang for paper writing help, and
Fernando Perez and Brian Granger for welcoming us so generously to the wonderful
world of Jupyter.

For their valuable feedback on drafts I am grateful to
Jake Bolewski, Oscar Blumberg, Tim Holy, and my committee members
Saman Amarasinghe and Gerry Sussman.

It is quite a privilege to be able to cite other people's work to
demonstrate and evaluate my own.
In this regard I am grateful to Miles Lubin, Iain Dunning, Keno Fischer,
and Andreas Jensen.

At this point, I need to thank so many more people that doing so requires
automation.
The full list can be found in our \texttt{git} history.
As a partial list, I thank for their excellent contributions to the
software:
Jameson Nash,
Tim Holy,
Mike Nolta,
Carlo Baldassi,
Elliot Saba,
Andreas Jensen,
Tony Kelman,
Jake Bolewski,
Kevin Squire,
Isaiah Norton,
Amit Murthy,
Simon Kornblith,
Patrick O'Leary,
Dahua Lin,
Jacob Quinn,
Douglas Bates,
Simon Byrne,
Mike Innes,
Ivar Nesje,
Tanmay Mohapatra,
Matt Bauman,
Rafael Fourquet (largely responsible for the speed of \texttt{randn}
described in section~\ref{sec:beating}),
Arch Robison,
Oscar Blumberg,
John Myles White,
Shashi Gowda,
and Daniel Jones.

For generally advancing the Julia community, I thank
Evan Miller, Hunter Owens, James Porter, Leah Hanson, and
Mykel Kochenderfer.

These collaborators made this far more than a research project: they
made it a formidable software project, they gave it global reach, and they
made it fun.


\end{singlespace}

%%%%%%%%%%%%%%%%%%%%%%%%%%%%%%%%%%%%%%%%%%%%%%%%%%%%%%%%%%%%%%%%%%%%%%
% -*-latex-*-
