\chapter{Conclusion}

% to what extent is this a technical computing language?
% it's not, really :)

%``I've yet to hear anyone explain how you decide what are the boundaries of a `domain-specific' language. Isn't the `domain' mathematics and science itself?''~\cite{bobharperdsl}

%tc is not a domain, and doesnt need to be:
% \cite{meyerovich2012socio} about appealing to some domain first
% technical computing was ripe for such a treatment

% something about how ``information hiding'' is anathema to science,
% and coincidentally is not the focus of t.c. languages or rigorous
% functional languages.

%% A generation of dynamic languages have been designed by trying variants of
%% the class-based object oriented paradigm. This process has been aided by
%% the development of standard techniques (e.g. bytecode VMs) and reusable
%% infrastructure such as code generators, garbage collectors, and whole VMs
%% like the JVM and CLR.

%% It is possible to envision a future generation of languages that generalize
%% this design to set-theoretic subtyping instead of just classes. This next
%% generation will require its own new tools, such as partial evaluators (already
%% under development in PyPy and Truffle). One can also imagine these future
%% language designers wanting reusable program analyses, and tools for
%% developing lattices and their operators.


% not a domain, just users whose priorities were left out
We began by observing that technical computing users have priorities
that have rarely been addressed directly by programming languages.
Arguably, even many of the current technical computing languages only
address these priorities for a subset of cases.
% in particular, they want flexible notation, and code specialized
% for information regardless of when that information is known.
In some sense all that was missing was an appropriate notion of
partial information.

partial information implies
- describing sets of values, therefore code applicability
- a performance model, capturing what sorts of information the compiler
  can determine
- staging. the whole idea of generating code depends on knowing something
  about what needs to be done, but not everything, so that it can be
  reused enough to be worthwhile.
  therefore partial information.

of course this sounds a lot like a type system, but the point is that
this idea can perform important roles at run time.

% what's wrong with dynamic languages is not just a lack of static
% guarantees, but that they don't seem to be pareto optimal.
% actually not enough is done with dynamic typing: you give up type
% checking, and all you get is untyped dictionaries in return?

%exploiting partial information is key.
%dispatch was an especially natural way to do this,
%are there others?
%What else can we do with partial information as part of a programming
%model?
% somehow being dual to ML-family languages; what's hard for them is
% easy for us and vice versa.

\iffalse
It is interesting to observe that the data model of a language like Julia
consists of two key relations: the subtype relation, which is relatively
well understood and enjoys useful properties like transitivity, but also
the typeof relation, which relates individual values to their types
(i.e. the `typeof` function). The typeof relation appears not to be
transitive, and also has a degree of arbitrariness: a value is of a type
merely because it is labeled as such, and because various bits of code
conspire to ensure that this labeling makes sense according to various
criteria.

We have speculated about whether future languages will be able to do away
with this distinction. One approach is $\lambda_{\aleph}$. We have also
speculated that this could be done using types based on non-well-founded
set theory, combining the subset-of and element-of relations using
self-containing sets. We are not yet sure what a practical language based
on this idea might look like.
\fi

\section{Performance}

If we are interested in abstraction, then type specialization is the
first priority for getting performance.
However in a broader view there is much more to performance.
For example ``code selection'' appears in a different guise in the
idea of \emph{algorithmic choice}~\cite{Ansel:2009:PLC:1542476.1542481,Ansel:2014:OEF:2628071.2628092}:
at any point in a program, we might have a choice of several algorithms and
want to automatically pick the fastest one.
Julia's emphasis on writing functions in small pieces and describing
applicability seems naturally suited to this kind of approach.
% could maybe be implemented by passing around an execution plan
% and dispatching on aspects of it

%There are several key aspects of performance programming that our design
%does not directly address, e.g. storage and in-place optimizations.

\section{Implementation issues}

As with any practical system, we must begin the process of trying to get back
what our design initially trades away.

For example, the specialization-based polymorphism we use does not support
separate compilation.
Fortunately that's never stopped anyone before: C++ templates have the same
problem, but separately compiled modules can still be generated.
We plan to do the same, with the same cost of re-analyzing all relevant code
for each module.
Another approach is to use layering instead of fully separate modules.
Starting with the core language, one next identifies and compiles a useful base
library.
That layer is then considered ``sealed'' and can no longer be extended ---
it effectively becomes part of the language.
Now applications using this library can be compiled more efficiently.
More layers can be added (slowest-changing first) to support more specific
use cases.
This is essentially the ``telescoping languages'' approach~\cite{telescoping}.

% steve j suggestions:
% Precompilation?  Issues with type-inference?  Inlining anonymous functions?
% \section{Separate compilation}
% Technically julia does not support it, but that's never stopped anybody
% from doing it anyway. e.g. C++

% issues
% - static comp, separate comp
% - functions, the result of `map`
% - static typing, TypeCheck.jl

\section{Future work}

\begin{itemize}
\item formalize more details of dispatch, meet, join, widen
\item product domains, how to incorporate finer types more smoothly
\item how to incorporate static checks? \cite{typecheckjl}
\item separate compilation
\item function types
\item shared memory, cilk
\item can we implement BLAS?
\item incorporate autotuning?
\end{itemize}


% oscar:
% - what we have now is the ``minimal julia''
% - ``easy to write code that's easy to analyze statically''
% - got non-programmer people to write easy to analyze code
%   - bad that they need to think about type stability, but
%     also good that is relatively easy to grasp, and it gets
%     people thinking about it who wouldn't normally
%     - the types give people the tools to deal with it
% - can't pretend that people wont have to worry about this stuff;
%   it's fundamental

% mention lines of code
%% How much of the past 30 years of handwritten Matlab internals can
%% be autogenerated with a compiler? (A lot)

% we know these programs need more organization and type info.
% this is the first truly viable approach

% easy polymorphism.
% - starts with a very simple HLL
% - performs well because of underlying mechanisms,
%   which then unfold gradually when you want more control over performance

% making everything a method, and everything glued together by types
% gives certain reflective properties.


\section{Project status}
