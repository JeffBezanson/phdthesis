The three people without whom this work would not have been possible
are Alan Edelman, Stefan Karpinski, and Viral Shah.
Their ideas and their support have been extraordinary.
I habitually mentally included them while writing this thesis in the first
person plural.

I am especially indebted to Steven Johnson, for
explaining to me why Julia is doomed, and subsequently working as much as
anybody to make it less doomed.
He is the author of the code in appendix~\ref{appendix:integration} and,
along with Homer Reid, is responsible for the example in section~\ref{sec:BEM}.

I thank Jiahao Chen for coauthoring, contributing to nearly
every aspect of our project,
and even coming up with the title of this very document,
Jean Yang for paper writing help, and
Fernando Perez and Brian Granger for welcoming us so generously to the wonderful
world of Jupyter.

For their valuable feedback on drafts I am grateful to
Jake Bolewski, Oscar Blumberg, Tim Holy, and my committee members
Saman Amarasinghe and Gerry Sussman.

It is quite a privilege to be able to cite other people's work to
demonstrate and evaluate my own.
In this regard I am grateful to Miles Lubin, Iain Dunning, Keno Fischer,
and Andreas Jensen.

At this point, I need to thank so many more people that doing so requires
automation.
The full list can be found in our \texttt{git} history.
As a partial list, I thank for their excellent contributions to the
software:
Jameson Nash,
Tim Holy,
Mike Nolta,
Carlo Baldassi,
Elliot Saba,
Andreas Jensen,
Tony Kelman,
Jake Bolewski,
Kevin Squire,
Isaiah Norton,
Amit Murthy,
Simon Kornblith,
Patrick O'Leary,
Dahua Lin,
Jacob Quinn,
Douglas Bates,
Simon Byrne,
Mike Innes,
Ivar Nesje,
Tanmay Mohapatra,
Matt Bauman,
Rafael Fourquet (largely responsible for the speed of \texttt{randn}
described in section~\ref{sec:beating}),
Arch Robison,
Oscar Blumberg,
John Myles White,
Shashi Gowda,
and Daniel Jones.

For generally advancing the Julia community, I thank
Evan Miller, Hunter Owens, James Porter, Leah Hanson, and
Mykel Kochenderfer.

These collaborators made this far more than a research project: they
made it a formidable software project, they gave it global reach, and they
made it fun.
