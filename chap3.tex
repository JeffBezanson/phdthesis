\chapter{Available technology}

\section{The language design space}

% ``technology transfer'' from PLT to TC

It is helpful to begin with a rather coarse classification of programming
languages, according to how expressive their type systems are, and whether
their type systems are dynamic or static:

\vspace{3ex}

\begin{table}
\begin{tabular}{|c||c|c|}
\hline 
 & More types & Fewer types\tabularnewline
\hline 
\hline 
Dynamic & Dylan, Julia & Scheme, Python, MATLAB\tabularnewline
\hline 
Static & ML, Haskell, Scala & Fortran, Pascal, C\tabularnewline
\hline 
\end{tabular}
\caption[A coarse classification of languages]{}
\label{languageclass}
\end{table}

\vspace{3ex}

The lower right corner tends to contain older languages.
The upper right corner contains many popular ``dynamic'' languages.
The lower left corner contains many modern languages resulting from research
on static type systems.
We are most interested in the upper left corner, which is notable for being
rather sparsely populated.
It has been generally believed that dynamic languages do not ``need'' types,
or that there is no point in talking about types if they are not going to be
checked at compile time.
These views have some merit, but as a result the top-left corner of this design
space has been under-explored.

\subsection{Classes vs.\ functions}

Technical computing is function oriented.

in C++ to ``extend'' an existing type you have to use another mechanism,
overloading.
however then you also need to accept different semantics, static resolution.


\subsection{Separate compilation vs.\ performance}

Writing the signature of a generic method that needs to be separately compiled,
as in Java, can be a difficult exercise.
The programmer must use the type system to write down sufficient conditions on all
arguments.
The following example from a generic Java graph
library~\cite{Garcia:2003:CSL:949305.949317} demonstrates the level of verbosity
that can be required:

\begin{singlespace}
\begin{lstlisting}[language=java,style=ttcode]
public class breadth_first_search {
  public static<Vertex, Edge extends GraphEdge<Vertex>,
                VertexIterator extends Iterator<Vertex>,
                OutEdgeIterator extends Iterator<Edge>,
                Visitor extends BFSVisitor,
                ColorMap extends ReadWriteMap<Vertex, Integer>>
  void go(VertexListAndIncidenceGraph<Vertex,Edge,
            VertexIterator,VerticesSizeType,OutEdgeIterator,
            DegreeSizeType> g,
          Vertex s, ColorMap c, Visitor vis);
}
\end{lstlisting}
\end{singlespace}

(other problems: primitive types int and double cannot be used,
static parameters can only be inferred directly from arguments,
not from constraints of other parameters. our subtyping system
does not have this problem)

If, however, we are going to specialize the method, the compiler can analyze it
using actual types from call sites, and see for itself whether the method works
in each case (this is how C++ templates are type-checked; they are analyzed
again for each instantiation).

It is quite interesting that performance and ease-of-use pull this design
decision in the same direction.

\subsection{Parametric vs.\ ad-hoc polymorphism}

The term \emph{polymorphism} refers generally to reuse of code or data
structures in different contexts.
A further distinction is often made between \emph{parametric} polymorphism
and \emph{ad-hoc} polymorphism.
Parametric polymorphism refers to reusing the \emph{same} code for
different purposes, while ad-hoc polymorphism refers to selecting
\emph{different} code for different circumstances.

Both forms occur frequently in technical computing.
For example, a programmer intuitively expects that \texttt{A[i]} selects
an element of an array regardless of what kind of array \texttt{A} is.
\texttt{A} might contain integers or strings, it might be a local
array or a distributed array, and so on.
This is parametric polymorphism.
However the ``parametric'' property only applies to the code \texttt{A[i]}
itself.
When we dig into how array indexing actually works, we will probably
need to resort to ad-hoc polymorphism.
For example, when \texttt{A} is a local array the code accesses local
memory, and when \texttt{A} is distributed it might be necessary to
send a network message instead.
At a lower level, the machine code for \texttt{A[i]} needs to be
different depending on the array's element type.

The idea of specialization unites parametric and ad-hoc polymorphism.
Beginning with a parametrically polymorphic function, one can imagine
a compiler specializing it for various cases, i.e.\ certain concrete argument
values. These specializations could be stored in a lookup table, for use
either at compile time or at run time.

Next we can imagine that the compiler might not optimally specialize
certain definitions, and that the programmer would like to provide
hints or hand-written implementations to speed up special cases.
For example, imagine a function that traverses a generic array. A compiler-generated
specialization might inline a specific array types's indexing operations, but a human
might further realize that the loop order should be switched for certain
arrays types based on their storage order.

However, if we are going to let a human specialize a function for performance,
we might as well allow them to specialize it for some other reason, including
entirely different behavior. But at this point separate ad-hoc polymorphism
is no longer necessary; we are using a single overloading feature for
everything.


Parametric polymorphism describes code that works for any object precisely
because it does not do anything meaningful to the object, for example the
identity function. In contrast, programming with tagged data (e.g.
symbolic expression systems, XML) permits code to work for any object
because every object has the same structure, allowing meaningful
operations.

In theory, a parametrically polymorphic function works on all data
types.
In practice, this can be achieved by forcing a uniform representation of
data such as pointers, which can be handled by the same code regardless of
what kind of data is pointed to.
However this kind of representation is not the most efficient, and
for good performance specialized code for different types must be
generated.

\subsection{Static checking vs.\ flexibility}

\subsection{Modularity vs.\ large vocabularies}

In the context of software engineering, modularity is a primary concern.
To build large systems, separate components must be isolated to some
degree.
Reasons for this include simple concerns like avoiding name conflicts,
and a desire to separate interface from implementation to allow
a component to change without affecting the rest of the system.

Modularity is sometimes taken to an extreme, and one will see
fully qualified names like \texttt{org.jboss.annotation.ejb.cache.simple.CacheConfig}
(selected from the JBOSS Java APIs).

Technical computing languages have often avoided and discouraged such
designs. For example MATLAB for most of its history supported only a single
namespace, which comes pre-populated with thousands of functions.


\subsection{Eliminating vs.\ embracing tagged data}

\subsection{Data hiding vs.\ explicit memory layout}

Examples of CSC and CSR sparse representation.
In one sense this is a perfect example of an interface with multiple implementations,
and therefore a good use case for object-oriented programming.
However in the technical computing world, \emph{hiding} this difference in
representation is not usually considered desirable.
Clearly a sparse matrix class cannot contain all functions of matrices that users
might want to compute.
Yet when new functions are added, the programmer needs and wants to exploit
representation details (CSC or CSR) for performance.
The performance differences involved here are quite significant (TODO cite).

The loss of encapsulation due to multi-methods weighed in \cite{binarymethods}
is less of a problem for technical computing, and in some cases even
advantageous.

\subsection{Dynamic dispatch is key}

It would be unpleasant if every piece of every program we wrote were forced
to do only one specific task. Every time we wanted to do something slightly
different, we'd have to write a different program. But if a language
allows the same program element to do different things at different times,
we can write whole classes of programs at once. This kind of capability is
one of the main reasons \emph{object-oriented} programming is popular: it
provides a way to automatically select different behaviors according to
some structured criteria.

In class-based OO there is essentially \emph{no way} to create an operation
that dispatches on existing types (the expression problem \cite{wadler1998expression}).
This clearly
does not match technical computing, where most programs deal with the same
few types (e.g. number, array), and might sensibly want to write new operations
that dispatch on them.

%Somewhat unfortunately, the term \emph{object-oriented} has many
%connotations, and the \emph{object-oriented} methodology tries to address
%multiple software engineering problems --- for example modularity,
%encapsulation, implementation hiding, and code reuse. These issues are
%important, but it may be because of them that over time too little
%emphasis has been placed on expressive power.

We use the non-standard term ``criteria'' deliberately, in order
to clarify our point of view, which is independent of any particular
object system.

The sophistication of the available ``selection criteria'' account for a
large part of the perceived ``power'' or leverage provided by a language.
In fact it is possible to illustrate a hierarchy of such mechanisms.
As an example, consider a simple simulation, and how it can be written
under a series of increasingly powerful paradigms. First, written-out
imperative code:

\begin{singlespace}
\begin{verbatim}
while running
    for a in animals
        b = nearby_animal(a)
        if a isa Rabbit
            if b isa Wolf then run(a)
            if b isa Rabbit then mate(a,b)
        else if a isa Wolf
            if b isa Rabbit then eat(a,b)
            if b isa Wolf then follow(a,b)
        end
    end
end
\end{verbatim}
\end{singlespace}

We can see how this would get tedious as we add more kinds of animals
and more behaviors. Another problem is that the animal behavior is
implemented directly inside the control loop, so it is hard to see
what parts are simulation control logic and what parts are animal
behavior. Adding a simple object system leads to a nicer implementation
\footnote{A perennial problem with simple examples is that better
implementations often make the code longer.}:

\begin{singlespace}
\begin{verbatim}
class Rabbit
    method encounter(b)
        if b isa Wolf then run()
        if b isa Rabbit then mate(b)
    end
end

class Wolf
    method encounter(b)
        if b isa Rabbit then eat(b)
        if b isa Wolf then follow(b)
    end
end

while running
    for a in animals
        b = nearby_animal(a)
        a.encounter(b)
    end
end
\end{verbatim}
\end{singlespace}

Here all of the simulation's animal behavior has been essentially
compressed into a single program point: \texttt{a.encounter(b)}
leads to all of the behavior by selecting an implementation based
on the first argument, \texttt{a}. This kind of criterion is
essentially indexed lookup; we can imagine that \texttt{a} is
simply an integer index into a table of operations.

The next enhancement to ``selection criteria'' adds a hierarchy
of behaviors, to provide further opportunities to avoid
repetition:

\todo[inline]{has  $<:$  been defined yet?}

\begin{singlespace}
\begin{verbatim}
abstract class Animal
    method nearby()
        # search within some radius
    end
end

class Rabbit <: Animal
    method encounter(b: Animal)
        if b isa Wolf then run()
        if b isa Rabbit then mate(b)
    end
end

class Wolf <: Animal
    method encounter(b: Animal)
        if b isa Rabbit then eat(b)
        if b isa Wolf then follow(b)
    end
end

while running
    for a in animals
        b = a.nearby()
        a.encounter(b)
    end
end
\end{verbatim}
\end{singlespace}

We are still essentially doing table lookup, but the tables have
more structure: every \texttt{Animal} has the \texttt{nearby}
method, and can inherit a general-purpose implementation.

This brings us roughly to the level of most popular object-oriented
languages. But in this example still more can be done. Notice that
in the first step to objects we replaced one level of \texttt{if}
statements with method lookup. However, inside of these methods
a structured set of \texttt{if} statements remains. We can
replace these by adding another level of dispatch.

\begin{singlespace}
\begin{verbatim}
class Rabbit <: Animal
    method encounter(b: Wolf) = run()
    method encounter(b: Rabbit) = mate(b)
end

class Wolf <: Animal
    method encounter(b: Rabbit) = eat(b)
    method encounter(b: Wolf) = follow(b)
end
\end{verbatim}
\end{singlespace}

We now have a \emph{double dispatch} system, where a method call
uses two lookups, first on the first argument and then on the
second argument.

This syntax might be considered a bit nicer, but the design
clearly begs a question: why is $n=2$ special?
It isn't, and we could clearly consider even more method arguments as part of
dispatch.
But at that point, why is the first argument special?
Why separate methods in a special way based on the first argument?
It seems arbitrary, and indeed we can remove the special treatment:

\begin{singlespace}
\begin{verbatim}
abstract class Animal
end

class Rabbit <: Animal
end

class Wolf <: Animal
end

nearby(a: Animal) = # search
encounter(a: Rabbit, b: Wolf) = run(a)
encounter(a: Rabbit, b: Rabbit) = mate(a,b)
encounter(a: Wolf, b: Rabbit) = eat(a, b)
encounter(a: Wolf, b: Wolf) = follow(a, b)

while running
    for a in animals
        b = nearby(a)
        encounter(a, b)
    end
end
\end{verbatim}
\end{singlespace}

Here we made two major changes: the methods have been moved ``outside''
of any classes, and all arguments are listed explicitly.
This is sometimes called \emph{external dispatch}.
This change has significant implications.
Since methods no longer need to be ``inside'' classes, there is no syntactic
limit to where definitions may appear.
Now it is easier to add new methods after a class has been defined.
Methods also now naturally operate on combinations of objects, not single objects.

The shift to thinking about combinations of objects is fairly revolutionary.
Many interesting properties only apply to combinations of objects, and not
individuals.
We are also now free to think of more exotic kinds of combinations.

We can define a method on \emph{any number} of objects:

\begin{verbatim}
encounter(ws: Wolf...) = pack(ws)
\end{verbatim}

\todo[inline]{We can also do diagonal dispatch:}

\begin{verbatim}
encounter{T<:Animal}(a: T, b: T) = mate(a, b)
\end{verbatim}



\section{Dispatch systems}

% CHART of different multiple dispatches. classify them

%Why ``just overloading'' is not enough. You intercept every operation and
%rewrite it, kind of an escape hatch for a language you don't like. If
%somebody else also tries to do this, there won't necessarily be any
%coherence.

A helpful way to classify languages with some kind of generic programming
support is to look at which language constructs are generic. For example,
in C++ the syntax \texttt{object.method(x)} is generic: a programmer can
get it to do different things by supplying different values for
\texttt{object}. The C++ syntax \texttt{f(a,b)} or \texttt{a+b} is
usually not generic, but can be overloaded by supplying definitions for
different argument types. However, this overloading only uses compile time
types (no run time information), and so is essentially a form of renaming ---
it can be implemented by renaming each definition with a unique name, and
replacing overloaded calls with an appropriate name based on compile time
types. This is a much weaker form of generic programming, since different
behaviors cannot be obtained by passing different values at run time.

For this reason, C++ programs are not fully generic: it is difficult to
substitute new behaviors for every part of a program. Some languages
take generic programming much further. For example, in MATLAB, \emph{every}
function is effectively generic, and can be overloaded by new classes.
This enables a programmer to ``intercept'' every function call (and therefore,
essentially everything a program does)


\subsection{Example: lattices}

This example will illustrate possible benefits of multiple dispatch and
dynamic typing for mathematical computing.
The benefits are not absolute, but notational and semantic:
they involve code size and clarity, and the extent to which the entities
provided by the language match a mental model of the domain.

A \emph{lattice} is an algebraic structure where some pairs of elements
satisfy a reflexive, antisymmetric, and transitive relation $\leq$.
For purposes of this example, we will consider lattices that have
a greatest, or \emph{top}, element ($\top$), and a least, or \emph{bottom}
element ($\bot$). When working with lattices one often wants to compute
a least upper bound, or \emph{join} ($\sqcup$), or a greatest lower bound,
or \emph{meet} ($\sqcap$).

Several interesting concerns arise when modeling lattices in a programming
language. First, the structure is very general, and so admits implementations
for many different kinds of elements. We want to write code using
the operators $\leq$, $\sqcup$, and $\sqcap$, and have it apply to any kind
of lattice. Therefore some kind of overloading or object-oriented programming
is desirable. Second, some properties apply to all lattices, and we would
like to avoid implementing them repeatedly.

Using ``duck typing'', the problem of modeling an abstraction like lattices
disappears almost entirely. One may simply define methods for
$\leq$, $\sqcup$, and $\sqcap$ at any time, for any type, and that type will
function as a lattice. That is certainly convenient, but it also fails to
provide any reusable functionality for those defining lattices.



\begin{figure}[!t]
\todo[inline]{what does $===$ mean here?}  
  \begin{center}
\begin{singlespace}
\begin{lstlisting}[language=julia]
abstract LatticeElement

<=(x::LatticeElement, y::LatticeElement) = x===y
==(x::LatticeElement, y::LatticeElement) = x<=y && y<=x
< (x::LatticeElement, y::LatticeElement) = x<=y && !(y<=x)

immutable TopElement <: LatticeElement; end
immutable BotElement <: LatticeElement; end

const ~$\top$~ = TopElement()
const ~$\bot$~ = BotElement()

<=(::BotElement, ::TopElement) = true
<=(::BotElement, ::LatticeElement) = true
<=(::LatticeElement, ::TopElement) = true

~$\sqcup$~(x::LatticeElement, y::LatticeElement) =  # join
    (x <= y ? y : y <= x ? x : ~$\top$~)

~$\sqcap$~(x::LatticeElement, y::LatticeElement) =  # meet
    (x <= y ? x : y <= x ? y : ~$\bot$~)
\end{lstlisting}
\end{singlespace}
  \end{center}
  \label{julialattices}
  \caption{A small Julia library for lattices}
\end{figure}





Figure~\ref{julialattices} shows a small Julia library for lattices.
It defines an abstract class \texttt{LatticeElement} that may be subclassed
by objects that will be used primarily as elements of some lattice.
The library also provides standard
\texttt{LatticeElement} provides some useful default method definitions.\todo{sentence grammar?}
\todo[inline]{Join of two incomparables in general lattices does not have to be top}



\subsection{Symbolic programming}

There has always been an essential divide between ``numeric'' and ``symbolic''
languages in the world of technical computing. To many people the
distinction is fundamental, and we should happily live with both
kinds of languages. But if one insists on an answer as to which approach
is the right one, then the answer is: \todo{Love it!}  symbolic.

Systems based on symbolic rewrite rules arguably occupy a further tier of
dispatch sophistication. In these systems, you can dispatch on essentially
anything, including arbitrary values and structures. These systems are
typically powerful enough to concisely define the kinds of behaviors we
are interested in.

However, symbolic programming lacks data abstraction: the concrete
representations of values are exposed to the dispatch system
(e.g. there is no difference between being a list and being something
represented as a list).
\todo{Wish you would say more!}

\subsection{Predicate dispatch}

%Patterns are very powerful, but the tradeoff is that there is not
%necessarily a useful relationship between what your program does and
%what a static analysis (based on a finite-height partial order over
%patterns) can discover. Maybe julia could be considered a sweet spot
%somewhere in between.

Predicate dispatch is a powerful object-oriented mechanism that allows
methods to be selected based on arbitrary predicates \cite{ErnstKC98}.
It is, in some
sense, the most powerful \emph{possible} dispatch system, since any
computation may be done as part of method selection.
Since a predicate denotes a set (the set of values for which it is true),
it also denotes a set-theoretic type.
Some type systems of this kind, notably that of Common
Lisp~\cite{steele1990common:types}, have actually included predicates as types.
However, such a type system is obviously undecidable, since it
requires computing the predicates themselves or, even worse, computing
predicate implication.\footnote{
Many type systems involving bounded quantification, such as system $F_{<:}$,
are already undecidable \cite{Pierce1994131}.
However, they seem to terminate for most practical programs, and also admit
minor variations that yield decidable systems \cite{Castagna:1994:DBQ:174675.177844}.
It is fair to say they are ``just barely'' undecidable, while predicates
are ``very'' undecidable.
}

For a language that is willing to do run time type checks anyway, the
undecidability of predicate dispatch is not a problem.
Interestingly, it can also pose no problem for \emph{static} type systems
that wish to prove that every call site has an applicable method.
Even without evaluating predicates, one can prove that the available methods
are exhaustive (e.g. methods for both $p$ and $\neg p$ exist).
In contrast, and most relevant to this thesis, predicate types \emph{are} a
problem for code \emph{specialization}.
Static method lookup would require evaluating the predicates, and optimal code
generation would require understanding something about what the predicates mean.
One approach would be to include a list of satisfied predicates in a type.
However, to the extent such a system is decidable, it is essentially equivalent
to multiple inheritance.
Another approach would be to separate predicates into a second ``level'' of the
type system.
The compiler would combine methods with the same ``first level'' type, and then
generate branches to evaluate the predicates.
Such a system would be very useful, and could be
% TODO cite, as this has probably been done
combined with a language like Julia or, indeed, most object-oriented
languages.
However this comes at the expense of making predicates second-class
citizens of the type system.

In considering the problems of predicate dispatch for code specialization,
we seem to be up against a fundamental obstacle: some sets of values are
simply more robust under evaluation than others.
Programs that map integers to integers abound, but programs that map, say,
even integers to even integers are rare to the point of irrelevance.


\section{Domain theory}

In the late 1960s Dana Scott and Christopher Strachey asked how to assign
meanings to programs, which otherwise just appear to be lists of symbols
\cite{scott1971toward}.
For example, given a program computing the factorial function, we
want a process by which we can assign the meaning ``factorial'' to it.
This led to the invention of domain theory, which can be interpreted
as modeling the behavior of a program without running it.
A ``domain'' in this theory is a
partial order of sets of values that a program might manipulate. 
Domain theory models computation as follows: a program starts with no
information, the lowest point in the partial order (``bottom'').
Computation steps accumulate information, gradually moving higher through
the order. The advantage of this model, in essence, is that it provides a
way to think about the meaning of a program without running the program.
Even the ``result'' of a non-terminating program has a representation ---
the bottom element. Other elements of the partial order might refer to
intermediate results.

The idea of ``running a program without running it'' is of great interest
in compiler implementation. A compiler would like to discover as much
information as possible about a program without running it, since
running it might take a long time, or produce side effects that the
programmer does not want yet, or, of course, not terminate.
The general recipe for doing this is to design a partial order (lattice)
that captures program properties of interest, and then describe all
primitive operations in the language based on how they transform
elements of this lattice.
Then an input program can be executed, in an approximate sense, until
it converges to a fixed point.
For example, given a program that outputs an integer, we might decide
that we only care whether this integer is even or odd. Then our posets
are the even and odd integers, and we will classify operations in the
program according to whether they are evenness-preserving,
evenness-inverting, always even, always odd, of uncertain evenness,
etc. % TODO figure
Most modern compilers use a technique like this
(sometimes called abstract interpretation \cite{abstractinterp})
to semi-decide
interesting properties like whether a variable is a constant, whether
a variable might be used before it is initialized, whether a variable's
value is never used, and so on.
%Domain theory gave rise to the study of denotational semantics and the
%design of type systems. However, the original theory is quite general
%and invites us to invent any domains we wish for any sort of language.
%Abstract interpretation \cite{abstractinterp} is an especially
%elegant and general implementation of this idea.
%It should be clear that this sort of analysis, while clearly related
%to type systems, is fairly different from what most programmers
%think of as type \emph{checking}.

Given the general undecidability of questions about programs, analyses
like these are \emph{conservative} in nature.
For example, if our goal is to warn programmers about use of uninitialized
variables, we want to be ``better safe than sorry'' and print a warning
if such a use \emph{might} occur.
Such a use corresponds to any lattice element greater than or equal to
the element representing ``uninitialized''.
Such conservative analyses are the essence of compiler transformations
for performance (optimizations): we only want to perform an optimization
if the program property it relies on holds for sure, and not if there is
any uncertainty.

The generality of this approach makes it a kind of ``ur type system'':
we can discover a large variety of program properties as long as we are
willing to accept some uncertainty.
Of course, many programmers and language designers prefer to maximize
safety, leading to different approaches that trade away some precision
(e.g. syntactic type systems such as those in the ML language
family \cite{hindley1969principal}, \cite{MLtypeinf}).
But if static guarantees are not the priority, or if a language considers
itself unconcerned with ``types'', then we can take the domain-theoretic
model as our type system.
In some sense, it is our type system whether we like it or not
(as implied by \cite{scott1976data})!

\subsection{Monotonicity}

When this kind of type system is used, it usually becomes part of the
meta-theory of the language: values are ``lifted'' into an abstract
domain, and abstract values are manipulated only by a compiler
or other program analysis.
One good reason for this is that the abstract values (which we will
now interchangeably call ``types'') must meet certain requirements in
order to be useful.
For an abstract interpretation to terminate, its lattice must be of
finite height, and all functions of lattice elements must be
monotonic (given lattice elements $a$ and $b$, we must have
$a\subseteq b$ implies $f(a)\subseteq f(b)$ for all $f$).


Functions of types are not monotonic. \texttt{isa(2,3)} is \texttt{false},
but \texttt{isa(2,Int)} is \texttt{true}.
This provides one way to separate ``real types'' from set-valued terms:
if there is a way to interpret execution over types such that all functions
are monotonic in the (alleged) type lattice, then we are really dealing
with types.
One way to do this is to specify a genuinely multi-valued term language
like $\lambda_\aleph$ \cite{Glew:2013:MLD:2502409.2502412}.
In this approach a term like \texttt{isa(2,Int)} is interpreted as a
union of all results obtained by substituting all possible single values
for each sub-term. One cannot help but be impressed by such a language,
though it seems prohibitively difficult to actually implement.
%$\lambda_\aleph$ was introduced a couple years after we began work on Julia
We opt instead for a simpler approach where \texttt{Int} is
interpreted as \texttt{Type\{Int\}} (in a near-borrowing of notation from
\cite{cardelli1986polymorphic} and \cite{Glew:2013:MLD:2502409.2502412}),
and multi-valued terms exist only statically within a separate
maximum-fixed-point evaluator.

\section{Subtyping}

Something about semantic subtyping and type systems for processing XML.
XML at first seems unrelated to numerical computing, and indeed it
was quite a while before we discovered these papers and noticed the
overlap. However if one substitutes ``symbolic expression'' for
``XML document'', the similarity becomes clearer.

% maybe subarray as an example of a fancy type here
\begin{singlespace}
\begin{lstlisting}[language=julia]
const NonSliceIndex = Union(Colon, Range{Int}, UnitRange{Int}, Array{Int,1})
const ViewIndex = Union(Int, NonSliceIndex)
immutable SubArray{T,N,P<:AbstractArray,I<:Tuple{Vararg{ViewIndex}},LD} <: AbstractArray{T,N}
\end{lstlisting}
\end{singlespace}


%% \subsection{Objections to dynamic typing}

% there are run time things, and just flow-sensitive things

% coming up with examples where something needs to be done at
% run time seems to be difficult. as soon as a concrete example
% is raised, one immediately imagines doing it statically.


%% it may be that the ``power'' of a language is equal to the complexity
%% of the criteria used by the language's run time dispatch mechanisms.

%% pre-OO: pointer indirection
%% OO: single dispatch, class hierarchies

%% Haskell: without typeclasses, a beautiful language, but one that nobody
%% would use.

%% I claim that compile time abstractions do not count. The problem is that
%% at some point you have to *actually run the program*. Specifying the
%% behavior of a program is hard; this is where we need help from the
%% language.

%% Scripting languages are often defended using the observation that most
%% code is not performance-critical. However, this fact does not mean that
%% it's ok for a language design to ignore performance, nor does it mean
%% that performance should be the default priority of every line of code.
%% Rather it means that the default should be convenience and safety, with
%% a path to performance easily in reach for when it's needed.
