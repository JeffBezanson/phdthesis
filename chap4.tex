\chapter{The Julia approach}

\section{Type system}

Our goal is to design a type system useful for describing method applicability,
and (similarly) for describing classes of values for which to specialize code.
Set-theoretic types are a natural basis for such a system.
A set-theoretic type is a symbolic expression that denotes a set of values.
In our case, these correspond to the sets of values methods are intended to apply
to, or the sets of values supported by compiler-generated method specializations.
Since set theory is widely understood, the use of such types tends to be intuitive.

These types
are less coupled to the languages they are used with, since one may design
a value domain and set relations within it without yet considering how types
relate to program terms (TODO cite Castagna). Since our goals only include
performance and expressiveness, we simply skip the later steps for now, and do
not address how to type-check terms (or, indeed, the question of whether checking
is even possible).

To avoid the dual traps of ``wasted power'' and divergence, the system we use
must have a decidable subtype relation, and must be closed under data-flow operations
(meet, join, and widen). It must also lend itself to a reasonable definition of
specificity, so that methods can be ordered automatically (a necessary property for
extensibility). These requirements are fairly strict, but still admit many possible
designs. The one we present here is aimed at providing the minimum level of
sophistication needed to yield a language that feels ``powerful'' to most modern
programmers. Beginning with the simplest possible system, we added features as
needed either to satsify the aforementioned closure properties, or to allow us to
write method definitions that seemed particularly useful (as it turns out, these
two considerations lead to essentially the same features). The presentation that
follows will partially reproduce the order of this design process.

We will define our types by formally describing their denotations as sets.
We use the notation $\llbracket T \rrbracket$ for the set denotation of
type expression $T$.
Concrete language syntax and terminal symbols of the type expression grammar
are written in typewriter font, and metasymbols are written in mathematical italic.
First there is a universal type \texttt{Any}, an empty type \texttt{Bottom}, and
a partial order $\leq$:

\vspace{-2ex}
\begin{align*}
  \llbracket \texttt{Any} \rrbracket &= \mathcal{D} \\
  \llbracket \texttt{Bottom} \rrbracket &= \emptyset \\
  T \leq S &\Leftrightarrow \llbracket T \rrbracket \subseteq \llbracket S \rrbracket
\end{align*}

\noindent
where $\mathcal{D}$ represents the domain of all values.

Next we add data objects with structured tags.
The tag of a value is accessed with \texttt{typeof(x)}.
Each tag consists of a declared type name and some number of sub-expressions,
written as \texttt{Name\{}$E_1, \cdots, E_n$\texttt{\}}.
The center dots ($\cdots$) are meta-syntactic and represent a sequence of expressions.
Tag types may have declared supertypes (written as \texttt{super(T)}).
Any type used as a supertype must be declared as abstract, meaning it
cannot have direct instances.

\vspace{-2ex}
\begin{align*}
  \llbracket \texttt{Name\{}\cdots\texttt{\}} \rrbracket &= \{ x\mid \texttt{typeof(}x\texttt{)} = \texttt{Name\{}\cdots\texttt{\}} \} \\
  \llbracket \texttt{Abstract\{}\cdots\texttt{\}} \rrbracket &= \bigcup_{\texttt{super(}T\texttt{)} = \texttt{Abstract\{}\cdots\texttt{\}}} \llbracket T \rrbracket
\end{align*}

These types closely resemble the classes of an object-oriented language with
generic (parametric) types, invariant type parameters, and no concrete inheritance.
We prefer parametric \emph{invariance} for reasons that have been addressed in the
literature \cite{Day:1995:SVC:217838.217852}.
Invariance preserves the property that the only subtypes of a concrete type are \texttt{Bottom}
and itself. We also find that most uses of covariance are more flexibly
handled by union type connectives, which will be introduced below.

Next we add conventional product (tuple) types, which are used to represent the
arguments to methods. These are almost identical to the nominal types defined above,
but are different in two ways: they are \emph{covariant} in their parameters, and permit
a special form ending in three dots (\texttt{...}) that denotes any number of trailing
elements:

\vspace{-2ex}
\begin{align*}
  \llbracket \texttt{Tuple\{}P_1,\cdots,P_n\texttt{\}} \rrbracket &= \prod_{1\leq i \leq n} \llbracket P_i \rrbracket \\
  \llbracket \texttt{Tuple\{}\cdots,P_n\texttt{...\}} \rrbracket, n\geq 1 &= \bigcup_{i\geq n-1} \llbracket \texttt{Tuple\{}\cdots,P_n^i\texttt{\}} \rrbracket
  %\llbracket \texttt{Tuple\{}\cdots\texttt{\}} \rrbracket \cup \llbracket \texttt{Tuple\{}\cdots,P_n\texttt{\}} \rrbracket \cup \llbracket \texttt{Tuple\{}\cdots,P_n,P_n\texttt{...\}} \rrbracket \\
\end{align*}

\noindent
$P_n^i$ represents $i$ repetitions of the final element $P_n$ of the type expression.

The abstract tuple types ending in \texttt{...} correspond to variadic methods, which
provide convenient interfaces for tasks like concatenating any number of arrays.
Multiple dispatch has been formulated as dispatch on tuple types before \cite{Leavens1998}.
This formulation has the advantage that \emph{any} type that is a subtype of a
tuple type can be used to express the signature of a method. It also makes the system
simpler, since subtype queries can be used to ask questions about methods.

The types introduced so far would be perfectly sufficient for many programs, and are
roughly equal in power to several multiple dispatch systems that have been designed
before. However, these types are not closed under data-flow operations. For example,
when the two branches of a conditional expression yield different types, a program
analysis must compute the union of those types to derive the type of the conditional.
The above types are not closed under set union. We therefore add the following
type connective:

\[
  \llbracket \texttt{Union\{}A,B\texttt{\}} \rrbracket = \llbracket A \rrbracket \cup \llbracket B \rrbracket \\
\]

As if by coincidence, \texttt{Union} types are also tremendously useful for expressing
method dispatch. For example, if a certain method applies to all 32-bit integers regardless
of whether they are signed or unsigned, it can be specialized for \texttt{Union\{Int32,UInt32\}}.

\texttt{Union} types are easy to understand, but complicate the type system considerably.
To see this, notice that they provide an unlimited number of ways to rewrite any type.
For example a type \texttt{T} can always be rewritten as \texttt{Union\{T,Bottom\}}, or
\texttt{Union\{Bottom,Union\{T,Bottom\}\}}, etc. Any code that processes types must
``understand'' these equivalences. \texttt{Union} types also commute with covariant
type constructors (tuples in our case), providing even more ways to rewrite types:

\[
\texttt{Tuple\{Union\{A,B\},C\}} = \texttt{Union\{Tuple\{A,C\},Tuple\{B,C\}\}}
\]

This is one of a few reasons that union types are often considered undesirable.
When used with type inference, such types can grow without bound, possibly leading
to slow or even non-terminating compilation. Their occurrence also typically
corresponds to cases that would fail most static type checkers. Yet from the
perspectives of both data-flow analysis and method specialization, they are
perfectly natural and even essential \cite{Igarashi} \cite{Smith:2008:JTI:1449764.1449804}
(TODO cite analyses that have used union types).

The next problem we need to solve arises from combining data-flow analysis
and parametric invariance. When a type constructor \texttt{C} is applied to a type
$S$ that is known only approximately at compile time, the type \texttt{C\{}$S$\texttt{\}}
does not correctly represent the result if \texttt{C} is invariant. The correct
result would be the union of all types \texttt{C\{}$T$\texttt{\}} where $T\leq S$.
Interestingly, there is again a corresponding need for such types in method
dispatch. Often one has, for example, a method that applies to arrays of any
kind of integer (\texttt{Array\{Int32\}}, \texttt{Array\{Int64\}}, etc.).
These cases can be expressed using a \texttt{UnionAll} connective, which denotes
an iterated union of a type expression for all values of a parameter in a specified
range:

\[
  \llbracket \texttt{UnionAll }L\texttt{<:T<:}U\ A \rrbracket = \bigcup_{L \leq T \leq U} \llbracket A[T/\texttt{T}] \rrbracket
\]

% TODO: The inclusion of lower bounds probably makes subtyping undecidable,
% since it can encode contravariance. Needs to be removed.

This is equivalent to an existential type \cite{boundedquant};
for each concrete subtype of it there exists a corresponding $T$.
Anecdotally, programmers often find existential types confusing.
We prefer the union interpretation because we are describing sets of values;
the notion of ``there exists'' can be semantically misleading since it sounds like
only a single $T$ value might be involved.

%Conjecture: these types are intuitive to dispatch on because they correspond
%to program behavior in the same way that dataflow analysis approximates program
%behavior.

% $T=S \longleftrightarrow (T\leq S) \wedge (S\leq T)$.

\subsection{Examples}

\texttt{UnionAll} types are quite expressive. In combination with nominal
types they can describe groups of containers such as
\texttt{UnionAll T<:Number Array\{Array\{T\}\}} (all arrays of arrays of
some kind of number) or
\texttt{Array\{UnionAll T<:Number Array\{T\}\}} (an array of arrays of
potentially different types of number).

In combination with tuple types, \texttt{UnionAll} types provide powerful
method dispatch specifications. For example
\texttt{UnionAll T Tuple\{Array\{T\},Int,T\}} matches three arguments:
an array, an integer, and a value that is an instance of the array's
element type. This is a natural signature for a method that assigns a
value to a given index within an array.


\subsection{Type constructors}

It is important for any proposed high-level technical computing language to be
simple and approachable, since otherwise the value over established
powerful-but-complex languages like C++ is less clear.
In particular, type parameters raise usability concerns.
Needing to write parameters along with every type is verbose, and requires users
to know more about the type system and to know more details of particular
types (how many parameters they have and what each one means).
Furthermore, in many contexts type parameters are not directly relevant.
For example, a large amount of code operates on \texttt{Array}s of any
element type, and in these cases it should be possible to ignore type parameters.

Consider \texttt{Array\{T\}}, the type of arrays with element type \texttt{T}.
In most languages with parametric types, the identifier \texttt{Array} would
refer to a type constructor, i.e. a type of a different \emph{kind} than
ordinary types like \texttt{Int} or \texttt{Array\{Int\}}.
Instead, we find it intuitive and appealing for \texttt{Array} to refer to
any kind of array, so that a declaration such as \texttt{x::Array} simply
asserts \texttt{x} to be some kind of array. In other words,

\[
\texttt{Array} = \texttt{UnionAll T Array$^\prime$\{T\}}
\]

\noindent
where \texttt{Array$^\prime$} refers to a hidden, internal type constructor.
The \texttt{\{ \}} syntax can then be used to instantiate a \texttt{UnionAll}
type at a particular parameter value.

\subsection{Associated types and type computation}

\subsection{Subtyping}

Describe algorithm.

Very likely $\Pi_2^{\textrm{P}}$-hard.
Checking a subtype relation with unions requires checking that for all choices
on the left, there exists a choice on the right that makes the relation hold.
This matches the quantifier structure of 2-TQBF problems of the form
$\forall x_i . \exists y_i . F$ where $F$ is a logical formula. If the formula
is rewritten in conjunctive normal form, it corresponds to subtype checking
between two tuple types, where the relation must hold for each pair of
corresponding types. Now use a type \texttt{N\{}$x$\texttt{\}} to
represent $\neg x$. The clause $(x_i \vee y_i)$ can be translated to
$x_i$\texttt{ <: Union\{N\{}$y_i$\texttt{\}, True\}} (where the $x_i$ and
$y_i$ are type variables bound by \texttt{UnionAll} on the left and right,
respectively).


\subsection{Type system variants}

features that are fairly straightforward to add:

\vspace{-2ex}
\begin{singlespace}
\begin{itemize}
\item structurally-subtyped records
\item mu-recursive types (regular trees)
\item regular types (allowing ... in more places)
\end{itemize}
\end{singlespace}

\noindent
features that are difficult to add, or possibly break decidability:

\vspace{-2ex}
\begin{singlespace}
\begin{itemize}
\item arrow types
\item negations
\item intersections, multiple inheritance
\item universal quantifiers
\item lower bounds in quantifiers
\item arbitrary predicates, theory of natural numbers, etc.
\end{itemize}
\end{singlespace}

\section{Dispatch mechanism}

\section{Data model}

\section{Performance model}

\subsection{Type inference}

\subsection{Specialization}
