\chapter{Introduction}

Scientific computing has evolved from being essentially the only kind of
computing that existed, to being a small part of how and why computers
are used.
Over this period of time, programming tools have advanced in terms of
the abstractions and generalizations they are capable of.

Science as a whole evolves through the incorporation of specialized bits
of knowledge into more powerful general theories.
There is a parallel in the world of programming languages:\ special-purpose
languages have been subsumed by more powerful and general
languages.
How has this trend affected scientific computing?
The surprising answer is:\ not as much as we would like.
We see scientists and technical users generally either turning away
from the best new programming languages, or else pushing them
to their limits in unusual ways.

\begin{singlespace}
\begin{figure}
  \begin{center}
    \begin{tabular}{|llll|l|}\hline
      \multicolumn{4}{|c|}{Productivity} & Performance \\
      \hline
      Matlab  &  Maple &  Mathematica & SciPy & Fortress\\
      SciLab  &  IDL   &  R  & Octave         & Chapel \\
      S-PLUS  & SAS & J & APL                 & X10 \\
      Maxima & Mathcad & Axiom & Sage         & SAC \\
      Lush & Ch & LabView & O-Matrix          & ZPL \\
      PV-WAVE & Igor Pro & OriginLab & FreeMat &\\
      Yorick & GAUSS & MuPad & Genius &\\
      SciRuby & Ox & Stata & JLab &\\
      Magma & Euler & Rlab & Speakeasy &\\
      GDL & Nickle & gretl & ana &\\
      Torch7 & A+ & PDL & Nial & \\
      \hline
    \end{tabular}
  \end{center}
  \caption[49 technical computing languages]{
\small{
    49 technical computing languages classified as primarily designed for productivity or performance
}
  }
  \label{gangof40}
\end{figure}
\end{singlespace}

In fact, we have discovered at least \emph{49} programming languages
designed for technical computing since Fortran (figure~\ref{gangof40}).
This high number begs for an explanation.
We propose that technical users crave the flexibility to pick notation
for their problems, and language design --- the highest level of
abstraction --- is where you go when you need this level of flexibility.
%Language design is associated with the highest level of abstraction.
The desire for scientists to rework their code is not so surprising when one reflects
on the tradition of inventing new notation in mathematics and science.
%This certainly
%helps distinguish scientific programming from other application areas.

The need for language-level flexibility is corroborated by
the ways that the technical computing domain uses general purpose
languages.
Effective scientific libraries extensively employ
polymorphism, custom operators, and compile time abstraction.
Code generation approaches (writing programs that write programs)
are unusually common.
Most of these techniques are presently fairly difficult to use, and so
programmers working at this level give up the notational conveniences
of the purpose-built languages above.

The goal of this thesis is to identify and address the fundamental
issues that have made productive and performant technical computing
so difficult to achieve.
%We attempt to make the smallest
%change that can be provided by a simple high-level language
We ask:\ which general purpose language would \emph{also} be able to handle
technical computing, such that the above workarounds would not have
been necessary?

% our theory is that t.c. is not just about matrices and numbers
% but a specific pattern of code specialization and dispatch that
% can be automated.

\iffalse
Compiler techniques, library design, high-performance
computational kernels, new algorithms, and approaches to parallelism are
all important to technical computing.
However these sorts of technologies can usually be applied to multiple
languages, as has happened in the C and Fortran language families.
%They do not by themselves insist that a new language be developed.
However in conjunction with a new language, these technologies can flourish.
Indeed, we believe that language design
has reached a level of importance that even (perhaps temporarily) supersedes
the traditional focus of cache utilization and the use of matrix operations.
\fi

In brief, our answer is to \emph{integrate code selection and specialization}.
The aforementioned flexibility requirement can be explained as
a sufficiently powerful means for \emph{selecting} which piece of code
to run.
%This notion subsumes method dispatch, function overloading,
%and potentially even branches.
When such mechanisms are used in technical computing, there is nearly
always a corresponding need to \emph{specialize} code for specific cases
to obtain performance.
Polymorphic languages sometimes support forms of specialization,
but often only through a designated mechanism (e.g.\ templates), or
deployed conservatively out of resource concerns.
We intend to show that an equilibrium point in the design space can be found
by combining selection and specialization into a single appropriately designed
dynamic type-based dispatch mechanism.
%The specific design is identified via subtyping theories and closure under
%data flow operations.


%\subsubsection{Thesis statement}

%Integrating code selection and specialization using
%type-based dynamic dispatch captures the performance and
%productivity requirements of technical computing.


% why dynamic dispatch?
%% somewhat counter-intuitively, dynamic dispatch can be good for performance
%% since it permits invoking the most specialized possible method.
%% static overloading can lead to calling a sub-optimal case when multiple
%% overloads exist for the sake of performance.

%Extensive code specialization is a key feature of
%technical computing. ``what to specialize on'' has been an open problem.
%our types are a possible solution to this for two reasons:

%the amazing thing about programming languages it that a better
%explanation can directly lead to better performance!)

%% Specialization entails program analysis. By necessity, a language's
%% code selection mechanisms must inform this process, to ensure the
%% correctness of the analysis. A central argument of this thesis is that
%% specialization should feed back into selection, allowing the
%% \emph{approximate values} processed by a compiler to be used in the
%% source language for dispatch. We argue that this is the key feature
%% missing from the present generation of dynamically typed languages.
%This conclusion would not be surprising to advocates of static
%type systems, but the approach we propose is actually a run time
%mechanism that does not restrict the class of valid programs.

%Why integrate code specialization and code selection?
%\begin{itemize}
%\item specialization requires selection anyway
%\item simpler system
%\item can sometimes collapse 2 layers of dispatch into 1
%\item can replace library code with a code generator without either changing
%  client code *or* any extra overhead
%\end{itemize}

%Instead we intend to argue that, at a
%deeper level, it is about certain kinds of flexibility, particularly
%flexibility in the behavior of key operators and functions. This flexibility
%is needed to maximize composability and reusability in scientific code.
%Without it, certain programs may be easy to write today, but changing
%functional and performance requirements can become difficult to meet.

\section{The technical computing problem}
\label{sec:tcproblem}

\begin{singlespace}
\begin{table}
  \begin{center}
    \def\arraystretch{1.25}
    \begin{tabular}{|l|l|}\hline
      \textbf{Mainstream PL} & \textbf{Technical computing} \\
      \hline \hline
      classes, single dispatch             &  complex operators \\
      \hline
      separate compilation                 &  performance, inlining \\
      \hline
      parametric polymorphism              &  ad hoc polymorphism, extensibility \\
      % - concrete reuse of generated code
      \hline
      static checking                      &  experimental computing \\
      \hline
      modularity, encapsulation            &  large vocabularies \\
      \hline
      eliminating tags                     &  self-describing data, acceptance of tags \\
      \hline
      data hiding                          &  direct manipulation of data \\
      \hline
    \end{tabular}
  \end{center}
  \caption[Programming language priorities]{
\small{
Priorities of mainstream object-oriented and functional programming language research and
implementation compared to those of the technical computing domain.
}
  }
  \label{PLpriorities}
\end{table}
\end{singlespace}

Table~\ref{PLpriorities} compares the general design priorities of mainstream
programming languages to those of technical computing languages.
The priorities in each row are not necessarily opposites or even mutually exclusive,
but rather are a matter of emphasis.
It is certainly possible to have both parametric and ad hoc polymorphism within
the same language, but syntax, recommended idioms, and the design of the standard
library will tend to emphasize one or the other.
Of course, the features on the left side can also be useful for technical computing;
we exaggerate to help make the point.

It is striking how different these priorities are.
We believe these technical factors have contributed significantly to the persistence
of specialized environments in this area.
Imagine you want just the features on the left.
Then there are many good languages available:\ Haskell, ML, Java, C\#, perhaps C++.
Getting everything on the right, by comparison, seems to be awkward.
The most popular approach is to use multiple languages, as in NumPy~\cite{numpy},
with a high-level productivity language layered on top of libraries
written in lower-level languages.
Seeking a similar trade-off, others have gone as far as writing a C++
interpreter \cite{vasilev2012cling}.

% something about complicated architecture and not-always-good performance
%technical computing systems have an unusually large amount of
%``failure of abstraction'' --- manual duplication of facts
%all over the system. imagine changing from column-major to
%row-major.

%problem of finding connections between array programming and OOP.
%array programming is a powerful paradigm for describing computational
%kernels operating over potentially large amounts of data.

%an ``object system'' in this context is often considered a separate
%part of the language, to be used only when arrays no longer
%suffice.

It is clear that any future scientific computing language will need to be able to
match the performance of C, C++, and Fortran.
To do that, it is almost certain that speculative optimizations such as tracing
\cite{tracingjit} will not be sufficient ---
the language will need to be able to \emph{prove} facts about types, or at least
let the user request specific types.
It is also clear that such a language must strive for maximum convenience, or else
the split between performance languages and productivity languages will persist.

It is fair to say that two approaches to this problem are being tried:\ one is
to design better statically typed languages, and the other is to apply
program analysis techniques to dynamically typed languages.
Static type systems are getting close to achieving the desired level
of flexibility (as in Fortress \cite{fortresspec} or Polyduce \cite{polyduce1},
for instance),
%, but it is still too early to call a winner between these two
%approaches (if, indeed, there even needs to be a winner).
but here we will use a dynamically typed approach.
We are mostly interested in types in the sense of \emph{data types} and
\emph{partial information}
(for an excellent discussion of the many meanings of the word ``type''
see \cite{Kell2014}).
Additionally, some idioms in this domain appear to be inherently, or at least
naturally, dynamically typed (as we will explore in chapter~\ref{chap:2}).
The mechanism we will explore provides useful run time behavior and, we hope,
performance improvements.
Many approaches to statically checking its properties are likely possible,
however we leave them to future work.

%Second, insufficient static checking is most likely not the current limiting
%factor in scientific computing productivity.
%In our view the first priority is to get the combination of
%performance and flexibility we already know users want, and then try to
%improve safety later.


\section{The importance of code selection}

Most languages provide some means for selecting one of many pieces of code
to run from a single point in the program.
This process is often called \emph{dispatch}.
High-level languages, especially ``dynamic'' languages tend to provide a
lot of flexibility in their dispatch systems.
This makes many sophisticated programs easier to express, but at a considerable
performance cost.

There has been a large amount of work on running dynamic languages faster
(notable examples are Self~\cite{selflang}, and more recently PyPy~\cite{pypyjit}).
These efforts have been very successful, but arguably leave something
out:\ a better interface to performance.
% put differently: HLLs excel at providing certain features, but not at
% defining other things like those that still perform well
Performance-critical aspects of most dynamically typed languages (e.g.\ arithmetic)
have fixed behavior, and it is not clear what to do when your code does not
perform as well as it could.
Adding type annotations can work, but raises questions about how powerful
the type system should be, and how it relates to the rest of the language.
We speculate that this relative lack of control keeps experts away,
perpetuating the split between performance and productivity languages.

According to our theory, the issue is not really just a lack of types,
but an exclusive focus on specialization, and not enough on selection.
%Code selection is crucial to performance at all levels: at the lowest
%level, it means picking the best machine instruction,
%and at the highest level it means picking the best algorithm.
In technical computing, it is often so important to select the best
optimized kernel for a problem that it is worth doing a slow dynamic
dispatch in order to do so.
Most of the productivity languages mentioned above are designed
entirely around this principle.
But their view of code selection is incomplete:\ the dispatch
mechanisms used are ad hoc and not employed consistently.
Even when an optimizing compiler for the source language is developed,
it does not necessarily take all relevant forms of dispatch into
account.
For example, NumPy has its own dispatch system underneath Python's,
and so is not helped as much as it could be by PyPy, which only
understands Python method selection.

A dispatch specification, for example a method signature in a
multimethod system, describes when a piece of code is applicable.
%A more powerful dispatch system brings many benefits.
A more expressive dispatch system then allows code to be written
in smaller pieces, helping the compiler remove irrelevant code
from consideration.
It is also more extensible, providing more opportunities to
identify special cases that deserve tailored implementations.
Method-at-a-time JIT compilers
(e.g.\ \cite{grcevski2004java, Suganuma:2005:DED:1075382.1075386})
can specialize method bodies on all arguments, and so are
arguably already equipped to implement and optimize multiple dispatch
(even though most object-oriented languages do not offer it).
At the same time, dispatch specifications are a good way to provide
type information to the compiler.
%For example an optimizing compiler for Self~\cite{Dean:1994:TBI:182590.182489}
%tracked all argument types per call site to help guide inlining.
%Such a system might even use multiple dispatch internally to
%select optimized methods.
%In that case, why not expose that capability to the programmer?

%%%%%%%%%
%compilation is expensive, so traditional languages have forced the equation
%(what you can generate code for) = (known at compile time)
%wanting specialized code for more dynamic properties is the issue.
%%%%%%%%

%how to write things so they perform, and how to
%add algorithms for special cases in an extensible way

%performance is related to dispatch since it lets you describe when some
%efficient kernel is applicable.

% also for linking purposes the compiler will want to leave some
% record of what it has done, i.e. what assumptions it has made to
% generate the code it left behind.
% here you will want a type system.

%% Why do we prefer method signatures to imperative logic?
%% - it's extensible
%% - its declarative style is more compact
%% - allows writing functions in small pieces
%%   the more descriptive the types, the smaller the pieces
%%   might make compiler more efficient; removes large bits of code from
%%   consideration just from looking at types. don't need to analyze a
%%   large function just to discover that only a small part of it
%%   remains.
%% - clearer performance model. it's hard for programmer to guess
%%   when the compiler will statically evaluate an arbitrary expression
%% - allows defining semantics about things that are optimized, but
%%   *not* compile-time constants.
%%   method table can be (ab)used as a mutable lookup table structure.

%Finally, it is often the case that a seemingly minor
%modification to a type system makes type checking undecidable. Rejecting
%such a system is totally reasonable. However if we are willing to
%accept undecidable checking from the beginning, opportunities arise to
%simplify and increase the power of the language.

% one of our main innovations is a way to deftly navigate between two
% problems

%% \subsection{The excess power problem}

%% We claim that the amount of information that can be statically inferred
%% exceeds most dynamic languages' capacity to exploit it.
%% Much work has been done on program analysis and optimization techniques
%% for dynamically typed languages.
%% When static analyses (often incorporating run time information) are applied
%% to dynamically typed programs, it is typically possible to recover a
%% significant amount of type information (e.g. \cite{druby}).
%% What, then, can one do with this information?
%% If the goal is performance, various partial evaluations can be done:
%% generating code without type checks, removing branches, type-specializing
%% the storage of variables, and compile time method lookup are all valuable
%% and yield large real-world gains.

%and might use multiple dispatch \emph{internally} to select implementations
%at run time.
%% Arguably, if method calls are dispatched on the first argument, but the types of all
%% arguments can be inferred, some power has been ``left on the table'' ---
%% we could have had multimethods for little extra cost.
%% This argument does not apply equally to statically typed languages, since they
%% cannot simply ``switch'' their functions to generic functions without significant
%% consequences for type checking.

%% This ``excess power'' problem applies to data structures as well.
%% For example, static or run time analysis might reveal that a certain array
%% can be represented as a native \texttt{Int32} array \cite{Bolz2013}.
%% If this information is not reflected in the source language, then
%% certain uses like passing data to native code become unnecessarily more
%% complicated.
%% And if one is going to implement homogeneous arrays anyway, why not
%% let programmers request them?

%% Some levels of performance are difficult to reach with implicitly
%% specialized code and data.
%% Given the knowledge that an array contains only \texttt{Int32} data,
%% we might want to go beyond essential optimizations like storing intermediate
%% values in registers, and actually use different algorithms.
%% For example, in Miller-Rabin primality testing~\cite{Rabin1980128}, checking
%% three ``witness'' values suffices for all 32-bit arguments, but up to seven
%% values might be needed for 64-bit arguments.
%% In cryptographic applications, exploiting this difference in an inner loop
%% could bring significant benefits.


%\subsection{The divergence problem}

The remaining question is what exactly we should dispatch on.
Naturally, it might be nice to be able to dispatch on any criteria
at all.
However there is a balancing act to perform:\ if dispatch signatures
become too specific, it becomes difficult to predict whether they
will be resolved statically.
At one end of the spectrum are mechanisms like static overloading
in C++, where the compiler simply insists that it be possible to
resolve definitions at compile time.
At the other end of the spectrum, we might dispatch on arbitrary
predicates (as in the Scmutils system~\cite{Sussman:2001:SIC:375178}).
% TODO more about scmutils
We do neither of these.
Instead, we only try to make it \emph{more likely} that types can be
statically inferred.
We do this by considering the kinds of types that arise in data flow
type inference, and allowing dispatch on those.
This will be developed in detail in section~\ref{sec:typesystem}.

%Instead, our dispatch is based on \emph{sets of values that are robust
%under evaluation}: abstract evaluation over such sets is less likely
%to diverge.

% TODO: needs to be clearer
%% For example, consider the set $\{1,2\}$.
%% This set is not closed under ``natural'' functions.
%% As we evaluate a recursive function over this abstract value, the
%% set will typically grow, and the fixed point will be much larger
%% than the original set.
%% So while specializing a method for $\{1,2\}$ could in some rare
%% cases be useful, it also leads to unpredictable performance.
%% To find reasonably robust sets to dispatch on, we
%% begin with structured type tags and take the closure under data flow
%% operations.

%However as a dispatch system gets more powerful, its utility as a
%performance model degrades.
% TODO: introduce before this how dispatch can be a performance model

%% Another problem that occurs when analyzing programs with complex
%% type behavior is divergence: the analysis is likely to infer an
%% overly large result from failing to eliminate enough possible
%% behaviors.
%% Narrowing the inferred types requires some extra source of type
%% information.
%% Multimethod signatures work well for this purpose.

%% past approaches include: optional user annotations, or a built-in
%% database of function type behavior, determined empirically or
%% analytically.
%% to solve the problem in general, library behavior should not be
%% built-in to the compiler.
%% however, when functions that are both highly complex and performance
%% critical are implemented at the library level it can be difficult
%% to ensure they are sufficiently optimized.

% specializer must decide when to widen, when to stop going around
% loops, etc.
% this is an opaque process.
% there is an essential divide between what it will statically
% evaluated, and what it won't.

%% However the divergence problem also places a limit on what
%% kinds of dispatch specifications are useful to program analyses:
%% some sets of values tend to be more robust under execution
%% than others.


% TODO:
% claim dispatch works better on approximate type info than branches
% when you have unionall types.
% dispatch is also extensible and declarative.

% doing analysis only, you can use arbitrarily complex types, but
% you are then limited by what the language can ``consume'', or
% by the sophistication of the t-functions you write.
% adding dispatch, the types become an abstraction under user
% control, so need to be meaningful.

%% what to dispatch on? dispatch power has been extended in many ways, but
%% there is no real limit to what somebody might want to dispatch on.
%% so what to do?
%% some sets of values are more robust under computation than others
%% (closure properties).
%% identify those sets using data flow concerns.

%% say we have a method defined for integers, and also for the special cases
%% ``2'' and ``odd integers''. a realistic implementation
%% will group all of these under ``integer'', and ideally generate a couple
%% branches to handle the other cases. we argue the concept of ``integer''
%% here is a more robust set, and so a more fundamental language concept.

%- vs predicate dispatch: we extend dispatch power in a different direction,
%guided by semantic subtyping. goal is maximum power that still yields high
%likelihood of resolving many calls to a single implementation.


\section{Staged programming}

In demanding applications, selecting the right algorithm might not
be enough, and we might need to automatically \emph{generate} code
to handle different situations.
While these cases are relatively rare in most kinds of programming,
they are remarkably common in technical computing.
Code generation, also known as \emph{staged programming}, raises
several complexities:

\vspace{-3ex}
\begin{singlespace}
\begin{itemize}
\item What is the input to the code generator?
\item When and how is code generation invoked?
\item How is the generated code incorporated into an application?
\end{itemize}
\end{singlespace}

\noindent
These questions lead to practical problems like ad hoc custom syntax,
extra build steps, and excessive development effort (writing parsers and
code generators might take more work than the core problem).
If code needs to be generated based on run time information, then a
% TODO cite julia GPU paper that did this
staged library might need to implement its own dispatch mechanism
for invoking its generated code.

We find that our style of dynamic type-based dispatch provides
an unusually convenient substrate for staged programming.
The reason is that the types we dispatch on contain enough
information to drive code generation for a useful class of problems,
while also allowing
the dispatch mechanism to be reused for caching and invoking staged code.
Code generators are invoked automatically by
the compiler, and the results can be called directly or even inlined
into user code.
Libraries can switch to staged implementations with method-level granularity
without affecting client code.

%past (static) type systems for dispatch were designed to ensure the absence of
%no-method errors and ambiguities (completeness and uniqueness). our goal
%is instead to statically resolve methods. this is inherently heuristic and
%best-effort. since static types shouldn't affect program behavior, we
%conclude that the dispatch must be dynamic, which is happily the same
%conclusion you would reach if you simply wanted dynamic typing.

%it is quite possible that some static type system will work well for this
%however we defer this question.
%interesting variants:
%- require static single method matches
%- reject programs with no-method errors
%- reject programs that yield Unions

%%%%%%

\section{Summary of contributions}

\iffalse
We began with a belief that we could design a technical computing language
sufficiently novel and powerful that it could gain popularity
in real applications.
The key idea was that language level abstractions could bring ease of use,
performance, and wider applicability to technical computation.
It is far from trivial to analyze and design this right set of abstractions
This thesis contributes a deep analysis of our findings.  

At the very core, the novel idea in this
thesis is the very notion that technical computing is amenable to deep language
analysis.
As simple as this may sound, many in the technical computing world did not
understand how a computer science approach to technical computing could be
useful to them.
Our personal experience describing the research plan to
the computer science world might be
characterized by the view that this space consists of library
functions such as FFTs and matmuls and would not be as interesting
to analyze.
Despite the early discouraging viewpoints, five years later,
the power of language analysis is manifesting itself
theoretically, opening new research directions, and practically in that Julia
has worldwide name recognition and worldwide adaption.
\fi

\iffalse
Our first contribution is a discussion of the nature of technical computing
that suggests which language-level abstractions might best support its use
cases.
Simply put technical computing has by and large happened
without sufficient probing and analysis as to what it is.

Based on the theory that technical computing is characterized by complex
operators and particular combinations of binding time behavior.
\fi

At a high level, our contribution is to identify the priorities
that have been implicit in the designs of technical computing systems,
and suggest fundamental changes that would better serve those
priorities.
In the process, we make the following specific contributions:

\begin{itemize}
\item To our knowledge, the first complete technical computing system
developed by applying automatic type specialization over libraries
written in a single high-level language.
%Past systems have applied specialization to a set of user-level routines.
%We show that it can be used for an entire system.

\item A novel dispatch system that supports a form of set-theoretic types
including variadic methods.
We contribute an answer to the question of how powerful a dynamic dispatch
system should be:\ even if type checking is not done, dispatch should still
use types that could be inferred by data flow analysis.
%The resulting system is both more and less powerful than most previous
%dispatch systems.
%should be, by suggesting that they reflect types discoverable by flow
%analysis.
%Interestingly, the resulting system seems to fall just short of the trap
%of undecidable subtyping.

%We introduce the idea of integrated code selection and specialization.
%This leads to a dispatch system that provides a novel combination of
%performance and expressiveness.

%It provides ``diagonal dispatch''
%Using data flow types to drive the design leads to novel dispatch features:
%- diagonal dispatch
%- unifying variadic methods with regular types
%- predictable performance without giving up flexibility

%This leads to easy-to-use polymorphism, prioritizing
%flexibility and performance over nearly all else.

% Something about dispatch for performance. Given a performance kernel, it takes
% a lot of info to describe when it is applicable.

% dispatch on everything, not just classes

% ?? % \item Easy polymorphism

%- library writers can decide what to put at the type level without affecting users,
%  dropping type parameters
%- a single mechanism to cover overloading, dispatch, and specialization
%%%- using types, but mostly for dispatch, where they can be used gradually
%- specialize by default (makes static polymorphism less of a black art)
%- types as ordinary values

\item A subtyping algorithm for our system, including a straightforward
implementation.
This subtyping system is adoptable by dynamic languages that might
benefit from a richer abstract value domain, for example for
describing method applicability or performance-critical aspects of data structures.

% we design a type system for this. and introduce the ``evaluate softly
% and carry a big subtype relation'' school of thought.

% we provide a straightforward implementation that should be easy to
% translate to other languages.

% at this point, certain language implementation techniques have been fairly widely
% disseminated: how to write a parser, how to write a bytecode VM, how to
% design an object system, etc.
% subtyping systems are still relatively esoteric, often not even mentioned
% in introductory texts.

%\item a novel form of staged programming?

\item We demonstrate a novel combination of staged programming with generic
functions that allows staging to be invoked implicitly, leading to improved
usability.


\item We show how to implement key features of technical computing languages
using our dispatch system, eliminating the need for them to be built in.
This preserves flexibility while gaining efficiency and extensibility.
(James Morris eloquently observed that
``One of the most fruitful techniques of language analysis is explication through
elimination.
The basic idea is that one explains a linguistic feature by showing
how one could do without it.''~\cite{morris})

%the application of our approach to
%features of technical computing environments that have not been subject to such
%analysis before.
%We provide an example application that shows how to use our language to
%more easily solve a difficult computational problem.

% hopefully will provide useful examples for evaluating future dispatch
% systems, object systems, and type systems

\item Finally, we contribute the Julia software system, currently a
growing free software project, demonstrating that our approach is both practical
and appealing to a significant number of users.

\end{itemize}
