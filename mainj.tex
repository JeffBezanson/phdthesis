


\chapter{Wholesale copy of SIAM REVIEW PAPER}
\section{Scientific computing languages: The Julia innovation}

The original numerical computing language was Fortran, short for
``Formula Translating System'', released in 1957.  Since those
early days, scientists have dreamed of writing high-level, generic
formulas and having them translated automatically into low-level,
efficient machine code, tailored to the particular data types they
need to apply the formulas to.  Fortran made historic strides towards
realization of this dream, and its dominance in so many areas of
high-performance computing is a testament to its remarkable success.

The landscape of computing has changed dramatically over the years.
Modern scientific
computing environments such as MATLAB~\cite{matlab}, R~\cite{Rlang},
Mathematica~\cite{mathematica}, Octave~\cite{Octave},
Python (with NumPy)~\cite{numpy}, and SciLab~\cite{scilab} have grown in popularity
and fall under the general category
known as { {\it dynamic languages} or {\it dynamically typed languages}.
In these programming
languages, programmers write simple, high-level code without any
mention of types like \code{int}, \code{float} or \code{double} that
pervade {\it statically typed languages} such as  C and Fortran.






Julia is dynamically typed, but it is different as it approaches statically typed performance.
New users can begin working with Julia as they did
in the traditional numerical computing languages, and work their way up when ready.
In Julia,  types are implied by the
computation itself together with input values.  As a result, Julia
programs are often completely generic and compute with data of
different input types without modification---a feature known as
``polymorphism."

% Much of the development of modern programming languages---both static
% and dynamic---can be understood by following the emergence of various
% forms of polymorphism.

% In addition to the increased leverage afforded by
% polymorphism, the The convenience of not needing to mention types or
% satisfy the requirements of a type checker gives dynamic systems a
% major advantage in productivity.

It is a fact that compilers need type information to emit efficient
code for high performance. It is also true that the convenience of not
 typing gives dynamic systems a major advantage in
productivity.  Accordingly, many researchers today do their day-to-day
work in dynamic languages.  Still, C and Fortran remain the gold
standard for computationally-intensive problems because high-level
dynamic languages have lacked  performance.

An unfortunate outcome of the currently popular languages is that the
most challenging areas of numerical computing have benefited the least
from the increased abstraction and productivity offered by higher
level  languages.  A pervasive idea that Julia wishes to
shatter is that numerical computing languages are ``prototyping"
languages and one must switch to C or Fortran for performance.  The
consequences to scientific computing have been more serious than many
realize.

We recommend signing up at \url{juliabox.org} to use Julia
conveniently through the Julia prompt or the IJulia notebook.  (We
like to say ``download,next,next,next" is three clicks too many!)  If
you wish to, you can download Julia at \url{www.julialang.org}, along
with the IJulia notebook.

We wish to show how users can be coaxed into writing better software
that is comfortably familiar, easier to work with, easier to maintain,
and can perform very well.  This paper brings together

\begin{itemize}
\item Examples to orient the new user and
\item The theory and principles that underly Julia's innovative design
\end{itemize}
specifically so that everyday users of numerical computing languages
may understand why Julia represents a truly fresh approach.

\subsection{Computing  Research Transcends Communities}

Numerical computing research has always lived on the boundary of computer science, engineering, mathematics, and computational  sciences.
Readers might enjoy trying to label the ``Top 10 algorithms"\cite{top10} by  field, and may quickly
see that  advances typically transcend any one field  with broader impacts to science
and technology as a whole.

Computing  is more than using an overgrown calculator. It is a cross cutting communication medium.
Research into programming languages  therefore breaks us out of our research boundaries.

The first decade of the 21st century saw a boost in such
research with the High Productivity Computing Systems DARPA funded
projects into the languages Chapel \cite{chapelref}, Fortress \cite{fortressref}
} and  X10 \cite{x10}.



Also contemporaneous   has been a growing acceptance of Python.
Up to around 2009 some of us were working on the  Star-P interactive high performance computing system
for parallelizing primarily MATLAB  but also Python, R, Excel and other high level languages
were researched.
(User Guide: \cite{starpug}, Getting Started: \cite{starpstart}, Papers \cite{starpright,starpstudy,Husbands98inter,Choy04star-p:high}).
Julia  continues our research into parallel computing.



More recently Google, Apple, and Mozilla have introduced
Go \cite{go}, Swift  \cite{swift}, and Rust \cite{rust} respectively as new interesting dynamic languages
which are gaining traction.








\section{A taste of Julia}
\subsection{A brief tour}

Since Julia is a young language, new users may worry that it is somehow
immature or lacking some functionality.  Our experience shows that
Julia is rapidly heading towards maturity.  Most new users are
surprised to learn that Julia has much of the functionality of the
traditional established languages, and a great deal of functionality
not found in the older languages.

New users also want a quick explanation as to why Julia is fast, and
whether somehow the same ``magic dust" could also be sprinkled on
their traditional scientific computing language.  Why Julia is fast is
a combination of many technologies some of which are introduced in
Sections \ref{sec:select} through \ref{sec:lang} in this paper. Julia is fast because we, the
designers, developed it that way for us, the users.  Performance is
fragile, like accuracy, one arithmetic error can ruin an entire
otherwise correct computation.  We do not believe that a language can
be designed for the human, and easily retrofitted for the computer.  Rather a
language must be designed from the start for the human and the
computer.  Section \ref{sec:humancomputer} and \cite{hoareessay}  explores this issue.

Before we get into the details of Julia, we provide a brief tour for
the reader to get a beginners' feel for the language. Users of other dynamic
numerical computing environments may find some of the syntax
familiar. In language design, there is always a trade-off between
making users feel comfortable in a new language vs.\ clean language
design. In Julia, we maintain superficial similarity to existing
environments to ease new users, but not at the cost of sacrificing
good language design or performance.




%For example, this familiar syntax shows how a random matrix is
%created, an identity matrix is added to it, and its inverse is
%computed:

\begin{jinput}
A = rand(3,3) + eye(3)  \sh {\# Familiar Syntax} \\
inv(A)
\end{jinput}
\begin{joutput}
3x3 Array\{Float64,2\}:\\
\hspace*{.02em} 0.698106  -0.393074  -0.0480912 \\
 -0.223584 \hspace*{.02em}  0.819635  -0.124946  \\
 -0.344861 \hspace*{.02em}  0.134927 \hspace*{.02em}  0.601952
\end{joutput}

The output from the Julia prompt says that $A$ is a two dimensional
matrix of size $3 \times 3$, and contains double precision floating
point numbers.

Indexing of arrays is performed with brackets, and
is 1-based. It is also possible to compute an entire array
expression and then index into it, without assigning the expression to
a variable:

\begin{jinput}
x = A[1,2]   \\
y = (A+2I)[3,3]     \sh{    \# The [3,3] entry of  A+2I}
\end{jinput}

\begin{joutput}
2.601952
\end{joutput}

In Julia, \verb+I+ is a built-in representation of the identity
matrix, which can be scaled and combined with other matrix operations,
without explicitly forming the identity matrix as is commonly done
using commands such as ``eye" which unnecessarily requires $O(n^2)$
storage and unnecessary computation in traditional languages and in
the end is a cute but ultimately wrong word for the identity
matrix. The built-in representation is fully implemented in Julia code
in the Julia base library.

Julia has symmetric tridiagonal matrices as a special type.  For
example, we may define Gil Strang's favorite matrix (the second order
difference matrix) in a way that uses only $O(n)$ memory.

\begin{figure}[h]
  \centering
  \includegraphics[width=3in]{cupcakes1.jpg}
\caption{Gil Strang's favorite matrix is {\tt strang(n) =
    SymTridiagonal(2*ones(n),-ones(n-1)) } \newline Julia only stores
  the diagonal and off-diagonal.  (Picture taken in Gil Strang's
  classroom.)  }
\end{figure}

\ja
strang(n) = SymTridiagonal(2*ones(n),-ones(n-1)) \\
strang(7)
\jb
7x7 SymTridiagonal\{Float64\}:    \vspace{-.05in}
\begin{verbatim}
  2.0  -1.0   0.0   0.0   0.0   0.0   0.0
 -1.0   2.0  -1.0   0.0   0.0   0.0   0.0
  0.0  -1.0   2.0  -1.0   0.0   0.0   0.0
  0.0   0.0  -1.0   2.0  -1.0   0.0   0.0
  0.0   0.0   0.0  -1.0   2.0  -1.0   0.0
  0.0   0.0   0.0   0.0  -1.0   2.0  -1.0
  0.0   0.0   0.0   0.0   0.0  -1.0   2.0
  \end{verbatim}
\jc

Julia calls  an efficient $O(n)$ tridiagonal solver:


\ja
strang(8)\textbackslash ones(8)
\jb
8-element Array\{Float64,1\}:   \vspace{-.1in}
\begin{verbatim}
  4.0
  7.0
  9.0
 10.0
 10.0
  9.0
  7.0
  4.0
  \end{verbatim}
\jc


Consider the sorting of complex numbers.   Sometimes it is handy
to have a sort that generalizes the real sort.  This can be done by
sorting first by the real part, and where there are ties, sort by the imaginary
part.  Other times it is handy to use the polar representation, which sorts
by radius than angle.

If a numerical computing language ``hard-wires" its sort to be one or the other,
it misses a wonderful opportunity.  A sorting algorithm need not depend on details
of what is being compared or how it is being compared.  One can abstract away these
details thereby reusing a sorting algorithm for many different situations.  One can specialize
 later.  Thus alphabetizing strings, sorting
real numbers, or sorting complex numbers in two or more ways all run with the same code.


In Julia, one can turn a complex number \verb+w+ into an ordered pair
of real numbers  (a tuple of length 2) such as the Cartesian form  \verb+(real(w),imag(w))+
or the polar form \verb+(abs(w),angle(w))+.  Tuples are then compared lexicographically
in Julia.  The sort command takes an optional ``less-than" operator, \verb+lt+, which is used to compare
elements when sorting.

\ja
\sh{\# Cartesian comparison sort of complex numbers} \\
complex\_compare1(w,z) = (real(w),imag(w)) < (real(z),imag(z)) \\
sort([-2,2,-1,im,1], lt = complex\_compare1 )
\jb
5-element Array\{Complex\{Int64\},1\}: \\
 -2+0im \\
 -1+0im \\
 \hspace*{.02em} 0+1im \\
\hspace*{.02em}  1+0im \\
 \hspace*{.02em} 2+0im
\jc

\ja
\sh{\# Polar comparison sort of complex numbers} \\
complex\_compare2(w,z)  = (abs(w),angle(w)) < (abs(z),angle(z)) \\
sort([-2,2,-1,im,1], lt =  complex\_compare2)
\jb
5-element Array\{Complex\{Int64\},1\}: \\
\hspace*{.02em} 1+0im \\
 \hspace*{.02em} 0+1im \\
 -1+0im \\
\hspace*{.02em}  2+0im \\
 -2+0im
\jc

%In Julia one can define a ``less than" operator for any object, which
%can then be used for sorting. A total ordering of complex numbers is
%not well defined, and hence Julia does not define one by
%default. However, we are free to define our own ordering, and then use
%that to order a series of complex numbers. We show below a general
%approach that can work for \verb+(real,imag)+ or \verb+(abs,angle)+.
%We first define \verb+W+, a function that returns a function as its
%output.  It takes a pair of functions from the complexes to the reals  \verb+(f,g)+ as input and then
%converts a complex \verb+z+ into a real  tuple: \verb+(f(z),g(z))+.  Thus
%\verb+W(real,imag)(z)+ computes \verb+(real(z),imag(z))+.  We can then
%define an \verb+lt+ function that compares any pair of functions we
%like.  Julia already compares tuples lexicographically, so that does
%not need to be defined.

%\begin{jinput}
%\sh{   \# Convert z into a tuple (f(z), g(z)) e.g.\ (real(z),imag(z)) }\\
%\sh{   \# The -> syntax creates an anonymous function } \\
%\sh{\# W(f,g) creates a function that takes a Complex to the tuple}\\
%W(f,g)  = z->(f(z),g(z))  \\

%\sh{\# Create a custom ``less than'' operator }   \\
%\sh{\# lt(f,g) creates a function that takes a tuple to a Boolean} \\
%t(f,g) = (z,w)-> W(f,g)(z) < W(f,g)(w)\\

%\ti{ \# Sort by real part, then  imaginary part} \\
%sort([2,-2,-1,im,1], lt = lt(real,imag) )
%\end{jinput}
%\begin{joutput}
%5-element Array\{Complex\{Int64\},1\}: \\
% -2+0im \\
% -1+0im \\
% \hspace*{.02em} 0+1im \\
%\hspace*{.02em}  1+0im \\
% \hspace*{.02em} 2+0im
  %\end{joutput}
 % \begin{jinput}[4]
%\sh{  \# Sort by absolute value, and then  angle}\\
  %sort([2,-2,-1,im,1], lt = lt(abs,angle) )
  %\end{jinput}
%\begin{joutput}
 % 5-element Array\{Complex\{Int64\},1\}: \\
 %\hspace*{.02em} 1+0im \\
 %\hspace*{.02em} 0+1im \\
 %-1+0im \\
%\hspace*{.02em}  2+0im \\
 %-2+0im
 %\end{joutput}

Note the \verb+Array{ElementType,dims}+ syntax.  It shows up everywhere.
In the above example, the elements are complex numbers whose parts are \verb+Int64+'s.
The \verb+1+ indicates it is a one dimensional vector.

%Extensive
%help is also available for Julia's functions:

%\begin{jinput}[2]
%help(qr)
%\end{jinput}
%\begin{joutput}
%INFO: Loading help data...
%Base.qr(A, [pivot=false,][thin=true]) -> Q, R, [p]

   %Compute the (pivoted) QR factorization of "A" such that either  \\
   %"A = Q*R" or "A[:,p] = Q*R". Also see "qrfact". The default \\
   %s to compute a thin factorization. Note that "R" is not extended \\
   %with zeros when the full "Q" is requested.
   %\end{joutput}


%The QR factorization in Julia yields the Q and R matrices as
%expected. The result of  {\tt qr}  is  a tuple
%containing the two output matrices. Many Julia functions that return
%multiple output arguments return them in a tuple, from which they can
%be accessed by indexing.

%#\begin{verbatim}
%#julia> qr(A)
%#(
%#3x2 Array{Float64,2}:
 %-0.406048   0.431218
 %-0.235053  -0.901287
 %-0.883105   0.0416203,

%2x2 Array{Float64,2}:
 %-0.65531  -0.658722
 % 0.0      -0.379163)
%\end{verbatim}

\begin{comment}
Julia also provides an  interface to the QR factorization,
known as a
``factorization object."    The result of the
QR factorization from LAPACK  is more compact and efficient
than the alternative \verb+qr+ command which is also available.
\end{comment}

\begin{comment}
\begin{jinput}
 X=qrfact(rand(4,2))
 \end{jinput}
 \begin{joutput}
QRCompactWY\{Float64\}  (  4x2 Array\{Float64,2\}: \\
\hspace*{.1em}  -1.4584 \hspace*{.5em}    -1.38826   \\
\> 0.414351   \hspace*{.1em} 0.813698  \\
 \> 0.420327  \hspace*{.1em} 0.830722  \\
 \> 0.415317    -0.0241482,2x2 Array\{Float64,2\}: \\
 1.31505    \hspace*{1.5em}   -1.17218 \\
 6.93424e-310   1.18295)
\end{joutput}


In the example above $X$ is not a matrix, it is an efficient QR
Factorization object. We can build the components
upon request.  Here is \verb+R+

\begin{jinput}
X[:R]
\end{jinput}
\begin{joutput}
2x2 Array\{Float64,2\}: \\
 -1.4584  -1.38826  \\
 \hspace*{.02em} 0.0   \, 0.813698
\end{joutput}

One can use the factorization object to find the least squares
solution. The backslash operator for QR factorization objects has been
overloaded in Julia:

\begin{jinput}
b=rand(4) {\sh { \# One dimensional vector} }
\end{jinput}
\begin{joutput}
4-element Array\{Float64,1\}: \\
\> 0.299695 \\
\> 0.976353 \\
\> 0.688573 \\
\>  0.653433
 \end{joutput}

\begin{jinput}
 X\ /  b    {\sh{  \# Least squares solution is efficient due to packed format} }
\end{jinput}
\begin{joutput}
2-element Array \{Float64,1\}:  \\
\>  0.985617   \\
\hspace*{.12em} -0.0529438
 \end{joutput}

\end{comment}


To be sure, experienced computer scientists tend to suspect there is nothing new under the sun.
The C function \verb+qsort()+ takes a \verb+compar+ function.  Nothing really new there.
Python also has custom sorting with a key. MATLAB's sort is more basic.
The real contribution of Julia, as will be fleshed out further in this paper, is that
the design of Julia allows custom sorting to be much faster than other dynamic languages.

\newpage

The next example that we have chosen for the introductory taste of Julia is a quick plot of Brownian motion.
This example uses the matplotlib python package for graphics.


\ja
Pkg.add("PyPlot") \sh{\# Download the PyPlot package} \\
using PyPlot  \sh {\#  load the functionality into Julia} \\
\\
for i=1:5 \\
\,    y=cumsum(randn(500)) \\
\,    plot(y) \\
end \\ \\
 \includegraphics[width=4in]{myfig2.pdf}
\end{jinput}

\vspace{.2in}



The matplotlib package is popular for users coming from Python or MATLAB.
Gadfly  is  another very popular package for plotting.
Gadfly was built by Daniel Jones and was influenced by  the well admired  Grammar of Graphics
(see \cite{gg1} and \cite{gg2}).


The syntax might be less familiar to some users of numerical  computing languages,
but it is not hard to learn.\footnote{See tutorial on \code{http://gadflyjl.org}}
Many Julia users find Gadfly more flexible and
prefer the look of the output.
Gadfly has also added to the building of beautiful interactive graphics through the \code{Interact} package.
See \code{https://github.com/JuliaLang/Interact.jl/issues/36} for examples of Interact.
\newpage

\ja
Pkg.add("Gadfly") \sh{\# Download the Gadfly  package} \\
using Gadfly   \sh {\#  load the functionality into Julia} \\
\begin{verbatim}
n = 500
p = [layer(x=1:n, y=cumsum(randn(n)), color=[i], Geom.line)
    for i in ["First","Second","Third"]]
labels=(Guide.xlabel("Time"),Guide.ylabel("Value"),
        Guide.Title("Brownian Motion Trials"),Guide.colorkey("Trial"))
plot(p...,labels...)
\end{verbatim}

\end{jinput}




 \includegraphics[width=4in]{gadflyplot.pdf}

 \vspace{.2in}

 \newpage
In our last example in this
introductory tour, we illustrates graphics and histogramming: a user performs a simple
 Monte-Carlo experiment: the famous semicircle law from Random Matrix Theory for a random eigenvalue of a symmetric matrix. In Section \ref{sec:easypar} we provide examples
 of parallel monte carlo histograms.


 \ja
 n=100       	    \sh { \# Matrix Size} \\
 t=1000           \sh  {\# Number of Trials} \\
 sym(A)=A+A'   \sh {\# Define a function named} sym \sh{  to symmetrize a matrix} \\

 \sh {\# Eigenvalues of t nxn matrices saved as vector } \\
 z= [[eigvals(sym(randn(n,n))) for i=1:t]...] \\
 z/=sqrt(2n) \\
 \\
 \sh {\# Histogram Plot} \\
  {\# using PyPlot   }
  \vspace{-.05in}
   \begin{verbatim}
plt.figure(figsize=(8,3.5))
x=-2:.01:2;
 \end{verbatim}
  \vspace{-.2in}
plot(x,sqrt(4.-x.\^{}2)/(2$\pi$),"r")
  \vspace{-.05in}
 \begin{verbatim}
plt.hist(z,bins=50,normed=true)
 \end{verbatim}
 \sh {\# Label and save to file}
 \vspace{-.05in}
 \begin{verbatim}
axis([-3,3,0,.4])
 \end{verbatim}
 \vspace{-.2in}
 \verb&xlabel("normalized &$\lambda$\verb&")&
 \vspace{-.06in}
 \begin{verbatim}
ylabel("pdf")
title("Wigner Semi-Circle Law")
plt.savefig("myfig.pdf")
 \end{verbatim}
   \includegraphics[width=4in]{scfig.pdf}
 \end{jinput}

 \newpage



Julia has been in development since 2009; a public release was
announced in February of 2012.  It is an active open source project
with over 250 contributors and is available under the MIT
License~\cite{mitlicense} for open source software. Nurtured at the
Massachusetts Institute of Technology, but with contributors from
around the world, Julia is increasingly seen as a high-performance
alternative to R, Matlab, Octave, Python, and SciLab, or a
high-productivity alternative to C, C++ and Fortran.  It is also
recognized as being better suited to general purpose computing tasks
than traditional numerical  computing systems, allowing it to be used
not only to prototype numerical algorithms, but also to deploy those
algorithms, and even serve results of numerical computations to the
rest of the world.\footnote{Sudoku as a service, by Iain Dunning,
  \url{http://iaindunning.com/2013/sudoku-as-a-service.html}, is a
  wonderful  example where a Sudoku puzzle is solved using the
  optimization capabilities of the JUMP.jl Julia package \cite{jump} and made
  available as a web service.}  Perhaps most significantly, a rapidly
growing ecosystem of high-quality, open source, composable numerical
packages (over 450 in number, see \url{http://pkg.julialang.org} and especially
\url{http://pkg.julialang.org/pulse.html})
written in Julia has emerged, including
libraries for linear algebra, statistics, optimization, data analysis,
machine learning and many other applications of numerical computing.


\subsection{An invaluable tool for numerical integrity}

One popular feature of Julia is that it gives the user the ability to
``kick the tires" of a numerical computation.  We thank Velvel Kahan
for the sage advice\footnote{Personal communication, January 2013, in the Kahan
home, Berkeley, California} concerning the importance of this feature.


The idea is simple: a good engineer tests his or her code for numerical stability.
In Julia this can be done by changing IEEE rounding modes.
There are five modes to choose from, yet most engineers silently only choose the
 \verb+RoundNearest+ mode default available in many numerical computing systems.
 If a difference is detected, one can also run the computation in higher precision.
 Kahan writes

 \begin{quotation}
 Can the effects of roundoff upon a floating-point computation be assessed without submitting it to
a mathematically rigorous and (if feasible at all) time-consuming error-analysis? In general, No.

$\ldots$


 Though far from foolproof, rounding every inexact arithmetic operation (but not
constants) in the same direction for each of two or three directions besides the
default To Nearest is very likely to confirm accidentally exposed hypersensitivity
to roundoff. When feasible, this scheme offers the best {\it Benefit/Cost }ratio.
\cite{kahan:mindless}
\end{quotation}

 As an example, we round a 15x15 Hilbert-like matrix, and take the [1,1] entry of the inverse
 computed in various round off modes.  The radically different answers  dramatically indicates
 the numerical sensitivity to roundoff.
 We even noticed that slight changes to LAPACK give radically different answers.  Very likely
 you will see different numbers when you run this code due to the very high sensitivity to roundoff errors.

\begin{jinput}
h(n)=[1/(i+j+1) for i=1:n,j=1:n]
\end{jinput}
\begin{joutput}
h (generic function with 1 method)
\end{joutput}
\begin{jinput}
 H=h(15); \\
 with\_rounding(Float64,RoundNearest) do \\
    \,    inv(H)[1,1] \\
       end
       \end{jinput}
    \begin{joutput}
154410.55589294434
\end{joutput}
\begin{jinput}
with\_rounding(Float64,RoundUp) do \\
    \,       inv(H)[1,1] \\
       end
       \end{jinput}
       \begin{joutput}
-49499.606132507324
\end{joutput}
\begin{jinput}
 with\_rounding(Float64,RoundDown) do \\
       \,     inv(H)[1,1] \\
       end
       \end{jinput}
       \begin{joutput}
-841819.4371948242
 \end{joutput}

With 300 bits of precision we obtain \\

\begin{jinput}
with\_bigfloat\_precision(300) do \\
          \,  inv(big(H))[1,1]    \\
           end
           \end{jinput}
           \begin{joutput}
-2.09397179250746270128280174214489516162708857703714959763232689047153\\ 50765882491054998376252e+03
\end{joutput}

Note this is the [1,1] entry of the inverse of the rounded Hilbert-like
matrix, not the inverse of the exact Hilbert-like matrix. Also, the results
are senstive to the BLAS and LAPACK, and the results may differ on
different machines with different versions of Julia.

%whose [1,1] entry would be the integer  225.

\subsection{Julia architecture and language design philosophy}

Many popular dynamic languages were not designed with the goal of high
performance.  After all, if you wanted really good performance you
would use a static language, or so the popular wisdom would say.  Only with the increased need in the
day-to-day life of scientific programmers for simultaneous
productivity and performance in a single system has the need for
high-performance dynamic languages become pressing.  Unfortunately,
retrofitting an existing slow dynamic language for high performance is
almost impossible \textit{specifically} in numerical computing
ecosystems.  This is because numerical computing requires
performance-critical numerical libraries, which invariably depend on
the details of the internal implementation of the high-level language,
thereby locking in those internal implementation details.
 For example, you can run Python code much faster than the standard
 CPython implementation using the PyPy just-in-time compiler; but PyPy
 is currently  incompatible with NumPy and the rest of SciPy.
% Ironically, while you \textit{can} retrofit dynamic languages with
% faster implementations for non-numerical computing, \textit{because}
% of the intense performance requirements numerical computing places on
% libraries you cannot significantly improve the speed of a slow
% language without losing all the high-performance libraries that make
% it appealing in the first place. While one can leverage
% high-performance libraries such as BLAS and LAPACK, it is libraries
% that are written natively in the dynamic language itself that are hard
% to extract performance from.



Another important point is that just because a program is available in C or
Fortran, it may not run
efficiently from the high level language or be easy  to ``glue" it in.
For example when Steven Johnson tried to include his C erf function
in Python he
reported that Pauli Virtane
had  to write glue code\footnote{\url{https://github.com/scipy/scipy/commit/ed14bf0}} to vectorize the erf function over the native
structures in Python in order to realize the performance
and Johnson had to write  glue code for Matlab, Octave,
and Scilab. The Julia effort was,
by contrast, effortless.\footnote{Steven Johnson, personal communication. See \url{http://ab-initio.mit.edu/wiki/index.php/Faddeeva_Package})}
As another example, a benchmark of the normal random number generator
written in pure Julia is twice as fast as MATLAB's \verb+randn+ written
in a lower level language.




The best path to a fast, high-level system for scientific and
numerical computing is to make the system fast enough that all of its
libraries can be written in the high-level language in the first
place. The JUMP.jl library~\cite{jump} for mathematical optimization
written by Miles Lubin and Iain Dunning is a great example of the
success of this approach---the entire library is written in Julia and
uses many language features such as metaprogramming, user-defined
parametric types, and multiple dispatch extensively to provide a
seamless interface to describe various kinds of optimization problems
and solve them with any number of commercially or freely available
solvers.

 {\bf The Two Language Problem:}
As long as the developers' language is harder than the users' language, numerical computing will always be hindered

This is an essential part of the design philosophy of Julia: all basic
functionality must be possible to implement in Julia---never force the
programmer to resort to using C or Fortran.
Julia solves the two language problem.
Basic
functionality must be fast: integer arithmetic, for loops, recursion,
floating-point operations, calling C functions, manipulating C-like
structs.  While these are not only important for numerical programs,
without them, you certainly cannot write fast numerical code.
``Vectorization languages'' like Matlab, R, and Python+NumPy hide
their for loops and integer operations, but they are still there, inside the  C
and Fortran, lurking behind the  thin veneer.  Julia removes this
separation entirely, allowing high-level code to ``just write a for
loop'' if that happens to be the best way to solve a problem.

We believe that the Julia programming language fulfills much of the
Fortran dream: automatic translation of formulas into efficient
executable code.  It allows programmers to write clear, high-level,
generic and abstract code that closely resembles mathematical
formulas, as they have grown accustomed to in dynamic systems, yet
produces fast, low-level machine code that has traditionally only been
generated by static languages.



Julia's ability to combine these levels of performance and
productivity in a single language stems from the choice of a number of
features that work well with each other:
\begin{enumerate}
  \item An expressive parametric type system, allowing optional type
  annotations;  (See Section \ref{sec:firstclass} for parametric types and Section \ref{sec:dispatch} for optional type annotations.)
  \item Multiple dispatch using those types to select implementations (Section \ref{sec:select});
  \item A dynamic dataflow type inference algorithm allowing types of
    most expressions to be inferred (Section \ref{sec:inference});
  \item Careful design of the language and standard library to be
    amenable to type analysis (Section \ref{sec:lang});
  \item Aggressive code specialization against run-time types (Section \ref{sec:reuse});
  \item Metaprogramming for sophisticated code generation (Section \ref{sec:reuse});
  \item Just-In-Time (JIT) compilation using the Low-level Virtual
    Machine (LLVM) compiler framework~\cite{LLVM}, which is also used
    by a number of other compilers such as Clang~\cite{clang} and
    Apple's Swift~\cite{swift} (Section \ref{sec:reuse}).
\end{enumerate}
Although a sophisticated type system is made available to the
programmer, it remains unobtrusive in the sense that one is never
required to specify types, nor are type annotations necessary for
performance.  Type information flows naturally from having actual
values (and hence types) during code generation, and from the multiple
dispatch paradigm: by expressing polymorphic behavior of functions
declaratively with multiple dispatch, the programmer provides the
compiler with extensive type information, while also enjoying
increased expressiveness.

In what follows, we describe the benefits of Julia's language design
for numerical computing, allowing programmers to more readily express
themselves and obtain performance at the same time.
% data flow type inference algorithm, as well as the various language
% features and how they amplify the effectiveness of this algorithm
% and allow We also provide examples of how Julia translates, clear,
% high-level, generic code into the kind of fast, native machine code
% that a human would write, with all the layers of abstraction
% removed, reducing the computation to its bare minimum.

%% \item{\bf A social organization for scientists and scientific ideas}\\
%%   Lively high quality discussions on Google Groups, Github, and
%%   StackOverflow discuss the nuances of scientific numerical computing
%%   at an expert level.  In the past, a university or organization was
%%   lucky to have an expert on numerical analysis nearby.  Now there is
%%   an entire community online ready to discuss the nuances of error
%%   analysis, IEEE standards, effective algorithms, and best coding
%%   practices.

%% \end{enumerate}


\section{Writing programs with and without types}
\label{sec:types}

\subsection{The balance  between human and the computer}
\label{sec:humancomputer}

Graydon Hoare, author of the Rust programming
language~\cite{rust}, in an essay on ``Interactive Scientific
Computing''~\cite{hoareessay} defined programming languages
succinctly:

\begin{quote}
  Programming languages are mediating devices, interfaces that try to
  strike a balance between human needs and computer needs. Implicit in
  that is the assumption that human and computer needs are equally
  important, or need mediating.
\end{quote}

Catering to the needs of both the human and the computer is a
repeating theme in this paper. A program consists of data and
operations on data. Data is not just the input file, but everything
that is held ---an array, a list, a graph, a constant---during the
life of the program. The more the computer knows about this data, the
better it is at executing operations on that data. Types are
exactly this metadata. Describing this metadata, the types, takes real
effort for the human. Statically typed languages such as C and Fortran
are at one extreme, where all types must be defined and are statically
checked during the compilation phase. The result is excellent
performance. Dynamically typed languages dispense with type
definitions, which leads to greater productivity, but lower
performance as the compiler and the runtime cannot benefit from the
type information that is essential to produce fast code. Can we strike
a balance between the human's preference to avoid types and the
computer's need to know?

\subsection{Julia's recognizable types}

Julia's design allows for the gradual learning of concepts, where users
start in a manner that is familiar to them and over time, learn to
structure programs in the ``Julian way'' (a term that captures
well-structured readable high performance Julia code). Julia users
coming from other numerical computing environments have a notion that
data may be represented as matrices that may be dense, sparse,
symmetric, triangular, or of some other kind. They may also, though
not always, know that elements in these data structures may be single
precision floating point numbers, double precision, or integers of a
specific width. In more general cases, the elements within data
structures may be other data structures. We introduce Julia's type
system using matrices and their number types:

\begin{jinput}
rand(1,2,1)
\end{jinput}
\begin{joutput}
1x2x1 Array\{Float64,3\}: \\
{}[ :, :,  1{}] = \\
\>  0.789166  0.652002
 \end{joutput}

%  \begin{jinput}
%   A=eye(3)
%   \end{jinput}
%   \begin{joutput}
% 3x3 Array\{Float64,2\}: \\
% \> 1.0  0.0  0.0 \\
% \> 0.0  1.0  0.0 \\
% \> 0.0  0.0  1.0
%  \end{joutput}

\begin{jinput}
{}[1 2; 3 4{}]
\end{jinput}
\begin{joutput}
2x2 Array\{Int64,2\}: \\
\> 1  2 \\
\>  3  4
 \end{joutput}

\begin{jinput}
{}[true; false{}]
\end{jinput}
\begin{joutput}
2-element Array\{Bool,1\}: \\
\>  true \\
\>  false
 \end{joutput}

 We see a pattern in the examples above. \noindent
 \verb+Array{T,ndims}+ is the general form of the type of a dense
 array with \verb+ndims+ dimensions, whose elements themselves have
a specific type \verb+T+, which is of type double precision floating
point in the first example, a 64-bit signed integer in the second, and
a boolean in the third example.
%Array size is not part of the type, even though it is displayed with the array.
Therefore \verb+Array{T,1}+ is a 1-d vector (first class
objects in Julia) with element type \verb+T+ and \verb+Array{T,2}+ is
the type for 2-d matrices.

It is useful to think of arrays as a
generic N-d object that may contain elements of any type
\verb+T+. Thus \verb+T+ is a type parameter for an array that can take
on many different values. Similarly, the dimensionality of the array
\verb+ndims+ is also a parameter for the array type. This generality
makes it possible to create arrays of arrays. For example, Using
Julia's array comprehension syntax, we create a 2-element vector
containing $2\times2$ identity matrices.

\begin{jinput}
 a = {}[eye(2) for i=1:2{}]
\end{jinput}
\begin{joutput}
2-element Array\{Array\{Float64,2\},1\}:
\end{joutput}
\noindent

Many first time users are already familiar with basic floating point
number types such as \verb+Float32+ (single precision) and
\verb+Float64+ (double precision). In addition, Julia also provides a
built-in \verb+BigFloat+ type that may interest a new Julia user.
With \verb+BigFloat+s, users can run computations in arbitrary
precision arithmetic, thereby exploring the numerical properties of
their computation.

\subsection{User's own types are first class too}
\label{sec:firstclass}

%We would be remiss if we did not illustrate how to define a new type
%in Julia.

Many dynamic languages for numerical computing have traditionally had
an asymmetry, where built-in types have much higher
performance than any user-defined types. This is not the case with
Julia, where there is no meaningful distinction between user-defined
and ``built-in'' types.





We have mentioned so far a few number types and two matrix types,
\verb+Array{T,2}+ the dense array, with element type \verb+T+ and
\verb+SymTridiagonal{T}+, the symmetric tridiagonal with element type \verb+T+. There
are also other matrix types, for other structures including
SparseMatrixCSC (Compressed Sparse Columns), Hermitian, Triangular, Bidiagonal, and Diagonal. Julia's sparse matrix type
has an added flexibility that it can go beyond storing just numbers as
nonzeros, and instead store any other Julia type as well. The indices
in SparseMatrixCSC can also be represented as integers of any width
(16-bit, 32-bit or 64-bit). All these different matrix types, although
available as built-in types to a user downloading Julia, are
implemented completely in Julia, and are in no way any more or less
special than any other types one may define in their own program.

For demonstration, we create a symmetric arrow matrix type that
contains a diagonal and the first row \verb+A[1,2:n]+.


\ja
\sh{ \# Parametric Type Example (Parameter T)}  \\
\sh{\# Define a Symmetric Arrow Matrix Type with elements of type T} \\

type  SymArrow\{T\}   \\
 \,          dv::Vector\{T\}                     \sh{   \# diagonal  }\\
 \,          ev::Vector\{T\}                     \sh{   \# 1st row{}[2:n{}] }\\
    end \\ \\
    \sh{ \# Create your first Symmetric Arrow Matrix}  \\
S = SymArrow([1,2,3,4,5],[6,7,8,9])
\jb
\begin{verbatim}
SymArrow{Int64}([1,2,3,4,5],[6,7,8,9])
\end{verbatim}
\jc

%We will flesh out this Symmetric Arrow Matrix example further in Section \ref{sec:xdispatch}.
Our purpose here is to
 introduces ``Parametric Types''.
 Later in Section \ref{sec:arrow}, we develop this example much further.
 The \verb+SymArrow+
matrix type contains two vectors, one each for the diagonal and the
first row, and these vector contain elements of type \verb+T+. In the type definition,
the type \verb+SymArrow+ is parametrized by the type of the
storage element \verb+T+. By doing so, we have created a generic type,
which refers to a universe of all arrow matrices containing elements
of all types. The matrix \verb+S+, is an example where \verb+T+ is \verb+Int64+.
When we write functions in Section~\ref{sec:arrow}
that operate on arrow matrices, those functions themselves will be
generic and applicable to the entire universe of arrow matrices we
have defined here.



Julia's type system allows for abstract types, concrete ``bits''
types, composite types, and immutable composite types. All of these
can be parametric and users may even write programs using unions of
these different types. We refer the reader to read all about Julia's
type system in the types chapter in the Julia manual\footnote{See the chapter
  on types in the Julia manual:
  \url{http://docs.julialang.org/en/latest/manual/types/}}.



\subsection{Vectorization: Key Strengths and Serious Weaknesses}

Users of traditional  high level computing languages know that vectorization
improves performance.  Do most users know exactly why vectorization
is so useful?  It is precisely because, by vectorizing, the user has promised
the computer that the type of an entire vector of data matches the very first element.
This is an example where users are willing to provide type information to the
computer without even knowing exactly that is what they are doing.
Hence, it is an example of a  strategy that balances the computer's needs with the human's.

From the computer's viewpoint, vectorization means that
operations on data happen largely in sections of the code where types
are known to the runtime system.  When the runtime is operating on
arrays, it has no idea about the data contained in an array until it
encounters the array. Once encountered, the type of the data within
the array is known, and this knowledge is used to execute an
appropriate high performance kernel. Of course what really occurs at
runtime is that the system figures out the type, and gets to reuse
that information through the length of the array.  As long as the
array is not too small, all the extra work in gathering type
information and acting upon it at runtime is amortized over the entire
operation.

The downside of this approach is that the user can achieve
high performance only with built-in types, and user defined types end
up being dramatically slower. The restructuring for vectorization is
often unnatural, and at times not possible. We illustrate this with an
example of the cumulative sum computation.


%\todo[inline]{This really is a good example, perhaps we can say more}
\begin{jinput}
\sh {\# Sum prefix  (cumsum) on vector w  with elements of type T} \\
function prefix\{T\}(w::Vector\{T\})\\
\,    for i=2:size(w,1)\\
\,\,         w[i]+=w[i-1]\\
\,    end\\
\,    w\\
end
\end{jinput}

We execute this code on a vector of double precision numbers and
double-precision complex numbers and observe something
that may seem remarkable: similar running times.


\ja
x = ones(1\_000\_000)\\
@time prefix(x) \\ \\
y = ones(1\_000\_000) + im*ones(1\_000\_000)\\
@time prefix(y);
\jb
elapsed time: 0.003243692 seconds (80 bytes allocated) \\
elapsed time: 0.003290693 seconds (80 bytes allocated)
\jc


\noindent

This simple example  is difficult to vectorize, and hence  is often
built into many numerical computing systems. In Julia, the
implementation is very similar to the snippet of code above, and runs
at speeds similar to C. While Julia users can write vectorized
programs like in any other dynamic language, vectorization is not a
pre-requisite for performance. This is because Julia strikes a
different balance between the human and the computer when it comes to
specifying types. Julia allows optional type annotations.

Generally, in Julia, type annotations are not for performance.  They are purely for
code selection. (See Section \ref{sec:select}.)
If the
programmer annotates their program with types, the Julia compiler will
use that information. But for the most part,
user code often includes minimal or no type annotations, and the Julia
compiler automatically infers the types.





% If the user's experience set consists largely of dynamically typed
% languages,
% (such as MATLAB, R, Mathematica, Octave, SciPy, and SciLab )
%  then specifying types occassionally in user code may be
% seen as a nuisance. First time Julia users usually do not specify
% types in their programs until they get comfortable with the
% language. Types must be specified in Julia in only two circumstances
% \begin{itemize}
% \item For code selection, where a Julia method may have different type
%   signatures for different behaviours. We discuss this in detail in
%   Section~\ref{codeselection}; and
% \item In a tight spot where the compiler is unable to infer the types
%   and falls back to a pessimistic implementation. The pessimistic and
%   by definition, the slowest type in Julia is \verb+Any+, where the
%   compiler knows nothing about the data it is operating on until it
%   encounters it and inspects it.
% \end{itemize}

\subsection{Type inference  rescues ``for loops" and so much  more}
\label{sec:inference}

A key component of Julia's ability to combine performance with
productivity in a single language is its implementation of dynamic
dataflow type
inference~\cite{graphfree},\cite{kaplanullman},\cite{Bezanson:2012jf}.
Unlike type inference algorithms for static languages, this algorithm
is tailored to the way dynamic languages work: the typing of code is
determined by the flow of data through it.  The algorithm works by
walking through a program, starting with the types of its input
values, and ``abstractly interpreting'' it: instead of applying the
code to values, it applies the code to types, following all branches
concurrently and tracking all possible states the program could be in,
including all the types each expression could assume.

Since programs can iterate arbitrarily long and recur to arbitrary
depths, this process must be iterated until it reaches a fixed
point. The design of the type lattice used ensures that the process
terminates.  Once that point is reached, the program is annotated with
upper bounds on the types that each variable and expression can
assume.

The dynamic dataflow type inference algorithm allows programs to be
automatically annotated with type bounds without forcing the
programmer to explicitly specify types.  Yet, in dynamic languages it
is possible to write programs which inherently cannot be concretely
typed.  In such cases, dataflow type inference provides what bounds it
can, but these may be trivial and useless---i.e. they may not narrow
down the set of possible types for an expression at all.  However, the
design of Julia's programming model and standard library are such that
a majority of expressions in typical programs \textit{can} be
concretely typed.  Moreover, there is a positive correlation between
the ability to concretely type code and that code being
performance-critical.

This type system supports productive interactions between the
programmer and compiler.  For example, in mathematical environments
users often want a  square root function defined for reals  that returns either a real or
complex number depending on the sign of the argument. In Julia this
can be defined as

\ja
\sh  {\# Define xsqrt for real inputs} \\
\sh  {\# return sqrt(complex(x)) if x<0} \\
xsqrt(x::Real) = x < 0 ? sqrt(complex(x)) : sqrt(x)
\end{jinput}

\vspace{.1in}

This definition works, but will not perform as well as a real-only
\verb+sqrt+ due to the overhead of tracking whether the result is real
or complex. However, in a particular use the programmer might know
that a real result is expected, and annotate the call as
\verb+xsqrt(y)::Real+.   If at runtime, the result is not real, an error is generated.

 \verb+Real+ is an abstract type, so this code
is still generic and works for any real numeric type.  (For example, a real
integer or a real float or a real rational.)  The compiler will combine this declaration
with the inferred type of \verb+xsqrt(y)+, recovering exact type
information as a result.

%\todo[inline]{The above has three points.  Is this a lot to take in?
%1. The definition of xsqrt. 2. The specialization when a user knows it's real and
%3. abstract types}

%\subsection{Fast \code{for} loops}

A lesson of the numerical computing languages is that one must learn to
vectorize to get performance.  The mantra is ``for loops" are bad,
vectorization is good.  Indeed one can find the mantra on p.72 of the
``1998 Getting Started with Matlab manual'' (and other editions):

\begin{quotation}
Experienced Matlab  users like to say ``Life is too short to spend writing for loops."
\end{quotation}

It is not that ``for loops'' are inherently slow by themselves. The
slowness comes from the fact that in the case of most dynamic
languages, the system does not have access to the types of the various
variables within a loop. Since programs often spend much of their time
doing repeated computations, the slowness of a particular operation
due to lack of type information is magnified inside a loop. This leads
to users often talking about ``slow for loops'' or ``loop overhead''.

Consider the example of the cumulative sum earlier. The function
definition tells the compiler that the input will be a vector,
containing elements of the same type. Within the loop, the compiler is
able to infer that the input vector is indexed with scalar integers
and the partial sums are stored in the same array as the computation
progresses. The sums are computed using the ``+'' operator. The
correct version of  \verb-+-  is chosen by the system from the 123 \verb-+-
methods that are available in Julia at the time of this writing, which define
how various types
may be added --- combinations of integers, floating point numbers,
rational numbers, dense arrays, sparse matrices, etc.

\ja  + \jb
+ (generic function with 123 methods)
\jc

In statically typed languages, full type information
is always available at compile time, allowing
compilation of a loop into a few machine instructions. This is
not the case in most dynamic languages, where the types are discovered
at run time, and the cost of determining the types and selecting the
right operation can run into hundreds or thousands of
instructions.

Julia has a transparent performance model. For example a
\verb+Vector{Float64}+ as in our example here, always has the same
in-memory representation as it would in C or Fortran; one can take a
pointer to the first array element and pass it to a C library function
using \verb+ccall+ and it will just work. The programmer knows exactly how
the data is represented and can reason about it. They know that a
\verb+Vector{Float64}+ does not require any additional heap
allocation besides the \verb+Float64+ values and that
arithmetic operations on these values will be machine arithmetic
operations. In the case of say, \verb+Complex128+, Julia stores
complex numbers in the same way as C or Fortran. Thus complex arrays
are actually arrays of complex values, where the real and imaginary
values are stored consecutively. Some systems have taken the path of
storing the real and imaginary parts separately, which leads to some
convenience for the user, at the cost of performance and
interoperability. With the \verb+immutable+ keyword, a programmer can
also define immutable data types, and enjoy the same benefits of
performance for composite types as for the more primitive number types
(bits types). This approach is being used to define many interesting
data structures such as small arrays of fixed sizes, which can have
much higher performance than the more general array data structure.

The transparency of the C data and performance models has been one of
the major reasons for C's long-lived success. One of the design goals
of Julia is to have similarly transparent data and performance
models. With a sophisticated type system and type inference, Julia
achieves both.

\section{Code Selection}
\label{sec:select}


Code selection or code specialization from one point of view is the
opposite of code reuse enabled by abstraction.  Ironically, viewed
another way, it enables abstraction.  Julia allows users to overload
function names, and select code based on argument types.  This can
happen at the highest and lowest levels of the software stack.  Code
specialization lets us optimize for the details of the case at hand.
Code abstraction lets calling codes, probably those not yet even
written or perhaps not even imagined, work all the way through on
structures that may not have been envisioned by the original
programmer.

We see this as the ultimate realization of the famous 1908 quip that
\begin{quote}
Mathematics is the art of giving the same name to different things.
\end{quote}
by noted mathematician Henri Poincar\'{e}.\footnote{ A few versions of
  this quote are very relevant to Julia's power of abstractions and
  numerical computing. They are worth pondering:
\begin{quote}
 It is the harmony of the different parts, their symmetry, and their
 happy adjustment; it is, in a word, all that introduces order, all
 that gives them unity, that enables us to obtain a clear
 comprehension of the whole as well as of the parts. Elegance may
 result from the feeling of surprise caused by the unlooked-for
 occurrence of objects not habitually associated. In this, again, it
 is fruitful, since it discloses thus relations that were until then
 unrecognized. {\bf Mathematics is the art of giving the same names to
   different things.}
\end{quote}
 http://www.nieuwarchief.nl/serie5/pdf/naw5-2012-13-3-154.pdf.
 and
 \begin{quote}
 One example has just shown us the importance of terms in mathematics;
 but I could quote many others. It is hardly possible to believe what
 economy of thought, as Mach used to say, can be effected by a
 well-chosen term. I think I have already said somewhere that {\bf
   mathematics is the art of giving the same name to different
   things}. It is enough that these things, though differing in
 matter, should be similar in form, to permit of their being, so to
 speak, run in the same mould. When language has been well chosen, one
 is astonished to find that all demonstrations made for a known object
 apply immediately to many new objects: nothing requires to be
 changed, not even the terms, since the names have become the same.
 \end{quote}
 {\tt
   http://www-history.mcs.st-andrews.ac.uk/Extras/Poincare\_Future.html
 }

}



In this upcoming section we provide examples of how plus can apply to so many objects. Some examples are floating point numbers, or integers.
It can also apply to sparse and dense matrices.  Another example is the use of the same name, ``det", for determinant, for the very different algorithms that apply to very different matrix structures.  The use of overloading not only for single argument functions, but for
multiple argument functions is already a powerful abstraction.




\subsection{Dispatch by Argument Type}
\label{sec:dispatch}
In Julia one can define a function of two arguments without argument types:

\ja

\begin{verbatim}
g(x,y)=sqrt(x^2+y^2)\end{verbatim}
\vspace{-0.08in}
 \sh {\# x,y of any type}
\end{jinput}

\noindent
or with argument types.  (This is known as optional type annotation.):

\jav
f(x::Real,   y::Real)    = sqrt(x^2+y^2)
f(x::Complex,y::Complex) = sqrt(abs(x)^2-abs(y)^2)
\end{verbatim}
\end{jinput}


The function $g(x,y)$ will compute $\sqrt{x^2+y^2}$ no matter what the type of $x$ or $y$.
\noindent
By contrast, the function $f(x,y)$ amounts to the separation of cases:

\begin{verbatim}
if x is Real and y is Real compute sqrt(x^2+y^2}
If x is Complex and y is Complex compute sqrt(abs(x)^2-abs(y)^2)
\end{verbatim}
If $x$ and $y$ are not both real  or not both complex, then \verb+f(x,y)+ is an error.



The notation


\ja
\sh{\# Function definitions  for the four possibilities of  x,y having Type1/Type2 }
f(x::Type1, y::Type1)\ =  \ {\sh {\it code  specialized for  both arguments Type1 } } \\
f(x::Type1, y::Type2)\ =  \ {\sh {\it code  specialized for arguments Type1, Type2 respectively} } \\
f(x::Type2, y::Type1)\ =  \ {\sh {\it code  specialized for arguments Type2, Type1 respectively } } \\
f(x::Type2, y::Type2)\ =  \ {\sh {\it code  specialized for both arguments Type2} }
\end{jinput}
\noindent
is a convenient way of expressing code selection (known as dispatch) based on the types of the first two arguments.
Other numerical computing libraries might use Case statements, or If/ElseIf constructs.  Julia avoids the time that can be wasted
during this code selection process  which degrades performance.

In Julia, the dispatch by type goes straight to the compiler leading to remarkable efficiencies.  These efficiencies
can be measured by  small numbers of lines of assembler or by  short execution times.

We have not seen this in the literature but it seems worthwhile to point out four possibilities:

\begin{enumerate}
\item static single dispatch (not done)
\item static multiple dispatch (frequent in static languages, e.g. C++ overloading)
\item dynamic single dispatch  (MATLAB's object oriented system might fall in this category though it has its own special characteristics)
\item dynamic multiple dispatch (usually just called multiple dispatch).
\end{enumerate}

In Section \ref{sec:traditional} we discuss the comparison with
traditional object oriented approaches. Class-based object oriented
programming could reasonably be called dynamic single dispatch, and
overloading could reasonably be called static multiple dispatch.
Julia's (dynamic!) multiple dispatch
approach is more flexible and adaptable while still retaining
powerful performance capabilities. Julia programmers often find that
dynamic multiple dispatch makes it easier to structure their programs
in ways that are closer to the underlying science.

\subsection{Code selection from bits to matrices}

Julia uses the same mechanism for code selection at all levels, from
the top to the bottom.  In Sections \ref{sec:bits},   \ref{sec:summing}, and \ref{sec:further} we
consider code selection problems that have the common general format
defined in In[24] above.

\begin{center}
\begin{tabular}{|c|c|c|} \hline
f &  Function &  Operand  Types \\\hline
Low Level ``+" &  Add Numbers&    \{Float , Int\}  \\
High Level ``+"  &  Add  Matrices &  \{Dense Matrix ,  Sparse Matrix\}  \\
 `` * " & Scale or Compose &  \{Function , Number \} \\ \hline
\end{tabular}
\end{center}

\subsubsection{Summing Numbers: Floats and Ints}
\label{sec:bits}


We begin at the lowest level.   Mathematically, integers are thought of as being special real numbers, but
on a computer,  an Int and a Float have two very different representations.
Ignoring for a moment that there are even many choices of Int and Float representations,
if we add two numbers,
 code selection based on numerical representation is taking place at a very low level.
 Most users are blissfully unaware of this code selection,  because it is hidden somewhere that is usually
off-limits to the user.
Nonetheless, one can follow the evolution of the high level code all the way down to the assembler level which ultimately would
reveal an ADD instruction for integer addition, and, for example,  the AVX\footnote{AVX: \color{red}A\color{black}danced \color{red}V\color{black}ector e\color{red}X\color{black}tension to the x86 instruction set} instruction  VADDSD\footnote{VADDSD: \color{red}{V}\color{black}ector \color{red}ADD S\color{black}calar \color{red}D\color{black}ouble-precision}  for floating point addition in the language
of x86 assembly level instructions.  The point being these are ultimately two different algorithms being called, one for a pair of Ints and one for a pair of Floats.


Figure \ref{fig:nativeadd} takes a close look at what a computer
 must do to perform  \verb-x+y- depending on whether (x,y) is (Int,Int), (Float,Float), or  (Int,Float) respectively.
In the first case, an integer add is called, while in the second case a float add is called.  In the last case,  a promotion of the int to float is called through the x86 instruction VCVTSI2SD\footnote{VCVTSI2SD: \color{red}{V}\color{black}ector
\color{red}{C}\color{black}{on}\color{red}{V}\color{black}{er}\color{red}{T}\color{black}{ Doubleword} (\color{red}S\color{black}calar) \color{red}{I}\color{black}{nteger to}\color{red}{(2) S}\color{black}{calar} \color{red}{D}\color{black}ouble Precision Floating-Point Value}, and then the float add follows.



It is instructive to build a  Julia simulator in Julia itself.
Let us define the aforementioned assembler instructions using Julia.


\ja
\sh {\# Simulate the assembly level add, vaddsd, and vcvtsi2sd commands}
\begin{verbatim}
add(x::Int       ,y::Int)     = x+y
vaddsd(x::Float64,y::Float64) = x+y
vcvtsi2sd(x::Int)             = float(x)
\end{verbatim}
\end{jinput}

\ja
\sh{\# Simulate Julia's definition of + using $\oplus$} \\
\sh{\# To type $\oplus$, type as in TeX,  \textbackslash oplus  and hit the  <tab> key} \\
$\oplus$\verb+(x::Int,    y::Int)     = add(x,y)+ \\
$\oplus$(x::Float64,y::Float64) = vaddsd(x,y) \\
$\oplus$\verb+(x::Int,    y::Float64) = vaddsd(vcvtsi2sd(x),y)+ \\
$\oplus$\verb+(x::Float64,y::Int)     = y+ $\oplus$  x\\
\end{jinput}


\ja
methods($\oplus$)
\jb
4 methods for generic function $\oplus$:\\
$\oplus$ (x::Int64,y::Int64) at In[26]:3 \\
$\oplus$ (x::Float64,y::Float64) at In[26]:4 \\
$\oplus$ (x::Int64,y::Float64) at In[26]:5\\
$\oplus$ (x::Float64,y::Int64) at In[26]:6
\jc

\begin{figure}
\caption{\label{fig:nativeadd} While assembly code may seem intimidating, Julia disassembles readily.  Armed with the {\tt code\_native}  command in Julia and perhaps a good list of
assembler commands such as  may be found on {\tt http://docs.oracle.com/cd/E36784\_01/pdf/E36859.pdf}
or
{\tt http://en.wikipedia.org/wiki/X86\_instruction\_listings}
 one can really learn to
see the details of
code selection in action at the lowest levels.}



\hspace*{.14in}
\ja
f(a,b) = a + b
\jb
f (generic function with 1 method)
\jc



\hspace*{.14in}
\ja
\sh{\# Ints add with the x86  \underline{add}  instruction}
\vspace{-.06in}
\begin{verbatim}
@code_native f(2,3) \end{verbatim}
\jb
 push	RBP \\
 mov	RBP, RSP \\
 \sh{add}	RDI, RSI \\
 mov	RAX, RDI \\
 pop	RBP \\
 ret
\jc


\hspace*{.14in}
\ja
\sh{\# Floats add, for example,  with the x86  \underline{vaddsd}  instruction}
\vspace{-.06in}
\begin{verbatim}
@code_native f(1.0,3.0)
\end{verbatim}
\jb
 push	RBP \\
 mov	RBP, RSP\\
 \sh{vaddsd} XMM0, XMM0, XMM1\\
 pop	RBP\\
 ret
 \jc



\hspace*{.14in}
\ja
\sh{\# Int + Float requires a convert to scalar double precision, hence \\  \#  the  x86 \underline{vcvtsi2sd} instruction}
\vspace{-.06in}
\begin{verbatim}
@code_native f(1.0,3)
\end{verbatim}
\jb

 push	RBP  \\
 mov	RBP, RSP  \\
 \sh{vcvtsi2sd}	XMM1, XMM0, RDI \\
 \sh{vaddsd}	XMM0, XMM1, XMM0 \\
 pop	RBP \\
 ret

 \jc






\end{figure}

\subsubsection{Summing Matrices: Dense and Sparse}
\label{sec:summing}
We now move to a much higher level: matrix addition.
The versatile ``+" symbol lets us add matrices.
Mathematically, sparse matrices are thought of as being special matrices
with enough zero entries.
On a computer, dense matrices are (usually)  contiguous blocks of data with a few parameters attached,
while sparse matrices (which may be stored in many ways) require storage of index information one way or another.
If we add two matrices, code selection must take place depending on whether the summands are (dense,dense),
(dense,sparse), (sparse,dense) or (sparse,sparse).

While this is at a much higher level, the basic pattern is unmistakably the same as that
of Section \ref{sec:bits}.  While including all the bells and whistles is not appropriate
in this paper, we can show one way of implementing and algorithms that works with
dense matrices if either $A$ or $B$ are dense, but uses a sparse algorithm if both are sparse.

\ja
\sh{\# Dense + Dense} \\
$\oplus$(A::Matrix,              B::Matrix)               =\\
\hspace*{0.3in}  [A[i,j]+B[i,j] for i in 1:size(A,1),j in 1:size(A,2)] \\
\sh{\# Dense + Sparse} \\
$\oplus$(A::Matrix,              B::AbstractSparseMatrix) = A $\oplus$ full(B) \\
\sh{\#  Sparse + Dense} \\
$\oplus$(A::AbstractSparseMatrix,B::Matrix)          \     = B $\oplus$ A  \\
\sh{\# Sparse + Sparse is best written using the long form function definition:}
function $\oplus$(A::AbstractSparseMatrix, B::AbstractSparseMatrix)
\vspace{-0.06in}
\begin{verbatim}
    C=copy(A)
    (i,j)=findn(B)
    for k=1:length(i)
        C[i[k],j[k]]+=B[i[k],j[k]]
    end
    C
end
\end{verbatim}
\end{jinput}



We now have eight methods for $\oplus$, four for the low level sum, and four more for
the high level sum.  The most important point is that the mechanism for code selection,
this dispatch notation which goes straight to the compiler, is the same.

\ja
methods($\oplus$)
\jb
8 methods for generic function $\oplus$: \\
\sh{\it a listing of all eight follows: four low level, four matrix level}
\jc





\newpage

\subsubsection{Further examples}
\label{sec:further}

We have a few further examples of the framework provided by In[24] at the beginning of this section
of the paper.







%This mechanism allows mathematical abstractions to be defined in
%expected ways. For example, we can model points and vectors such that
%adding a point and a vector gives a point, but adding two vectors
%gives a vector:

%\begin{verbatim}
%type Point
%    x::Float32
%    y::Float32
%end

%type Vector2D
%   x::Float32
%   y::Float32
%nd

%+(p::Point,    v::Vector2D) = Point(p.x + v.x, p.y + v.y)
%(u::Vector2D, v::Vector2D) = Vector2D(u.x + v.x, u.y + v.y)
%\end{verbatim}

%Despite the high level of abstraction, this code runs efficiently.

It is possible to model mathematical notations that are often used
in print, but are difficult to employ in programs.
For example, we can teach the computer some natural ways to multiply
numbers and functions. Suppose
that $a$ and $t$
are scalars, and $f$ and $g$ are functions, and we wish to define
\begin{enumerate}
\item   { \bf Number x Function \  = scale output:}  $a*g$ is the function that takes $x$ to $a*g(x)$
\vspace{-.05in}
\item {\bf Function x Number \ = scale argument :}  $f*t$ is the function that takes $x$ to $f(tx)$ and
\vspace{-.05in}
\item  {\bf Function x Function = composition of functions:} $f*g$ is the function that takes $x$ to $f(g(x))$.
\end{enumerate}

If you are a mathematician who does not program, you would not see the
fuss.  If you thought how you might implement this in your favorite
computer language, you might immediately see the benefit.  In Julia,
multiple dispatch makes all three uses of \verb+*+ easy to express:

\ja
*(a::Number,  g::Function)= x->a*g(x)   \>  \sh{    \# Scale output  }   \\
*(f::Function,t::Number)  = x->f(t*x)     \> \sh{  \# Scale  argument  } \\
*(f::Function,g::Function)= x->f(g(x))     \sh{  \# Function composition}
\end{jinput}

Here, multiplication is dispatched by the type of its first and second
arguments.  It goes the usual way if both are numbers, but there are
three new ways if one, the other, or both are functions.

These definitions exist as part of a larger system of generic definitions,
which can be reused by later definitions.
Consider the case of the mathematician Gauss' preference for
$\sin^2 \phi $ to refer to $\sin(\sin(\phi))$ and not
$\sin(\phi)^2$ (writing ``$\sin^2(\phi)$ is odious to me, even
though Laplace made use of it."(Figure~\ref{fig:gauss}).)
By defining \verb+*(f::Function,g::Function)= x->f(g(x))+,
{\tt (f\^{}2)(x)} automatically computes $f(f(x))$ as Gauss
wanted. This is a consequence of a generic definition that evaluates
\verb+x^2+ as \verb+x*x+ no matter how \verb+x*x+ is defined.

\begin{figure}
  \centering
  \includegraphics[width=3in]{gauss.jpg}
  \caption{\label{fig:gauss}  Gauss quote hanging from the ceiling of the Boston Museum of Science Mathematica Exhibit.
  }
\end{figure}




This paradigm is a natural fit for numerical computing, since so
many important operations involve interactions among multiple
values or entities. Binary arithmetic operators are obvious examples,
but other uses abound. For example,\footnote{From \url{http://assoc.tumblr.com/post/71454527084/cool-things-you-can-do-in-julia}.} code performing collision
detection among different shapes can define




\ja
function collide(me::Circle,  other::Rectangle)  \\
function collide(me::Polygon, other::Circle)  \\
function collide(me::Polygon, other::Rectangle)
\end{jinput}

\vspace{.2in}



When the call \verb+collide(self, other)+ takes place, the appropriate
method is selected based on both arguments. This dispatch is dynamic,
based on run-time types, unlike function overloading in C++. This design
point is important, since it provides full flexibility --- the
expected method is called no matter how much run-time variation there is
in the types of shape arguments, and independent of whether the compiler
can prove anything about those types.
Dynamic multiple dispatch encompasses both object-oriented techniques used in
general-purpose programming, and the behaviors of mathematical objects
compilers do not know how to reason about.

Note this example allows for three shape types: {Circle, Rectangle, Polygon}.
There could thus be nine possible functions total of a function with two arguments and three types for each
argument.



\subsection{Is ``code selection"  just traditional object oriented programming?}
\label{sec:traditional}


It is worth crystallizing some key aspects of the code selection as described above.

\begin{enumerate}
\item The same  name can be used for different functions.  (See Footnote 5). (Method or function overloading.)
\item The collection of functions that might be called by the same name is thought of as an entity itself. (Generic functions.)
\item Code selection at the lowest and highest levels use one and the same mechanism
\item  Which method is called is based entirely on the types of the arguments of the methods. (This is sometimes the emphasis of the term ad-hoc polymorphism.)
\item The  types of the arguments of a method that is being called may or may not be knowable by a static compiler.  Instead,
the method may be chosen dynamically at runtime as the types become known. (Dynamic Dispatch)  (See Figure \ref{fig:java}.)
In particular, one can recover from a loss of type information.
\item The method is not chosen by only one argument (Single Dispatch) but rather by all the arguments
(Multiple Dispatch)
\item Julia is not encumbered by the encapsulation restrictions (class based methods) of most object oriented languages.
%(See Figure \ref{fig:dispatch}.)
 The generic functions play a more important role than the types.  (Some call this ``verb" based
languages as opposed to most object oriented languages being ``noun" based.  In numerical computing, it is the
concept of ``solve $Ax=b$"  that often feels more primary, at the highest level, rather than whether the matrix  $A$ is full, sparse, or structured.)
\end{enumerate}

Readers familiar with Java might think, so what? One can easily create methods based
on the types of the arguments.  An example is provided in Figure \ref{fig:java}.

\begin{figure}

\centering
\hspace*{1in}\begin{minipage}{5in}
\begin{shaded}
\sh{\textbackslash * Polymorphic Java Example. Method defined by types of two arguments.  *\textbackslash} \\
\begin{verbatim}
public class OverloadedAddable {

	   public int    addthem(int i, int f} {
	      return i+f;
	   }

	   public double addthem(int i, double f} {
	      return i+f;
	   }

	   public double addthem(double i, int f} {
	      return i+f;
	   }

	   public double addthem(double i,  double  f} {
	      return i+f;
	   }

}
\end{verbatim}}
\end{shaded}
\end{minipage}  \caption{\label{fig:java}  Advantages of Julia:  It is true that the above Java code is polymorphic based on the types of the two arguments.
However,   in Java
if  the method  {\tt addthem} is called, the types of the arguments must be known at compile time.   This is static dispatch. Java is also
encumbered by encapsulation: in this case {\tt addthem}  is encapsulated inside the {\tt OverloadedAddable}  class.
While this is considered a safety feature in Java culture, it becomes a terrible burden for numerical computing.
}
\end{figure}

However a moment's thought shows that the following dynamic situation in Julia is impossible to express in Java:

\ja
\sh {\# It is possible for a static compiler to know that x,y are Float} \\
x = randbool() ?  1.0 : 2.0 \\
y = randbool() ?  1.0 : 2.0 \\
x+y \\

\sh {\# It is impossible to know until runtime if x,y are Int or Float} \\
x = randbool() ? 1 : 1.0 \\
y = randbool() ? 1 : 1.0 \\
x+y
\end{jinput}

\vspace{0.1in}


In Julia as in mathematics, functions are as important as their arguments.  Perhaps even more so.
We saw in Out[21] that ``+" already has 123 methods attached, while Out[8] and Out[28] are functions
with one method.  In Out[33] we took advantage of Julia's ability to name variables and methods with
unicode to create the generic  $\oplus$ with eight methods.
  We  can create a new function \verb+foo+ and gave it six definitions depending
on the combination of types. In the following example we introduce terms from computer science
language research for the benefit of an audience of numerical computing programmers.

\begin{jinput}
\sh {\# Define a} generic function \sh{with 6 methods. Each method is itself a} \\
\sh{ \# function. In Julia generic functions are far more convenient than the } \\
\sh{ \# multitude of  case statements seen in other languages.
When Julia sees} \\
\sh{\#}  foo, \sh{  it decides which method to use, rather than first seeing and deciding} \\
\sh{\#  based on the type.}
\begin{verbatim}
foo() = "Empty input"
foo(x::Int) = x
foo(S::String) = length(S)
foo(x::Int, S::String) = "An Int and a String"
foo(x::Float64,y::Float64) = sqrt(x^2+y^2)
foo(a::Any,b::String)= "Something more general than an Int and a String"
\end{verbatim}
\sh{\# The function name} foo  \sh{is overloaded. This is an example of} polymorphism. \\
\sh{\# In the jargon of computer languages this is called} ad-hoc polymorphism. \\
\sh{\# The} multiple dynamic dispatch \sh{idea captures the notion that the generic } \\
\sh{\# function is deciphered dynamically at runtime.  One of the six choices} \\
\sh{\# will be made or an error will occur.}
\end{jinput}
\begin{joutput}
foo (generic function with 6 methods)
\end{joutput}





Any one instance of \verb+foo+ is known as a method or function.  The collection of six methods is referred to as a
{\bf generic function}.
Contemplating the Poincar\'{e} quote in Footnote 5, it is handy to reason about everything that you are giving the same name.
In real life coding, one tends to use the same name when the abstraction makes a great deal of sense.  Humans often lose sight it is an
abstraction.  That we use "+" for ints,floats, dense matrices, and sparse matrices is the same name for different things.
Methods are grouped into
(generic) functions. A generic function can be applied to several
arguments, and the method with the most specific signature matching
the arguments is invoked.

Readers familiar with MATLAB may be familiar with MATLAB's single dispatch mechanism.  It is unusual in that it is
not completely class based, as the code selection is based on MATLAB's own custom hierarchy.  In MATLAB the leftmost
object has precedence, but user-defined classes have precedence over built-in classes.   MATLAB also has a mechanism
to create a custom hierarchy.

Julia generally shuns the notion of ``built-in" vs.\  ``user-defined" preferring to focus on the method to be performed based
on the combination of types, and obtaining high performance as a byproduct.
A high level library writer, which we do not distinguish from any user,
 has to match the best algorithm for the best input
structure.  A sparse matrix would match to  a sparse routine, a dense matrix to a dense routine.
A low level language designer has to make sure that integers are added with an integer adder, and
floating points are added with a float adder.
Despite the very different levels, the reader might recognize that deep down, these are both examples
of code being selected to match the structure of the problem.


Readers familiar with object-oriented paradigms such as C++ or Java
are most likely familiar with the approach of encapsulating methods
inside classes.  This is very similar to our single dispatch examples,
where we have a \verb+newdet+ method for each type of matrix.
Julia's more general
multiple dispatch mechanism (also known as generic functions, or multi-methods)
is a paradigm where methods are defined on
combinations of data types (classes)  %(Figure~\ref{fig:dispatch}).
Julia has proven that this is remarkably well suited for numerical computing.

    A class based language might express the sum of a sparse matrix with a full matrix  as follows:
  {\tt A\_sparse\_matrix.plus(A\_full\_matrix)}.  Similarly it might express indexing as \newline
    {\tt A\_sparse\_matrix.sub(A\_full\_matrix)} .  If a tridiagonal were added to the system, one
    has to find the method {\tt plus} or {\tt subsref} which is encapsulated in the sparse matrix class,
    modify it and test it. Similarly, one has to modify every full matrix method, etc.
    We believe that class-based methods, which can be taken quite far,  are not sufficiently powerful to express the full gamut of abstractions in scientific computing.
  Further, the burdens of encapsulation create a wall around objects and methods that are counterproductive for
  numerical computing.
    \newline
       \hspace*{.2in}
   The generic function idea captures the notion that a method for a general operation on
  pairs of matrices may exist (e.g. ``+'') but if a more specific operation is possible (e.g. ``+'' on sparse matrices, or ``+'' on a special matrix structure like Bidiagonal), then the more specific operation is used.
  We also mention  indexing as another example,  Why should the indexee take precedence over the index?




%
%\begin{figure}
%  \centering
%  \includegraphics[width=3in]{fig-dispatch-class}
%   \rule{.1mm}{1in}
%  \includegraphics[width=3in]{fig-dispatch-multiple}
%  \caption{\label{fig:dispatch}{Class-based method (single) dispatch (left) vs. multiple dispatch (right).
%  \newline
%  \hspace*{.2in}
%  \small
%    A class based language might express the sum of a sparse matrix with a full matrix  as follows:
%  {\tt A\_sparse\_matrix.plus(A\_full\_matrix)}.  Similarly it might express indexing as
%    {\tt A\_sparse\_matrix.subsref(A\_full\_matrix)} .  If a tridiagonal were added to the system, one
%    has to find the method {\tt plus} or {\tt subsref} which is encapsulated in the sparse matrix class,
%    modify it and test it. Similarly, one has to modify every full matrix method, etc.
%    We believe that class-based methods, which can be taken quite far,  are not sufficiently powerful to express the full gamut of abstractions in scientific computing.
%  Further, the burdens of encapsulation create a wall around objects and methods that are counterproductive for
%  numerical computing.
%    \newline
%       \hspace*{.2in}
%   The generic function idea (right) captures the notion that a method for a general operation on
%  pairs of matrices may exist (e.g. ``+'') but if a more specific operation is possible (e.g. ``+'' on sparse matrices, or ``+'' on a special matrix structure like Bidiagonal), then the more specific operation is used.
%  We also mention  indexing as another example,  Why should the indexee take precedence over the index?
%   }}
%\end{figure}



\subsubsection{Quantifying use of multiple dispatch}


In~\cite{juliaarray} we performed an analysis to substantiate the
claim that multiple dispatch, an esoteric idea for numerical computing from computer
languages, finds its killer application in scientific computing.
We wanted to answer for ourselves the question of whether there
was really anything different about how \julia\ uses multiple
dispatch.

Table \ref{dispatchratios} gives an answer in terms of Dispatch ratio (DR),
Choice ratio (CR). and Degree of specialization (DoS).
While multiple dispatch is an idea that has been circulating for some time,
its application to numerical computing appears to have significantly favorable
characteristics compared to previous applications.








% To quantify how heavily
%he language feature is used,
%we use the following metrics for evaluating the extent of multiple
%dispatch \cite{Muschevici:2008}:

%\begin{enumerate}
%\item Dispatch ratio (DR): The average number of methods in a generic function.
%\item Choice ratio (CR): For each method, the total number of methods over all
%generic functions it belongs to, averaged over all methods. This is essentially
%the sum of the squares of the number of methods in each generic function, divided
%by the total number of methods. The intent of this statistic is to give more weight
%to functions with a large number of methods.
%\item Degree of specialization (DoS): The average number of type-specialized
%arguments per method.
%\end{enumerate}

%Table~\ref{dispatchratios} shows the mean of each metric over the
%entire Julia \code{Base} library, showing a high degree of multiple
%dispatch compared with corpora in other languages
%\cite{Muschevici:2008}.  Compared to most multiple dispatch systems,
%\julia\ functions tend to have a large number of definitions. To see
%why this might be, it helps to compare results from a biased sample of
%common operators. These functions are the most obvious candidates for
%multiple dispatch, and as a result their statistics climb
%dramatically. \julia\ is focused on numerical computing, and so is
%likely to have a large proportion of functions with this character.

\begin{table}
\label{dispatchratios}
\begin{center}
\begin{tabular}{|l|r|r|r|}\hline
Language & DR & CR & DoS \\
\hline \hline
Gwydion    & 1.74 & 18.27 & 2.14 \\
\hline
OpenDylan  & 2.51 & 43.84 & 1.23 \\
\hline
CMUCL      & 2.03 &  6.34 & 1.17 \\
\hline
SBCL       & 2.37 & 26.57 & 1.11 \\
\hline
McCLIM     & 2.32 & 15.43 & 1.17 \\
\hline
Vortex     & 2.33 & 63.30 & 1.06 \\
\hline
Whirlwind  & 2.07 & 31.65 & 0.71 \\
\hline
NiceC      & 1.36 &  3.46 & 0.33 \\
\hline
LocStack   & 1.50 &  8.92 & 1.02 \\
\hline
\julia\      & 5.86 & 51.44 & 1.54 \\
\hline
\julia\ operators & 28.13 & 78.06 & 2.01 \\
\hline
\end{tabular}
\end{center}
\caption{
A comparison of \julia\ (1208 functions exported from the \code{Base} library)
to other languages with multiple dispatch.
The ``\julia\ operators'' row describes 47 functions with special syntax
(binary operators, indexing, and concatenation).
Data for other systems are from \cite{Muschevici:2008}.
The results indicate that \julia\ is using multiple dispatch far more heavily
than previous systems.  }
\end{table}


% Some languages offer various forms of overloading.  MATLAB , for
% example, dispatches on the one dominant argument.  Mathematica has an
% extensive pattern recognition system.  Neither has a multiple dispatch
% system nor a language architected for the full set of conveniences and
% performance offered by the \julia\ approach.

%% \julia\ does not require users to use types or multiple dispatch. Type
%% inference passes in the \julia\ compiler automatically guess the types
%% of most variables, which generates high performance machine code.  As
%% users get used to a more ``Julian'' way of writing programs, they
%% usually start incorporating types and multiple dispatch in a judicious
%% way to express their programs more clearly and to get higher
%% performance.

%% \subsubsection{Analysis of performance}

%% The lesson of the numerical computing languages is one must learn to
%% vectorize to get performance.  The mantra is ``for loops" are bad,
%% vectorization is good.  Indeed one can find the mantra on p.72 of the
%% ``1998 Getting Started with MATLAB manual'' (and other editions):

%% \begin{quotation}
%% Experienced MATLAB  users like to say ``Life is too short to spend writing for loops."
%% \end{quotation}

%% Instead, \julia\ users like to say the opposite:

%% \begin{quotation}
%% ``Life is too short to vectorize programs just for performance''.
%% \end{quotation}

%% So how is it that multiple dispatch and Julia's type system turns the
%% tables on this well established mantra? It is not that ``for loops''
%% are inherently slow by themselves. The slowness comes from the fact
%% that in the case of most dynamic languages, the system does not have
%% access to the types of the various variables within a loop. Consider
%% the following example that adds the elements of two input arrays and
%% returns a result containing the sum:

%% \begin{verbatim}
%% function +{T}(A::Array{T}, B::Array{T})
%%     C = similar(A)
%%     for i=1:length(A)
%%        C[i] = A[i] + B[i]
%%     end
%%     return C
%% end
%% \end{verbatim}

%% The function definition tells the compiler that the inputs will be two
%% arrays containing elements of the same type. Within the loops, the
%% input and result arrays are indexed with scalar indices. This is just
%% one of the 123 ``+'' methods available in \julia\ that define how
%% various types may be added - combinations of integers, floating point
%% numbers, rational numbers, dense arrays, sparse matrices, etc.

%% \begin{verbatim}
%% julia> +
%% + (generic function with 123 methods)
%% \end{verbatim}

%% In statically typed languages, this information is available to the compiler at
%% compile time, which can then compile this entire loop into a few
%% machine instructions. This is not the case in many dynamic languages,
%% where the types of $A$, $B$, $C$, and $i$ are discovered at
%% runtime, and the cost of doing so can run into hundreds or thousands
%% of instructions. With multiple dispatch and type inference, we are
%% able to combine the best of both worlds. Consider the following code
%% fragment now:

%% \begin{verbatim}
%% A = randn(Float32, 1000, 1000)
%% B = randn(Float32, 1000, 1000)
%% C = A + B
%% \end{verbatim}

%% The code looks no different from any other dynamic high-level language
%% that most scientists are comfortable with. When \julia\ executes this
%% code, it knows that $A$ and $B$ are arrays, and that $i$ is an
%% integer. Based on this knowledge, it knows that it must dispatch to
%% the ``+'' method we defined above. Due to the rich type information
%% available, where the type of $C$ is inferred through type inference,
%% the code generation within the ``+'' method is able to generate
%% specialized machine code for fast addition of arrays that contain 32-bit
%% floats.




\subsection{Case Study for Numerical Computing}

For a reader wishing to get started with some of these powerful features in Julia we provide
an introductory example.

\subsubsection{Determinant: Simple Single Dispatch}

In traditional numerical computing there were people with special skills known as
library writers.  Most users were, well, just users of libraries.  In this case study,
we show how anybody can dispatch a new determinant function based solely
on the type of the argument.


For triangular and diagonal structures the obvious formulas are used.
For general matrices, the programmer will compute QR and use the diagonal elements of $R$.\footnote{LU is more efficient. We simply wanted to illustrate other ways are possible.}
For symmetric tridiagonals the usual 3-term recurrence formula is used.
(The first four are defined as one line functions; the symmetric tridiagonal uses the long form.)


\begin{jinput}
\sh{\# Simple determinants defined using the short form for functions} \\
newdet(x::Number)     = x \\
newdet(A::Diagonal )  = prod(diag(A))   \\
newdet(A::Triangular) = prod(diag(A))   \\
newdet(A::Matrix) = -prod(diag(qrfact(full(A))[:R]))*(-1)\^{}size(A,1)  \\

\sh{\# Tridiagonal determinant defined using the long form for functions} \\
function newdet(A::SymTridiagonal)  \\
\, \sh{\# Assign c and d as a pair }\\
    \,  c,d = 1, A[1,1]   \\
    \,  for i=2:size(A,1)    \\
\,  \,     {\sh{ \#  temp=d, d=the expression, c=temp}} \\
     \, \,    c,d = d, d*A[i,i]-c*A[i,i-1]{}\^ {}{2}      \\
       \,   end \\
   \,  d \\
end  \\
\end{jinput}


We have illustrated a mechanism to select a determinant formula at runtime based on the type of
the input argument.  If Julia knows an argument type early, it can make use of this information
for performance.  If it does not, code selection can still happen, at runtime.  The reason why Julia
can still perform well is that once code selection based on type  occurs, Julia can return to performing well
once insider the method.





\subsubsection{A Symmetric Arrow Matrix Type}
\label{sec:arrow}
In the field of Matrix Computations, there are matrix structures and operations on these matrices.  In Julia, these
structures exist as Julia types.
Julia has a number of predefined matrix structure types: (dense) \verb+Matrix+, (compressed sparse column) \verb+SparseMatrixCSC+,
\verb+Symmetric+, \verb+Hermitian+, \verb+SymTridiagonal+,
\verb+Bidiagonal+, \verb+Tridiagonal+. \verb+Diagonal+,
and \verb+Triangular+ are all examples of Julia's matrix structures.

The operations on these matrices exist as Julia's methods.  Familiar examples of operations are indexing, determinant, size,
and matrix addition.  Since matrix addition takes two arguments, it may be necessary to reconcile two different types when
computing the sum.




Some languages do not allow you to extend their built in methods.  This ability is known
as external dispatch.   In the following example, we illustrate how the user can add symmetric
arrow matrices to the system, and then add a specialized \verb+det+ method to compute
the determinant of a symmetric arrow matrix efficiently.



\ja
\sh{\# Define a Symmetric Arrow Matrix Type} \\
immutable SymArrow\{T\}  <: AbstractMatrix\{T\} \\
 \,          dv::Vector\{T\}                     \sh{   \# diagonal  }\\
 \,          ev::Vector\{T\}                     \sh{   \# 1st row{}[2:n{}] }\\
 \>      end
\end{jinput}


\ja
\sh {\# Define its size} \\
importall Base \\
size(A::SymArrow, dim::Integer) = size(A.dv,1) \\
size(A::SymArrow)= size(A,1), size(A,1)
\jb
size (generic function with 52 methods)
\jc





\ja
\sh {\# Index into a SymArrow} \\
 function getindex(A::SymArrow,i::Integer,j::Integer) \\
\,           if     i==j; return A.dv[i]   \\
\, \>         elseif i==1; return A.ev{}[j-1{}]  \\
\,  \>       elseif j==1; return A.ev{}[i-1{}] \\
\,    \>         else         return zero(typeof(A.dv{}[1{}]))  \\
\,           end \\
    end
       \jb
getindex (generic function with 168 methods)
\jc

\ja
\sh {\# Dense version of SymArrow} \\
full(A::SymArrow) =[A[i,j] for i=1:size(A,1), j=1:size(A,2)]
\jb
full (generic function with 17 methods)
\jc

\ja
\sh {\# An example} \\
S=SymArrow({}[1,2,3,4,5{}],{}[6,7,8,9{}])
\jb
5x5 SymArrow\{Int64\}: \\
\sh{ 1  6  7  8  9} \\
\sh{ 6  2}  0  0  0 \\
 \sh{7}  0  \sh{3}  0  0 \\
 \sh{8}  0  0  \sh{4}  0 \\
 \sh{9}  0  0  0  \sh{5}
 \jc


\ja
\sh{\# det for SymArrow  (external dispatch example)} \\
     function exc\_prod(v)  \sh{\# prod(v)/v[i] } \\
\,        [prod(v{}[{}[1:(i-1),(i+1):end{}]{}]) for i=1:size(v,1)]  \\
       end \\
       \sh{\# det for SymArrow  formula} \\
      det(A::SymArrow) = prod(A.dv)-sum(A.ev.\textasciicircum 2.*exc\_prod(A.dv{}[2:end{}]))
       \jb
       det (generic function with 17 methods)
\jc

The above julia code uses the  special formula
$$\det(A)=\prod_{i=1}^n d_i - \sum_{i=2}^n  e_i^2 \prod_{2 \le j \ne i \le n} d_j ,$$
valid for symmetric arrow matrices with diagonal $d$ and first row starting with the second entry $e$.


In Julia terminology \verb+det+  and \verb+newdet+ are {\it functions.}  In some numerical computing languages,
a  function might begin with a lot of argument checking to pick which algorithm to use.  In Julia, one creates
a number of {\it methods.}  Thus \verb+newdet+ on a diagonal is one method for \verb+newdet+, and \verb+newdet+
on a triangular matrix is a second method.  \verb+det+ on a \verb+SymArrow+ is a new method for \verb+det+.
Code is selected, in advance if the compiler knows the type, otherwise the code is selected at run time.  The selection
of code is known as {\it dispatch.}

We have seen a number of examples of code selection for single dispatch.
We can now turn to a very powerful feature,  Julia's multiple dispatch mechanism.
Now that we have created a symmetric arrow matrix, we might want to add it
to all possible matrices of all types.  However, we might notice that
a symmetric arrow plus a diagonal does not require operations on full dense matrices.

The code below starts with the most general case, and then allows for specialization
for the symmetric arrow and diagonal sum:


\ja
\sh{\# SymArrow + Any Matrix: (Fallback: add full dense arrays )} \\
+(A::SymArrow, B::Matrix) = full(A)+B  \\
+(B::Matrix, A::SymArrow) = A+B  \\
\sh{\# SymArrow + Diagonal:  (Special case: add diagonals, copy off-diagonal) }\\
+(A::SymArrow, B::Diagonal) = SymArrow(A.dv+B.diag,A.ev) \\
+(B::Diagonal, A::SymArrow) = A+B
\end{jinput}

\newpage

\section{Code reuse: Code Generation is not only for the compiler}
\label{sec:reuse}


At the crossroads of code selection and code reuse is the linear algebra
software engineer's dilemma.  The complexity of the full solution has been nicely captured in the context of LAPACK
and ScaLAPACK by Demmel and Dongarra, et.al.,
\cite{lawn181}
and reproduced verbatim here:
\begin{verbatim}
(1) for all linear algebra problems
     (linear systems, eigenproblems, ...)
(2)     for all matrix types
            (general, symmetric, banded, ...)
(3)           for all data types
                  (real, complex, single, double, higher precision)
(4)                  for all machine architectures
                       and communication topologies
(5)                        for all programming interfaces
(6)                             provide the best algorithm(s) available in terms of
                                  performance and accuracy (``algorithms" is plural
                                  because sometimes no single one is always best)
\end{verbatim}

Obviously (1) and (6) and perhaps (4)  and even (5) are about code selection.
(2) and (3) are about code reuse at the high level, and code selection at the lowest levels.

In the language of Computer Science, code reuse is about taking advantage of polymorphism.
In the general language of mathematics it's about taking advantage of abstraction, or the sameness of two things.
Either way, programs are efficient, powerful, and maintainable if programmers are given powerful mechanisms
to reuse code.

Reusing code can be very tricky.
Much of the development of modern programming
language theory, both static and dynamic, can be understood by following
the emergence of various forms of polymorphism.
Saying the same notion another way, figuring out how to enable users to reuse code is perhaps far more
difficult than training mathematicians to recognize that two different mathematical objects share
sufficient commonality that they may be given the same name.


\subsection{Example: Non-floating point linear algebra}

Increasingly, the applicability of linear algebra has gone well beyond
the LAPACK world of floating point numbers.  These days linear algebra
is being performed on, say, high precision numbers, integers, elements
of finite fields, or rational numbers.


There will  always be a special place for the BLAS, and the performance it provides
for floating point numbers.  Nonetheless, linear algebra operations like ``inv'' transcend
any one data type.  One can write a general ``inv'' and as long as the necessary
operations are available, the code just works.  That is the power of code reuse.

Here we show an example of rational types just working.  A rational could be replaced
by other constructs, which if they have well defined operations of algebra, will continue to work.


\ja nHilbert = 8
\jb 8
\jc


\ja
H = Rational\{BigInt\}[1//(i+j-1) for i=1:nHilbert, j=1:nHilbert]
\jb
8x8 Array\{Rational\{BigInt\},2\}: \vspace{-.0in}
\begin{verbatim}
 1//1  1//2  1//3   1//4   1//5   1//6   1//7   1//8
 1//2  1//3  1//4   1//5   1//6   1//7   1//8   1//9
 1//3  1//4  1//5   1//6   1//7   1//8   1//9   1//10
 1//4  1//5  1//6   1//7   1//8   1//9   1//10  1//11
 1//5  1//6  1//7   1//8   1//9   1//10  1//11  1//12
 1//6  1//7  1//8   1//9   1//10  1//11  1//12  1//13
 1//7  1//8  1//9   1//10  1//11  1//12  1//13  1//14
 1//8  1//9  1//10  1//11  1//12  1//13  1//14  1//15
 \end{verbatim}
 \jc


\ja inv(H) \jb
8x8 Array\{Rational\{BigInt\},2\}:  \vspace{-.0in}
\begin{verbatim}
      64//1      -2016//1  ... -288288//1       192192//1      -51480//1
   -2016//1      84672//1       15567552//1    -10594584//1     2882880//1
   20160//1    -952560//1      -204324120//1    141261120//1   -38918880//1
  -92400//1    4656960//1       1109908800//1   -776936160//1   216216000//1
  221760//1  -11642400//1      -2996753760//1   2118916800//1  -594594000//1
 -288288//1   15567552//1  ...  4249941696//1  -3030051024//1   856215360//1
  192192//1  -10594584//1      -3030051024//1   2175421248//1  -618377760//1
  -51480//1    2882880//1       856215360//1   -618377760//1    176679360//1
\end{verbatim}
\jc


\subsection{Generic Programming for library development}






Julia blurs the line between library developer and user.
The modern numerical computing world depends heavily on domain libraries
for all areas of applied math, and specialized analyses for different
branches of science. These libraries make end users more productive, but
developing the libraries themselves is a difficult problem. A useful
library needs to be both high performance  and generic, providing enough
flexibility to adapt to the varying problems of different users.

Meeting these requirements places a large burden on library developers.
The need for code that is both generic and performant pushes programming
languages to their limits, requiring use of advanced features such as
C++ templates. Next, the popularity of dynamic languages means that
interfacing work is needed to make these libraries callable from \python,
\rlang, and other systems. Overall, this workflow requires not only
domain knowledge, but a high level of programming skill and knowledge of
language internals. These requirements on time and expertise are unreasonable.

The design of Julia changes this situation. While being ``high level'' and
``fast'' are prerequisites to solving the problem, the true requirements
involve subtleties that are not addressed just by taking any high level
language and speeding it up. To obtain the best performance for
complex library constructs, a compiler must understand code more deeply,
and perform more computations at compile time. This is where a fresh
language design is needed.

A good example is determining the ranks of multi-dimensional arrays.
Numerical computing systems typically have complex rules for how
various functions operate on array dimensions, and knowing the
ranks of arrays at compile time is crucial for performance. Therefore
compilers for ``array languages'' like APL and MATLAB typically
have built-in rules for such functions. However built-in rules do not
address the flexibility needs of library developers. More general
mechanisms like C++ templates are needed to ``teach'' compilers new
rules. Unfortunately, templates are notoriously difficult to use.

We addressed the array-rank-inference problem in a recent paper~\cite{juliaarray}.
It turns out that Julia's combination of multiple dispatch and dataflow
type inference allows array rank rules to be defined in a natural way,
with minimal code. Furthermore, this style of expression allows our
compiler to infer array ranks without specialized array knowledge being
built in. The following example of this technique is taken directly from
Julia's standard library:

\ja
index\_shape(I::Real...) = ()  \\
index\_shape(i, I...) = tuple(length(i), index\_shape(I...)...)
\end{jinput}

The arguments are indices that might be used to index an array, and the output
is the shape.  These two lines define a generic function that computes the shape of the
array $I$ resulting from an indexing operation. It implements the rule that
trailing dimensions indexed with scalars are dropped.

This code may look cryptic to the newcomer, so let's break it down.


\ja
\sh {\# Indexing entirely by scalars returns an empty tuple} \\
index\_shape(I::Real...) = ()
\end{jinput}

The \verb+I::Real+ indicates a real scalar number type.  The \verb+...+ indicates
a sequence of such types, which can be any length including $0$.  Thus calling
\verb+index_shape+ with any sequence of scalars yields an empty array:

\ja
println(  index\_shape() )  \\
println(  index\_shape(1,2,3) )  \\
println(  index\_shape(1,2,3,4) )
\jb
() \\
() \\
()
\jc


The method on the second line is dispatched  if any one argument is not a scalar:

\ja
\sh {\# Peel off the length of the first argument, and dispatch the remaining part} \\
index\_shape(i,I...)=tuple(length(i),index\_shape(I...)...) \\
\end{jinput}


\ja
println( index\_shape(7:9) )  \\
println( index\_shape(5,7:9) )  \\
println( index\_shape(7:9,1,2,3,4) )
\jb
(3,)  \\
(1,3)  \\
(3,)
\jc


 Different rules
(e.g. dropping all dimensions indexed with scalars) can be obtained by
adjusting the definitions slightly. When dataflow type inference is
applied to these definitions for any particular invocation, a tuple
type of the correct length emerges, telling the compiler the array rank.

We emphasize that this can be written in other ways in other numerical computing languages,
but doing it this way gives the information to the compiler.  This is the moral of Julia, that
the programmer can help provide information to the compiler so that code is not slowed
down at run time.






\subsection{Macros}

Julia has a macro system that provides easy custom code generation,
bringing a level of performance that is otherwise difficult to
achieve.
A macro is a function that runs at parse-time, and takes parsed symbolic expressions in and returns transformed symbolic expressions out, which are inserted into the code for later compilation.

For example, a library developer implemented an
\verb+@evalpoly+ macro that uses Horner's rule to evaluate polynomials
efficiently.
Consider

\ja
@evalpoly(10,3,4,5,6)
\end{jinput}
\noindent which returns 6543 (the polynomial $3+4x+5x+6x^2$, evaluated at $10$ with Horner's rule).
Julia allows us to see the inline generated code with the command

\ja
macroexpand(:@evalpoly(10,3,4,5,6))
\end{jinput}

We reproduce the key lines below
\begin{joutput}
\#471\#t = 10  \sh{\# Store 10 into a variable named \#471\#t } \\
Base.Math.+(3,Base.Math.*(\#471\#t,Base.Math.+(4,Base.Math.*
            (\#471\#t,Base.Math.+(5,Base.Math.*(\#471\#t,6))))
           ))
\jc

This code-generating macro only needs to produce the correct symbolic
structure, and Julia's compiler handles the remaining details of fast
native code generation. Since polynomial evaluation is so important
for numerical library software it is critical that users can evaluate
polynomials as fast as possible.
The overhead of implementing an explicit loop, accessing coefficients in an array, and possibly a subroutine call (if it is not inlined), is substantial compared to just inlining the whole polynomial evaluation.


Steven Johnson reports in his EuroSciPy
https://www.euroscipy.org/2014
 notebook \url{https://github.com/stevengj/Julia-EuroSciPy14/blob/master/Metaprogramming.ipynb}
\begin{quotation}

This is precisely how erfinv is implemented in Julia (in single and double precision), and is 3
to 4� faster than the compiled (Fortran?) code in Matlab, and 2 to 3� faster than the compiled (Fortran Cephes) code used in SciPy.

The difference (at least in Cephes) seems to be mainly that they have explicit arrays of polynomial coefficients and call a subroutine for Horner's rule, versus inlining it via a macro.
\end{quotation}









Johnson also used the same trick in his implementation of the
digamma special function for complex argumens \footnote{\url{https://github.com/JuliaLang/julia/issues/7033}} following an idea of Knuth:
\begin{quote}
  As described in Knuth TAOCP vol. 2, sec. 4.6.4, there is actually an
  algorithm even better than Horner's rule for evaluating polynomials
  p(z) at complex arguments (but with real coefficients): you can save
  almost a factor of two for high degrees. It is so complicated that
  it is basically only usable via code generation, so it would be
  especially nice to modify the @horner macro to switch to this for
  complex arguments.
\end{quote}
No sooner than this was proposed, the macro was rewritten to allow for
this case giving a factor of four performance improvement on all real
polynomials evaluated at complex arguments.

\subsection{Just-In-Time compilation and code specialization}




       .

Assuming that users want performance, \julia\ specializes code for run-time types by
default. This technique is fairly effective, often yielding performance within a
factor of 2 of C (Fig.~\ref{fig:performance}).

Through specialization, our compiler can remove overhead for user defined types
with sophisticated behaviors.
The Units package written by Keno Fischer\footnote{\tt
  https://github.com/Keno/SIUnits.jl} is a good example.
With this package installed, it is possible to ask for the sum of two lengths:

\ja
Pkg.add("SIUnits") \sh{\# Needed once to download the package} \\
using SIUnits  \\
1Meter + 2Meter
\jb
3m
\jc

The \code{+} methods defined in the library get compiled to fast
machine code, consisting only of an
\code{add} instruction and stack operations, with the \code{Meter}s
completely optimized away.  It is possible to see the x86 assembler code
any time with the \verb+@code_native+ command.

(As already mentioned in the caption to Figure \ref{fig:nativeadd}, there are many online
listings such as  \\
 \verb+http://docs.oracle.com/cd/E36784_01/pdf/E36859.pdf+
or \\ \verb+http://en.wikipedia.org/wiki/X86_instruction_listings+
for users to learn what all the commands are doing.  Julia users who never
looked at assembler before, can get to understand assembler very quickly
by simply trying out a few simple commands.)

\ja
@code\_native 1Meter + 2Meter
\jb
\>       .section	\_\_TEXT,\_\_text,regular,pure\_instructions   \\
Filename: /Users/viral/.julia/SIUnits/src/SIUnits.jl  \\
Source line: 122  \\
 \>       push	RBP  \\
 \>       mov	RBP, RSP  \\
Source line: 122  \\
 \>       \sh{add}	RDI, RSI  \\
Source line: 123   \\
  \>      mov	RAX, RDI  \\
   \>     pop	RBP   \\
   \>     ret
\jc

 The data is stored exactly as an array of floating-point values with no overhead for storage or computation, because the units are attached to the type of the array rather than to the elements.
The trained eye sees this as minimal code generated for the instruction.  Specifically
there is some register activity and simply one add instruction (highlighted here).


%% \begin{verbatim}
%% julia> code_native(dot, (Vector2D, Vector2D))
%%     .section    __TEXT,__text,regular,pure_instructions
%%     push    RBP         ; Store stack frame pointer
%%     mov RBP, RSP        ; Let framepointer point to stack

%%     ; Move data at address RDI + 8 to
%%     ; floating point registers XMM0 and XMM1
%%     vmovsd  XMM0, QWORD PTR [RDI + 8]
%%     vmovsd  XMM1, QWORD PTR [RDI + 16]

%%     ; Multiply double number in XMM1 floating point
%%     ; register with number at RSI + 16 store result at XMM1
%%     vmulsd  XMM1, XMM1, QWORD PTR [RSI + 16]
%%     vmulsd  XMM0, XMM0, QWORD PTR [RSI + 8]

%%     ; Add results from both multiplication and store at XMM0
%%     vaddsd  XMM0, XMM0, XMM1
%%     pop RBP
%%     ret
%% \end{verbatim}


\begin{figure}
  \centering
  \includegraphics[width=6.5in]{benchmarks.pdf}
  \caption{\label{fig:performance}{Performance comparison of various language performing simple micro-benchmarks. Benchmark execution time relative to C.  (Smaller is better, C performance = 1.0).}}
\end{figure}




\section{Language and standard library design}
\label{sec:lang}

Seemingly innocuous design choices in a language can have profound, pervasive performance implications.
These are often overlooked in languages that were not designed from the beginning to be able to deliver excellent performance.
Other aspects of language and library design affect the usability, composability, and power of the provided functionality.

\subsection{Integer arithmetic}

A simple but crucial example of a performance-critical language design choice is integer arithmetic.
Julia uses machine arithmetic for integer computations.
This means that the range of Int values is bounded and wraps around at either end so that adding, subtracting and multiplying integers can overflow or underflow, leading to results that are familiar to C and Fortran programmers but which can be unsettling to users of systems like Mathematica or Python:

\ja
typemax(Int)
\jb
\hspace*{.1em}  9223372036854775807
\jc

\ja
 ans+1
 \jb
 -9223372036854775808
\jc

\ja
 -ans
 \jb
-9223372036854775808
\jc

\ja 2*ans \jb
0 \jc


In both Mathematica and Python, integers behave like mathematical integers, growing forever larger or smaller, never overflowing.
While this may be an acceptable choice if only end-users will ever write code with the system's integers, it becomes rapidly clear that it is too slow if every loop in every function uses these integers to count its loops and compute indices and offsets.

Since machine multiplication and addition are associative and distribute, Julia is free to aggressively optimize simple little functions like \code{f(k) = 5k-1}.
The machine code for this function is just this:

\ja
code\_native(f,(Int,))
\jb
\>    .section    \_\_TEXT,\_\_text,regular,pure\_instructions \\
Filename: none \\
Source line: 1  \\
\>    push    RBP \\
\>    mov RBP, RSP  \\
Source line: 1 \\
\>    \sh{lea} RAX, QWORD PTR [RDI + 4*RDI - 1] \\
\>    pop RBP \\
\>    ret
\jc


The actual body of the function is a single \code{lea} instruction, which computes the integer multiply and add at once.
(Assembler names such as ```Load Effective Address," have grown irrelevant and one may think of LEA as simply an integer operation.)
The benefit of minimal code generation is even more beneficial when \code{f} gets inlined into the inner loop of  another function:


\ja
function g(k,n) \\
 \,        for i = 1:n \\
 \, \>          k = f(k) \\
 \,        end  \\
 \,        k  \\
       end
\jb
g (generic function with 2 methods)
\jc

\ja
code\_native(g,(Int,Int))
\jb
 \>   .section    \_\_TEXT,\_\_text,regular,pure\_instructions \\
Filename: none \\
Source line: 3 \\
\>    push    RBP \\
 \>   mov RBP, RSP \\
 \>   test    RSI, RSI \\
\>    jle 22 \\
 \>   mov EAX, 1 \\
Source line: 3 \\
 \>   \sh{lea} RDI, QWORD PTR [RDI + 4*RDI - 1] \\
 \>   inc RAX \\
 \>   cmp RAX, RSI \\
Source line: 2 \\
\>    jle -17 \\
Source line: 5 \\
\>    mov RAX, RDI \\
\>    pop RBP \\
    ret
\jc

Since the call to \code{f} gets inlined, the loop body ends up being just a single \code{lea} instruction. Next, consider what happens if we make the number of loop iterations fixed:

\ja
\sh{\# 10 Iterations of f(k)=5k-1 on integers} \\
 function g(k) \\
\,         for i = 1:10  \\
\, \>           k = f(k) \\
\,         end \\
\,         k  \\
       end
       \jb
g (generic function with 2 methods)
\jc

\ja
code\_native(g,(Int,))
\jb
\>    .section   \_\_TEXT,\_\_text,regular,pure\_instructions \\
Filename: none \\
Source line: 3 \\
 \>   push    RBP \\
\>    mov RBP, RSP  \\
Source line: 3 \\
\>    \sh{imul}    RAX, RDI, 9765625 \\
  \>   \sh{add} RAX, -2441406 \\
Source line: 5 \\
\>    pop RBP \\
 \>   ret \\
\jc

Because the compiler knows that integer addition and multiplication are associative and that multiplication distributes over addition it can optimize the entire loop down to just a multiply and an add.  Indeed, if $f(k)=5k-1$, it is true that the tenfold iterate $f^{(10)}(k)=-2441406 + 9765625 k$.

\subsection{A powerful approach to linear algebra}

\subsubsection{Matrix factorizations}

  For decades, orthogonal matrices have been represented internally as
  products of Household matrices stored in terms of vectors, and displayed for humans as matrix
  elements.  $LU$ factorizations are often performed in place,
  storing the $L$ and $U$ information together in the data locations
  originally occupied by $A$.  All this speaks to the fact that matrix
  factorizations deserve to be first class objects in a linear algebra
  system.

  In Julia, thanks to the contributions of Andreas Noack Jensen and
  many others, these structures are indeed first class objects.  The
  structure \verb+QRCompactWY+ holds a compact $Q$ and an $R$ in memory.
  Similarly an \verb+LU+ holds an $L$ and $U$ in packed form in memory.
  Through the magic of multiple dispatch, we can solve linear systems,
  extract the pieces, and do least squares directly on these
  structures.

  The $QR$ example is even more fascinating.  Suppose one computes $QR$ of
  a $5 \times 3$ matrix.  What is the size of $Q$?  The right answer, of
  course, is that it depends: it could be $5 \times 5$ or $5 \times 3$.  The
  underlying representation is the same.

  In Julia one can compute \verb+Aqr = qrfact(rand(5,3))+.  Then one
  can take \verb+Q=Aqr[:Q]+.  This $Q$ retains its clever underlying structure
  and therefore is efficient and applicable when multiplying vectors
  of length 5 or length 3, contrary to the rules of freshman linear
  algebra, but welcome in numerical libraries for saving space and
  faster computations.

\ja
 Aqr = qrfact(rand(5,3)); \\
 Q = Aqr[:Q] \jb
\begin{verbatim}
5x5 QRCompactWYQ{Float64}:
 -0.251536   0.154937   0.77616
 -0.101711  -0.608041   0.512436
 -0.755597   0.217326  -0.0987686
 -0.014734  -0.714761  -0.210326
 -0.596021  -0.219468  -0.284593
 \end{verbatim}
 \jc

\ja Q*rand(5) \jb
\begin{verbatim}
5-element Array{Float64,1}:
  0.755784
 -1.10069
 -0.757399
 -0.272477
 -0.0510777
\end{verbatim}
 \jc

\ja Q*rand(3) \jb \begin{verbatim}
5-element Array{Float64,1}:
  0.152596
 -0.0562855
 -0.569295
 -0.303167
 -0.644554
\end{verbatim}
\jc

\subsubsection{User-extensible wrappers for BLAS and LAPACK}

The tradition in linear algebra is to leave the coding to LAPACK
writers, and call LAPACK for speed and accuracy.  This has worked
fairly well, but Julia exposes considerable opportunities for
improvement.

Firstly, all of LAPACK is available to Julia users, not just the most
common functions. All LAPACK wrappers are implemented fully in Julia
code, using
``ccall''\footnote{\url{http://docs.julialang.org/en/latest/manual/calling-c-and-fortran-code/}},
which does not require a C compiler, and can be called directly from
the interactive Julia prompt.  This makes it easy for users to
contribute LAPACK functionality, and that is how Julia's LAPACK
functionality has grown bit by bit. Wrappers for missing LAPACK
functionality can also be added by users in their own code.

Consider the following example that implements the Cholesky
factorization by calling LAPACK's \code{xPOTRF}. It uses Julia's
metaprogramming facilities to generate four functions, each
corresponding to the \code{xPOTRF} functions for \code{Float32}, \code{Float64}, \code{Complex64},
and \code{Complex128} types. The actual call to the Fortran functions is
wrapped in {\tt ccall}. Finally, the {\tt chol} function provides a
user-accessible way to compute the factorization.  It is easy to modify
the template below for any LAPACK call.


\begin{jinput}
\sh {\# Generate  calls to LAPACK's Cholesky for double, single, etc.}  \\
\sh {\# xPOTRF refers to POsitive definite TRiangular Factor}  \\
\sh{\# LAPACK signature: SUBROUTINE DPOTRF( UPLO, N, A, LDA, INFO ) }  \\
\sh{\# LAPACK documentation:}
\vspace{-0.05in}
\begin{verbatim}
*  UPLO    (input) CHARACTER*1
*          = 'U':  Upper triangle of A is stored;
*          = 'L':  Lower triangle of A is stored.
*  N       (input) INTEGER
*          The order of the matrix A.  N >= 0.
*  A       (input/output) DOUBLE PRECISION array, dimension (LDA,N)
*          On entry, the symmetric matrix A.  If UPLO = 'U', the leading
*          N-by-N upper triangular part of A contains the upper
*          triangular part of the matrix A, and the strictly lower
*          triangular part of A is not referenced.  If UPLO = 'L', the
*          leading N-by-N lower triangular part of A contains the lower
*          triangular part of the matrix A, and the strictly upper
*          triangular part of A is not referenced.
*          On exit, if INFO = 0, the factor U or L from the Cholesky
*          factorization A = U**T*U or A = L*L**T.
*  LDA     (input) INTEGER
*          The leading dimension of the array A.  LDA >= max(1,N).
*  INFO    (output) INTEGER
*          = 0:  successful exit
*          < 0:  if INFO = -i, the i-th argument had an illegal value
 *                positive definite, and the factorization could not be
*                completed.
\end{verbatim} \\
\sh {\# Generate Julia method potrf!}  \\
\vspace{-.1in}
\verb+ for  (potrf,  elty) in + \sh{\# Run through 4 element types}
\vspace{.05in}
 \begin{verbatim}
     ((:dpotrf_,:Float64),
      (:spotrf_,:Float32),
      (:zpotrf_,:Complex128),
      (:cpotrf_,:Complex64))
      \end{verbatim}
      \vspace{-.2in}
      \sh {\# Begin function potrf!}
\vspace{-.07in}
           \begin{verbatim}
     @eval begin
         function potrf!(uplo::Char, A::StridedMatrix{$elty})
             lda = max(1,stride(A,2))
             lda==0 && return A, 0
             info = Array(Int, 1)
             \end{verbatim}
                   \vspace{-.2in}
             \sh {\# Call to LAPACK:ccall(LAPACKroutine,Void,PointerTypes,JuliaVariables)}
             \vspace{-.07in}
             \begin{verbatim}
             ccall(($(string(potrf)),:liblapack), Void,
                    (Ptr{Char}, Ptr{Int}, Ptr{$elty}, Ptr{Int}, Ptr{Int}),
\end{verbatim}
\vspace{-.07in}
\verb+                    +\sh{\ \ \ \&uplo, \ \ \ \&size(A,1), \ \ \  A, \ \ \ \ \ \  \ \&lda, \ \  info})
\vspace{-.07in}
                    \begin{verbatim}
             return A, info[1]
          end
     end
end

chol(A::Matrix) = potrf!('U', copy(A))
\end{verbatim}

\end{jinput}



\subsection{Easy and flexible parallelism}
\label{sec:easypar}

%\begin{figure}
%  \centering
%  \includegraphics[width=\columnwidth]{parallel.png}
%  \caption{\label{fig:parallel} Example of readily adding 400
%    processors from the Amazon EC2 cloud computing service. Bottom
%    left: the code run on 1 processor produces a fuzzy and unfocused
%    solution. Bottom right: the same code run on 400 processors, which
%    produces a sharp and focused solution.}
%\end{figure}










Parallel computing remains an important research topic in numerical
computing.  Parallel computing  has yet to reach the level of richness and
interactivity required for innovation that has been achieved with
sequential tools.  The  issues discussed in Section
\ref{sec:humancomputer} on the balance  between the human and the
computer become more pronounced in the parallel setting. Part of the
problem is that parallel computing means different things to different
people:
\begin{enumerate}
\item At the most basic level, one wants instruction level parallelism
  within a CPU, and expects the compiler to discover such parallelism
  in the code. In Julia, this can be achieved explicitly with the use
  of the {\tt @simd} primitive. Beyond that,
\item In order to utilize multicore and manycore CPUs on the same node, one
  wants some kind of multi-threading. Currently, we have experimental
  multi-threading support in Julia, and this will be the topic of a
  further paper. Julia currently does provide a {\tt SharedArray} data
  structure where the same array in memory can be operated on by
  multiple different Julia processes on the same node.
\item Then, there is distributed memory,
often considered the most difficult kind of parallelism.
This can mean running Julia
  on anything between half a dozen to thousands of nodes, each with
  multicore CPUs.
\end{enumerate}
In the fullness of time, there may be a unified programming model that
addresses this hierarchical nature of parallelism at different levels,
across different memory hierarchies.



\begin{figure}
\begin{center}
\includegraphics[width=4.5in]{starpoverview.png}
\end{center}
\caption{\label{starp}
Star-P parallel computing architecture.  Julia was built on the heels of our  parallel computing experience
with Star-P which began as an MIT research project
and was a software product of Interactive Supercomputing.  Our experience taught us that
bolting parallelism onto an existing language that was not designed for performance or parallelism is difficult at best, and impossible at worst. One of our (not so secret) motivations to
build Julia was to have the language we wanted for parallel numerical computing.}
\end{figure}

Our experience with Star-P \cite{starpright} taught us a
valuable lesson.
%Star-P parallelism included distributed dense and
%sparse array data structures, along parallel {\tt map} capabilities to
%evaluate a function in parallel over different data across different
%processors. The system came with a complete library of distributed
%array operations, parallel dense and sparse linear algebra, and many
%other common algorithms such as parallel sorting.
Star-P parallelism \cite{starpstart,starpug}  (see Figure \ref{starp})  included global
dense, sparse, and cell arrays that were distributed on parallel shared or distributed memory computers.  Before the evolution of the cloud as we know it today,
the user used a familiar front end (usually MATLAB) as the client on a laptop or desktop, and connected seamlessly to a server (usually a large distributed computer).
Blockbuster functions from sparse and dense linear algebra, parallel FFTs,
parallel sorting, and many others were easily available and composable for the user.
In these cases Star-P called Fortran/MPI or C/MPI.
Star-P also allowed a kind of parallel for loop that worked on rows, planes or hyperplanes
of an array.  In these cases Star-P used copies of the client language on the backend,
usually MATLAB, octave, python, or R.



Our experience taught us that while we were able to get a useful
parallel computing system this way, bolting parallelism onto an
existing language that was not designed for performance or parallelism
is difficult at best, and impossible at worst.  One of our (not so
secret) motivations to build Julia was to have the right language for
parallel computing.




Julia provides many facilities for parallelism, which are described in
detail in the Julia
manual\footnote{\url{http://docs.julialang.org/en/latest/manual/parallel-computing/}}. Distributed
memory programming in Julia is built on two primtives - {\it remote
  calls} that execute a function on a remote processor and {\it remote
  references} that are returned by the remote processor to the
caller. These primitives are implemented completely within Julia. On
top of these, Julia provides a distributed array data structure, a
{\tt pmap} implementation, and a way to parallelize independent
iterations of a loop with the {\tt @parallel} macro - all of which can
parallelize code in distributed memory. These ideas are exploratory in
nature, and will certainly evolve. We only discuss them here to
emphasize that well-designed programming language abstractions and
primitives allow one to express and implement parallelism completely
within the language, and explore a number of different parallel
programming models with ease. We hope to have a detailed discussion on
Juila's approach to parallelism in a future paper.




We proceed with one example that demonstrates {\tt @parallel} at work,
and how one can impulsively grab a large number of processors and
explore their problem space quickly.



Suppose we wish to perform a complicated histogram
in parallel. We use an example from Random Matrix Theory,
(but it could easily have been from finance),
the computation of the scaled largest eigenvalue in magnitude
of the so called stochastic Airy operator
$$\frac{d^2}{dx^2}-x + \frac{1}{2\sqrt{\beta}} dW.$$


This is just the usual finite difference  discretization of
$\frac{d^2}{dx^2}-x $ with a ``noisy" diagonal.


We illustrate an example of the famous
Tracy-Widom law being simulated with Monte Carlo experiments for
different values of the inverse temperature parameter $\beta$.  The code on 1 processor is fuzzy and
unfocused, as compared to the same simulation on 75 processors,
which is sharp and focused.

\ja
\verb& &\sh{ @everywhere} \verb&begin                       & \sh {\# define on every processor} \\
\verb& function stochastic(&$\beta$\verb&=2,n=200)  &\\
\verb&   h=n^-(1/3)  & \\
\verb&   x=0:h:10  & \\
\verb&   N=length(x) & \\
\verb&   d=(-2/h^2 .-x) +  2 sqrt&(h*$\beta$\verb&)*randn(N)& \sh{ \# diagonal} \\
\verb&   e=ones(N-1)/h^2                       &                   \sh {\# subdiagonal} \\
\verb&   eigvals(SymTridiagonal(d,e))[N]    & \sh   { \# smallest negative eigenvalue} \\
\verb&end&  \\
\verb&end&
\vspace{-.1in}
\end{jinput}



\ja
\vspace{-.1in}
\begin{verbatim}
t = 10000
\end{verbatim}
\vspace{-0.07in}
\verb&for &$\beta$\verb&=[1,2,4,10,20]& \\
%\vspace{-0.07in}
%\verb&@time z = &\sh{@parallel (+)}\verb& for i=1:nprocs()& \\
\vspace{-0.07in}
\verb& hist([stochastic(&$\beta$\verb&) for i=1:t], -4:.01:1)[2]&
%\vspace{-0.07in}
\begin{verbatim}
  plot(midpoints(-4:.01:1),z/sum(z)/.01)
end
\end{verbatim}
 \includegraphics[width=2.8in]{ScreenShot1.png}
\end{jinput}

\hspace{ 1in}

\ja
\vspace{-.1in}
\sh{\# Readily adding 75 processors sharpens the Monte Carlo simulation} \\
addprocs(75)
\vspace{-.1in}
\end{jinput}


\ja
\vspace{-.1in}
\begin{verbatim}
t = 10000
\end{verbatim}
\vspace{-0.07in}
\verb&for &$\beta$\verb&=[1,2,4,10,20]& \\
%\vspace{-0.07in}
\sh{z = {@parallel (+)} for i=1:nprocs()} \\
\vspace{-0.07in}
\verb&    hist([stochastic(&$\beta$\verb&) for i=1:t], -4:.01:1)[2]& \\
\sh {end}
\vspace{-0.07in}
\begin{verbatim}
plot(midpoints(-4:.01:1),z/sum(z)/.01)
end
\end{verbatim}
 \includegraphics[width=2.8in]{ScreenShot2.png}
\end{jinput}


\subsection{Performance Recap}

In the early days of high level numerical computing languages,  the thinking was that
the performance of the high level language did not matter so much so long
as most of the time was spent inside the numerical libraries.
These libraries consisted  of  blockbuster algorithms that would be highly tuned,
making efficient use of computer memory, cache, and low level instructions.

What the world learned was that
only a few codes were spending a majority of their time in the blockbusters.
Real codes were being caught by interpreter overheads, stemming from processing
more aspects of a program at run time than are strictly necessary.

As we explored in Section \ref{sec:types}, one of the hindrances of completing this analysis is
type information.  Programming language design thus becomes an exercise in
balancing incentives to the programmer to provide type
information and the ability of the computer to infer type information. Vectorization
is one such incentive system.  Existing numerical computing languages would have
us believe that this is the only system, or even if there were others, that somehow
this was the best system.

Vectorization at the software level can be elegant for some problems.
There are many matrix computation problems that look beautiful vectorized.
These programs should be vectorized. Other programs
require heroics and skill to vectorize sometimes producing unreadable code all in the name
of performance.  These are the ones that we object to vectorizing.  Still other programs
can not be vectorized very well even with heroics.
The Julia message is to vectorize when it is natural, producing nice code.  Do not vectorize
in the name of speed.

Some users believe that vectorizing software  is required to make use of special hardware
capabilities including the ability to use SIMD instructions, multithreading, GPU units,
and other forms of parallelism.  The Julia message remains: vectorize when natural, when you  feel it is right.



\section{Conclusion and Acknowledgments}


Julia was created to meet the needs of numerical computing.  At the
time of writing, not a day goes by where we don't learn that someone
else has picked up Julia at universities and companies around the
world, in fields as diverse as engineering, mathematics, physical and
social sciences, finance, biotech, and many others. More than just a
language, Julia has become a place for programmers, physical
scientists, social scientists, computational scientists,
mathematicians, and others to pool their collective knowledge in the
form of online discussions and in the form of code. Numerical
computing is maturing and it is exciting to watch!

We thank Michael La Croix for his beautiful Julia display macros.
The authors gratefully acknowledge support from the MIT Deshpande
center for numerical innovation, the Intel Technology Science Center
for Big Data, the DARPA Xdata program, the Singapore MIT Alliance, NSF
Awards CCF-0832997 and DMS-1016125, VMWare Research, a DOE grant with
Dr. Andrew Gelman of Columbia University for petascale hierarchical
modeling, and a Citibank grant for High Performance Banking Data
Analysis.

%We speculate that, historically, computer scientists developing multiple
%dispatch were not thinking about numerical computing, and those who cared
%about numerical computing were not interested in the obscurer corners of
%object-oriented programming. However, we believe that the combination
%is indeed creating a revolution and the best is yet to come.
%% multiple
%% dispatch is not merely an optimization hack, but instead can be used to design
%% a programming language where they become core semantic features. The
%% implementation of these features in Julia is a remarkably efficient way to
%% allow code to be specialized to leverage static analyses for performance, yet
%% at the same time allows for other code written for maximal flexibility even in
%% the absence of detailed type information. Static analyzability becomes no
%% longer a property of an entire language, but instead an optional property of
%% specific programs in a particular language. Julia is therefore sufficiently
%% expressive to write complex, generic, dynamic multidimensional behaviors, and
%% more generally allows us to address the performance---flexibility compromise
%% in Julia with greater Pareto optimality than other existing solutions.

%\appendix
%\section{Appendix Title}
%
%This is the text of the appendix, if you need one.

%\acks

%% The authors gratefully acknowledge the enthusiastic participation of the Julia
%% developer community in many stimulating discussions, in particular Dahua Lin and
%% Keno Fischer for the \code{ArrayViews.jl}\cite{Lin:2014av} and
%% \code{SIUnits.jl}\cite{Fischer:2014si} packages, respectively. This
%% work was supported by the MIT Deshpande Center, an Intel Science and
%% Technology award, grants from VMWare and Citibank, a Horizontal Software
%% Fellowship in Computational Engineering, and NSF DMS-1035400.

% We recommend abbrvnat bibliography style.



